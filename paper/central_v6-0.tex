\documentclass[11pt,reqno]{amsart} \usepackage{fullpage}
\usepackage{mathrsfs} \usepackage{latexsym,gensymb,marvosym}
\usepackage[dvips]{graphics} \usepackage[dvips]{graphicx}
\usepackage{epsfig} \usepackage{amsmath,amsfonts,amsthm,amssymb,amscd}
\usepackage{hyperref} \usepackage{url}
\usepackage[usenames,dvipsnames]{color} \usepackage{mathtools}
\usepackage{enumitem} \usepackage{fourier} \usepackage{calrsfs}
\DeclareMathAlphabet{\pazocal}{OMS}{zplm}{m}{n}
\usepackage[T1]{fontenc} \usepackage[utf8]{inputenc}

\setlength{\oddsidemargin}{0in}
\setlength{\topmargin}{0in}
\setlength{\textwidth}{6.5in}
\setlength{\textheight}{8.5in}

% INDENT FIX
\usepackage{etoolbox}

\makeatletter
% uncomment the following if you don't want \clubpenalty\@M ...
% \let\nearly@afterheading\@afterheading
% \patchcmd\nearly@afterheading
%   {\@M}% original temporary setting for \clubpenalty replaced by ...
%   {\@clubpenalty}% ... or whichever value you deem right
%   {}{}
% ... and use \nearly@afterheading instead of \@afterheading here:
\newcommand*\NoIndentAfterEnv[1]{%
  \AfterEndEnvironment{#1}{\par\@afterindentfalse\@afterheading}}
\makeatother

\NoIndentAfterEnv{itemize}
\NoIndentAfterEnv{theorem}
\NoIndentAfterEnv{defn}
\NoIndentAfterEnv{prop}
\NoIndentAfterEnv{corollary}
\NoIndentAfterEnv{remark}
\NoIndentAfterEnv{conjecture}
% treat other environments you want to patch accordingly

\newtheorem{fact}{Fact}
\newtheorem{lemma}{Lemma}
\newtheorem{theorem}[lemma]{Theorem}
\newtheorem{defn}[lemma]{Definition}
\newtheorem{assumption}[lemma]{Assumption}
\newtheorem{corollary}[lemma]{Corollary}
\newtheorem{prop}[lemma]{Proposition}
\newtheorem{exercise}[lemma]{Exercise}
\newtheorem{claim}[lemma]{Claim}
\newtheorem{remark}[lemma]{Remark}
\newtheorem{prob}{Problem}
\newtheorem{conjecture}{Conjecture}
\newtheorem{heuristic}{Heuristic}
\renewcommand{\theheuristic}{\Alph{heuristic}}

% Owen abhors hollow tombstones
\renewcommand{\qedsymbol}{{\footnotesize$\blacksquare$}}

% Fourier-provided amssymb commands
\renewcommand{\leq}{\leqslant}
\renewcommand{\geq}{\geqslant}
\renewcommand{\nleq}{\nleqslant}
\renewcommand{\ngeq}{\ngeqslant}
\renewcommand{\subsetneq}{\varsubsetneq}

% important rings
\newcommand{\R}{\ensuremath{\mathbf{R}}}
\newcommand{\C}{\ensuremath{\mathbf{C}}}
\newcommand{\Q}{\ensuremath{\mathbf{Q}}}
\newcommand{\F}{\ensuremath{\mathbf{F}}}
\newcommand{\N}{\ensuremath{\mathbf{N}}}
\newcommand{\Z}{\ensuremath{\mathbf{Z}}}
\newcommand{\A}{\ensuremath{\mathbf{A}}}

% greek
\newcommand{\ga}{\alpha}
\newcommand{\gb}{\beta}
\newcommand{\ep}{\epsilon}
\newcommand{\vep}{\varepsilon}

% shortcuts
\newcommand{\ra}{\rightarrow}
\renewcommand{\d}[1]{\,\operatorname*{d}\!{#1}}
\newcommand{\dd}[2][]{\frac{\operatorname*{d}^{#1}}{\operatorname*{d}\! {#2}^{#1}}}
\newcommand{\pp}[2][]{\frac{\partial^{#1}}{\partial {#2}^{#1}}}
\newcommand{\yell}[1]{\textbf{#1}}
\newcommand\be{\begin{equation}}
\newcommand\ee{\end{equation}}
\newcommand\bi{\begin{itemize}}
\newcommand\ei{\end{itemize}}
\newcommand\ben{\begin{enumerate}}
\newcommand\een{\end{enumerate}}
\newcommand{\scnumeral}[1]{\textsc{#1}}
\newcommand{\eg}{\noindent\textit{Example}\quad}
\newcommand{\question}{\noindent\textit{Q}\quad}
\renewcommand{\H}{\mathfrak{H}}
\newcommand{\pc}[1]{\pazocal{#1}}
\newcommand{\nb}{\textsc{n.b.}}
\newcommand{\etal}{\textit{et al.}}

\newcommand{\GL}{\operatorname{GL}}
\newcommand{\nstd}{N_{\text{std}}}
\newcommand{\neff}{N_{\text{eff}}}
\newcommand{\cstd}{c_{\text{std}}}
\newcommand{\ceff}{c_{\text{eff}}}
\newcommand{\Lsym}[2][]{L_f#1\left(#2,\sym^2\right)}
\newcommand{\Lsymbar}[2][]{L_{\overline f}#1\left(#2,\sym^2\right)}
\newcommand{\Lad}[1]{L_f\left(#1,\ad^2\right)}
\newcommand{\Lchi}[2][]{L(#2,\chi_f#1)}
\newcommand{\Lchibar}[2][]{L\left(#2,\chi_{\overline f}#1\right)}
\DeclareMathOperator{\res}{res}
\DeclareMathOperator{\sym}{sym}
\newcommand{\symres}{r_{\sym^2f}}
\DeclareMathOperator{\ad}{ad}
\DeclareMathOperator{\Prob}{Prob}
\newcommand{\D}{\mathcal D_f}
\newcommand{\E}{\mathcal E_f}
\newcommand{\T}{\pazocal T_{f\otimes\psi_d}(\alpha,\beta,\gamma,\delta)}
\renewcommand{\F}{\mathcal F_f}
\renewcommand{\a}{{\mathfrak f}}
\renewcommand{\L}[1]{L_{f\otimes\psi_d}\left(#1\right)}
\newcommand{\gf}{\Phi_d}
\newcommand{\gfactor}[2][]{\gf#1\left(#2\right)}
\newcommand{\Lld}[1]{\frac{L_{f\otimes\psi_d}'}{L_{f\otimes\psi_d}}\left(#1\right)}
\DeclareMathOperator{\Nm}{N}
\newcommand{\Lunram}{L_{\star}}
\newcommand{\Lram}{L_{\text{ram}}}
\numberwithin{equation}{section}
\begin{document}

\title{An excised orthogonal model for families of cusp forms}
\date{Version 6.0}
\author{Owen Barrett, Steven J. Miller}
\thanks{The authors would like to thank James Cogdell, Zhengyu Mao, and Jeremy Rouse for
helpful discussions.}
\begin{abstract}
  The Katz-Sarnak philosophy dictates that the spacing statistics of the zeros of
  Hecke cusp form $L$-functions averaged within families coincide with those of
  eigenvalues of particular random matrices in the limit as the level tends to infinity.
  While these predictions have been supported by multiple density and correlation results
  across various families, S.\,J. Miller discovered a significant disagreement for finite
  conductors in the numerical data for elliptic curve $L$-functions. We use the
  $L$-functions ratios conjectures to calculate the 1-level density for the family
  of even quadratic twists of a holomorphic cuspidal newform for large but finite level
  and recover leading and lower-order terms. Then, using the arithmetic of these
  lower-order terms and the discretization of the $L$-functions at the central point,
  we extend a random matrix model proposed by Dueñez, Huynh, Keating, Miller, and Snaith
  for elliptic curves to Hecke cusp forms of weight $k>2$.
\end{abstract}

\maketitle

\tableofcontents

\vspace{.25in}

\section{Introduction}
In conversation with Freeman Dyson at tea at the Institute for Advanced Study shortly after
their introduction by Chowla one propitious afternoon in 1971, Hugh Montgomery learned that
his calculation of the pair-correlation of zeros of $\zeta(s)$ matched the pair-correlation
of eigenvalues of large unitary matrices.

In the intervening years, a general philosophy has developed in support of the notion that
spacing statistics of critical zeros of $L$-functions should be in agreement with
the corresponding statistic attached to associated random matrix ensembles in the correct
asymptotic limit \cite{hejhal,RS,KaSa1,KaSa2,ILS}.

While the general philosophy connecting zero statistics to eigenvalue statistics can yield
remarkable predictive insights for both local statistics (on the scale of only several mean
spacings) and global statistics, such as moments, classic matrix ensembles fail to reflect
some finer statistical properties of $L$-function zeros, as S.J. Miller~\cite{mil06}
observed in 2006. He discovered that, localized at the central point, elliptic curve
$L$-function zero statistics deviated qualitatively from the random matrix model.

Subsequently, Dueñez, Huynh, Keating, Miller, and Snaith~\cite{DHKMS} sought to craft a
refined matrix ensemble (the \textit{excised orthogonal ensemble}) that would more
accurately reflect the phenomena Miller observed. 
One key discrepancy between the arithmetic statistics of $L$-function zeros and the
statistics of random matrices is described by a theorem of Kohnen and Zagier~\cite{KZ},
which implies a discretization of $L(s,f)$ at the central
point that justifies the excised orthogonal model.

\begin{theorem}[Kohnen-Zagier~\cite{KZ}]\label{KZthm}
  Let $f\in S_k$ be a normalized Hecke eigenform, $g\in S_{(k+1)/2}^+$ a form
  corresponding to $f$ as above, $d$ a fundamental discriminant with $(-1)^kd>0$,
  and $L(s,f_d)$ the $L$-series of $f$ twisted by the quadratic character
  with fundamental discriminant $d$, which is defined by analytic
  continuation of the Dirichlet $L$-series
  $\sum_{n\geq1}\left(\frac d n\right)a(n)n^{-s}$. Then
  \be L_f(k,\psi_d) = \kappa_f \frac{c(|d|)^2}{|d|^{(k-1)/2}},\quad\text{where }
  \kappa_f = \frac{(k-1)!}{\pi^{k/2}}\frac{\langle f,f\rangle}{\langle g,g\rangle},\ee
  where $\langle\cdot,\cdot\rangle$ denotes the Petersson inner product,
  $c(|d|)$ is an integer, and $k$, the weight, also denotes the central point in
  the evaluation of $L_f(s,\psi_d)$.
\end{theorem}
Theorem~\ref{KZthm}, as stated, only holds for level 1 newforms. Baruch-Mao \cite{BM}
extend to square-free odd level and Mao~\cite{mao} extends to arbitrary odd level,
with modifications to $\kappa_f$ in both cases.

The philosophy of Dueñez~\etal\ is that values of $L$-functions at the
central correspond to characteristic polynomials of matrices evaluated at the point $1$.
Since the number $c(|d|)$ is an integer, Theorem~\ref{KZthm} implies that the value at
the central point is discretized on the order of $|d|^{(1-k)/2}$. Therefore, to model
the behavior of low-lying zeros using matrices, one should attempt to account for this
discretization at the central point with a discretization of the values of the
characteristic polynomials at $1$.
This provides a key ingredient to their recipe, the other being a modification to the
\textit{matrix size} of the ensemble.

While Dueñez~\etal\ seek to model $L$-functions attached to elliptic curves, the
Kohnen-Zagier formula in Theorem~\ref{KZthm} holds more generally for Hecke eigenforms.
The purpose of this work is to generalize the random matrix model of Dueñez~\etal\ to
Hecke eigenforms.

Given a form in $S_k^\ast(M,\chi)$, twisting by a character $\psi_d$,
where $d$ is a fundamental discriminant, produces a form in
$S_k^\star(Md^2,\chi)$. Hence, our density of zeros near the central point for a given
form is $\asymp \log d$, and we obtain the same relation
$\nstd\sim\log d$. Now, the values at the central point are no longer
discretized on a scale of $1/\sqrt d$; they are discretized on a scale of
$1/d^{(k-1)/2}$. Therefore, we excise characteristic polynomials whose value at 1 is of the
scale $\exp((1-k)\nstd/2)$. If we restrict to even forms, we now want
matrices in $SO(2N)$ satisfying
\be\label{eq:exciseddef} |\Lambda_A(1,N)|\geq \exp \mathcal X = c \cdot\exp((1-k)\nstd/2). \ee
The cutoff value $c$ will be discussed in~\S\ref{sec:cutoff}.
We also introduce an `effective' matrix size
$\neff$, based on lower-order terms of the one-level density for our family of Hecke cusp
forms, which we calculate below, and an effective cutoff value.

\section{Effective matrix size}\label{sec:neff}
As the asymptotic parameter for a family of $L$-functions tends to infinity, the
Katz-Sarnak philosophy dictates that the zero statistics, averaged across the family,
should tend to the scaling limit of eigenvalues of random matrices drawn from an
appropriate classical matrix group. However, for finite values of the asympototic
parameter, there is a deviation from the limit of number-theoretic origin.
These deviations need not generally cause qualitatively different phenomena than in the
limit, but S.J. Miller's 2006 observation~\cite{mil06} that there is a repulsion of zeros
away from the central point in elliptic curve families is an example of behavior that is
qualitatively different from that in the limit.

It is general philosophy that the arithmetic in families of $L$-functions is
absent in the main term of the limit, but is present in lower-order terms of the asympototic
expansion in the asymptotic parameter of the family. Since we are interested in modeling
the effect of the discretization at the central point of Hecke cusp forms on low-lying
zeros, motivated by Theorem~\ref{KZthm} and Miller's numerical observations in the elliptic
curve case, we would like to tune our random matrix model to the arithmetic of the
lower-order terms for our family, in the spirit of Bogomolny, Bohigas, Leboeuf, and
Monastra~\cite{bblm} as applied by Dueñez~\etal~\cite{DHKMS} in the elliptic curve case.

We will tune the matrix size of our ensemble to take into account the arithmetic of the
lower order terms of an asymptotic expansion of the one-level density formula for our
family, which is the family of even quadratic twists of a fixed holomorphic cuspidal
newform $f$; i.e., primitive Hecke cusp form.

We recall, following~\cite{conreymatrix},
the definition of the $1$-level density of a matrix ensemble $G(N)$.
For us, $G(N)$ stands for one of the matrix groups
$U(N), USp(2N)$, or $SO(2N)$. The definition of the (unscaled) $1$-level density is
\be\label{eq:matrix1lddef} \int_{G(N)}\sum_{j=1}^N \varphi(\theta_j)dX.\ee
As a consequence of Weyl's integration formula and Gaudin's 
lemma,~\eqref{eq:matrix1lddef} can be explicitly computed as
\be\int\varphi(\theta)S_{G(N)}(\theta)d\theta,\ee
where the bounds of integration are $[0,2\pi]$ for $G(N)=U(N)$, and $[0,\pi]$ for
$G(N)=USp(2N),SO(2N)$, and
\begin{align}
  S_{SO(2N)}(\theta)&\ =\ \frac{2N-1}{2\pi}+\frac{\sin(2N-1)\theta}{2\pi\sin\theta}, \\
  S_{U(N)}(\theta)&\ =\ \frac N{2\pi}, \\
  S_{USp(2N)}(\theta)&\ =\ \frac{2N+1}{2\pi}-\frac{\sin(2N+1)\theta}{2\pi\sin\theta}.
\end{align}
We call these functions the unscaled $1$-level density functions for these matrix groups.

The mean spacing of the eigenangles $\theta_j$ of matrices $X\in G(N)$ depends on $N$.
Therefore, for making statistical comparisons (such as with zeros of $L$-functions), it
is natural and convenient to scale the eigenangles to have mean spacing one.

The scaling densities for the ensembles $SO(2N)$ and $USp(2N)$ are both $N/\pi$.
The scaling density for $U(N)$ is $N/2\pi$.
It is then the case that the asymptotic scaled $1$-level density for large $N$ is given
by
\be\int_{\R_+}\varphi(\theta)R_{G(N)}(\theta)d\theta,\ee
where $R_{G(N)}(\theta)$ is obtained from $S_{G(N)}(\theta)$ by scaling the one-level
density formulas and expanding in powers of $1/N$:
\begin{align}
  R_{SO(2N)}(y)\ :=\ \frac \pi N S_{SO(2N)}(y\pi/N)&\
  \begin{aligned}[t]&\ =\ 1-\frac1{2N}
    +\frac{\sin\left[(2N-1)y\pi/N\right]}{2N\sin(y\pi/N)} \\
    &\ =\ 1+\frac{\sin(2\pi y)}{2\pi y}-\frac{1+\cos(2\pi y)}{2N}-\frac{\pi y\sin(2\pi y)}{6N^2}
    +O\left(N^{-3}\right),\end{aligned} \\
  R_{USp(2N)}(y)\ :=\ \frac\pi N S_{USp(2N)}(y\pi/N)&\
  \begin{aligned}[t]&\ =\ 1+\frac1{2N}-\frac{\sin[(2N+1)y\pi/N]}{2N\sin(y\pi/N)} \\
    &=1-\frac{\sin(2\pi y)}{2\pi y}+\frac{1-\cos(2\pi y)}{2N}
    +\frac{\pi y\sin(2\pi y)}{6N^2}+O\left(N^{-3}\right),
  \end{aligned} \\
  R_{U(N)}(y)\ :=\ \frac {2\pi}N S_{U(N)}(2 y\pi/N)&\ =\ 1.
\end{align}
We call these functions the scaled $1$-level density functions, or simply the $1$-level
density functions, for these matrix groups.

The standard approach to choose the matrix size is to equate mean densities of
eigenvalues with mean density of zeros; in our case, this would imply we should choose
\be \nstd = R \ :=\  \log \left(\tfrac{\sqrt M X}{2\pi e}\right),\ee
where $M$ is the level of our cusp form $f$.
The approach of~\cite{bblm,DHKMS} is to multiply $\nstd$ obtained in this way by a
constant of arithmetic origin so as to see agreement in lower-order terms of the
two-point correlation statistic or the one-level density, respectively.
Happily, it occurs that by correcting the matrix size in this way to agree with one of
these simpler statistics, the authors find better agreement for all correlation
functions, and therefore also for the nearest-neighbor spacing.

We will consider families of $L$-functions $\F$ which we define rigorously in
\S\ref{sec:expectations}, Definition~\ref{def:F}. Loosely speaking, they are all families
of quadratic twists of a fixed newform $f$. One novelty of our computations is that
we find a variety of different symmetry types for $\F$ depending on different qualities
of $f$. We use the Ratios conjectures of~\cite{recipe} to compute formulas
for the one-level density of $\F$ for a variety of different $f$, so as to have lower
terms for the zero statistics of our $L$-functions to match the random matrix theory
asymptotics against so as to produce an effective matrix size. For the family with
unitary symmetry, there are no lower-order terms against which to match. Therefore, we
compute a formula for the pair-correlation of zeros of $L(s,f\otimes\psi_d)$ for such an
$f$ giving rise to a family with unitary symmetry with a fixed twist, where our
asymptotic parameter is now $T$, the height along the critical line. Averaging over
`good' fundamental discriminants $d\in\D$ (the meaning of which is made precise in
\S\ref{sec:expectations}, Definition~\ref{def:D}), we obtain a suggested effective matrix
size for the family with unitary symmetry.

We excerpt a part of Corollary~\ref{cor:expansion} to illustrate how to arrive at the
effective matrix size for two of our families. If $f$ is a newform of (even) weight $k$
and (odd prime) level $M$, and with nebentypus $\chi_f$, then the one-level density for
the scaled zeros of the family $\F$ given in Definition~\ref{def:F} parametrized by
the set $\D$ of fundamental discriminants described in Definition~\ref{def:D} is

\begin{align}
    \frac1 {\left|\D(X)\right|} S_1(f) &=
    \int_{-\infty}^\infty g(\tau)
    \left(1+\frac{\sin(2\pi\,\tau)}{2\pi\,\tau}
      +a_1\frac{1+\cos(2\pi\,\tau)}{R}
      -a_2\frac{\pi\,\tau\,\sin(2\pi\,\tau)}{R^2}
      +O\left(R^{-3}\right)\right)\d\tau
    \intertext{if $\chi_f$ is principal, and}
    \frac1 {\left|\D(X)\right|} S_1(f) &=
    \int_{-\infty}^\infty g(\tau)
    \left(
      1-\frac{\sin(2\pi\,\tau)}{2\pi\,\tau}
      +d_1\frac{1-\cos(2\pi\,\tau)}R
      +d_2\frac{\pi\,\tau\sin(2\pi\,\tau)}{R^2}
      + O\left(R^{-3}\right)
    \right)\d\tau,
\end{align}
if $\chi_f$ is not principal and $f=\overline f$, for a variety of arithmetic constants
which are defined in \S\ref{sec:1ld}.
By matching the lower-order terms of these formul\ae\ with those of the scaled one-level
density functions, we find that
$\neff$ for the principal character family with $SO(2N)$ symmetry is
$R/2a_1$, and $\neff$ for the self-dual CM form family with $Sp(2N)$ symmetry is $R/2d_1$,
where the constants $a_1$ and $d_1$ appear above and are assigned at~\eqref{eq:a1def}
and~\eqref{eq:d1def}.

As mentioned above, we can't suggest an effective matrix size using the one-level density
for the generic family (non-principal character, not self-dual), since there are no
lower-order terms to match in the one level density for unitary matrices.
To remedy the situation, we perform an analysis of the pair-correlation for that family,
and discuss the choice of effective matrix size for such a family in \S\ref{sec:pcthm}.

\section{Cutoff value}\label{sec:cutoff}
We now investigate how best to optimize the cutoff value $c$ in our
definition~\eqref{eq:exciseddef} of the excision threshold. Our point of departure is the
work of Conrey, Keating, Rubinstein and Snaith~\cite{CKRS1,CKRS2} as
modified by Dueñez, Huynh, Keating, Miller and Snaith~\cite{DHKMS}.

We begin by introducing the definition of a moment generating function $M_f(X,s)$ for the
values at the central point of $L$-functions in our family $\mathcal F^+_f(X)$, the family
of even twists of the fixed holomorphic Hecke cusp form $f$ by quadratic characters
$\psi_d$ corresponding to fundamental discriminants $d$ in the range $0<d\leq X$
satisfying certain properties. We denote these `good' fundamental discriminants by
$\D(X)$; therefore, the cardinality of our family is $\left|\D(X)\right|$ as a function
of the asymptotic parameter $X$.
Define \be M_f(X,s):=\frac1{\left|\D(X)\right|}\sum_{d\in\D(X)}L_f(1/2,\psi_d)^s.\ee
We expect that values at the central point of $L$-functions averaged over a family
behave like values of characteristic polynomials evaluated at 1, averaged over the
appropriate family of random matrices. Therefore, as in~\cite{rmtzeta} and~\cite{rmtL}, we
posit an asymptotic equivalence of moment generating functions
\be\label{eq:momentequivalence} M_f(X,s)\sim a_f(s)M_{W(f)}(N,s),\ee where $W^+(f)$ is the
symmetry type of the family $\F^+(X)$, and $M_{W^+(f)}(N,s)$ is the corresponding
moment generating function
\be\label{eq:mgfdef} M_{W^+(f)}(N,s)\ =\ \int_{W^+(f)}\Lambda_A(1,N)^s\d A.\ee
Our one-level
density calculations indicate that $W^+(f)$ assumes the following values depending on $f$:
\be\label{eq:wfdef}W^+(f)=\begin{cases}SO(2N)&\chi_f\text{ principal}, \\
  U(N)&\chi_f\text{ non-principal, }f\ne\overline f, \\
  Sp(2N)&\chi_f\text{ non-principal, }f=\overline f.\end{cases}\ee
For now, we restrict to the first case, the case of principal character, which is the most widely-considered. We now need to explain the arithmetic function $a_f(s)$.
The origin of this arithmetic factor is in~\cite{recipe}, where it appears in conjectured
formulas for moments of $L$-functions. In particular, we refer to their recipe in \S4 for
conjecturing moments, referring also to \S4.3, equation (4.4.22), \S5.3, and Theorem 2.4.1.
Let us restrict now to $\chi_f$ principal so that we may refer to \S4.3 directly. Then,
start from their equation~(4.4.20), and excise the zeta factors as per Theorem 2.4.1.
This is the arithmetic factor $A_m(s;r_1,\ldots,r_m)$ in Theorem 2.4.1., where the $r_i$
are unspecified small shifts that we will soon let pass to 0;
$A_m$ specializes to the case of even twists of a primitive real $L$-function like ours as
\be A_m(s;r_1,\ldots,r_m)=\prod_p R_{m,p}(s)\prod_{1\leq i<j\leq k}
\left(1-\frac1{p^{2s+r_i+r_j}}\right),\ee where \be R_{m,p}(s)=\left(1+\frac1p\right)^{-1}
\left(\frac1p+\frac12\left[\prod_{j=1}^m \mathcal L_p\left(\frac1{p^{s+r_j}},f\right)+
    \prod_{j=1}^m \mathcal L_p\left(\frac{-1}{p^{s+r_j}},f\right)\right]\right)\ee when
$p\nmid M$, and
\be R_{m,p}(s)=\prod_{j=1}^m\mathcal L_p\left(\frac{\ep_f}{p^{s+r_j}},f\right)\ee
when $p=M$. Here, $\mathcal L_p(s,f)$ is not $L_p(s,f)$, the local factor of $L_f(s)$ at
$p$, but rather given implicitly by \be L_f(s)=\prod_p L_p(s,f)
=\prod_p\left(1-\lambda_f(p)p^{-s}+\chi_f(p)p^{-2s}\right)^{-1}
=:\prod_p\mathcal L_p\left(p^{-s},f\right).\ee
This expression for $A_m$ is valid only for a primitive real $L$-function with orthogonal
symmetry, which is what we have in the case that $\chi_f$ is principal, which we have
restricted ourselves to. When $d>0$, $\chi_d(-1)=1$, and so we require $\psi_d(M)=\ep_f$
so that the product $\psi_d(-M)\vep_f=+1$ and the functional equation of
$L(s,f\otimes\psi_d)$ is even. This explains why the local factor at the sole ramified
prime $M$ is evaluated at a fraction with $\ep_f$, the sign of the functional equation of
$f$, in the numerator, as opposed to $\pm\ep_f$ or otherwise. Setting $s=1/2$ and $r_i=0$,
we make the definition
\begin{multline} a_f(m):=A_m(1/2;0,\ldots,0) \\
=\left[\prod_p\left(1-\frac1p\right)^{m(m-1)/2}\right]
\times\left[\prod_{p\nmid M}\left(1+\frac1p\right)^{-1}\left(\frac1p
    +\frac12\left[\mathcal L_p\left(p^{-1/2},f\right)^m
      +\mathcal L_p\left(-p^{-1/2},f\right)^m\right]\right)\right]
\times \mathcal L_M\left(\frac{\ep_f}{M^{1/2}},f\right)^m.\end{multline}
This expression is specified to our case of even twists of a holomorphic Hecke cusp form
with principal character and prime level, and was first conceived for integral $m$, but we
will repurpose it for fractional $m$.

We now follow the recipe of~\cite[\S5]{DHKMS}. Since we have restricted to principal
$\chi_f$, $W^+(f)=SO(2N)$; let $M_{O^+}(N,s):=M_{SO(2N)}(N,s)$ as defined
in~\eqref{eq:mgfdef}. Then $M_{O^+}(N,s)$ is the moment generating function of the values
$\left|\Lambda_A(1,N)\right|$ as $A$ varies in the random matrix ensemble $SO(2N)$, and
$M_{O^+}(N,s)$ can be explicitly evaluated for $\Re s>-1/2$~\cite{rmtL} as
\be\label{eq:MOexplicit} M_{O^+}(N,s)=\int_{SO(2N)}\Lambda_A(1,N)^s\d A
=2^{2Ns}\prod_{j=1}^N\frac{\Gamma(N+j-1)\Gamma(s+j-1/2)}{\Gamma(j-1/2)\Gamma(s+j+N-1)}.\ee
Since the moment generating function of a random variable is related to the probability
density function by the two-sided Laplace transform, equation~\eqref{eq:MOexplicit} allows
us to calculate the distribution of the values the characteristic polynomials in the
ensemble assume at 1. Explicitly, we denote the probability density for values
$\Lambda_A(1,N)$ with $A\in SO(2N)$ by $P_{O^+}(N,x)$. For $c>0$, we then have
\be\label{eq:rmtlaplace} P_{O^+}(N,x)=\frac1{2\pi ix}\int_{(c)}M_{O^+}(N,s)x^{-s}\d s.\ee
For small $x>0$, which is our regime of interest, the simple pole of~\eqref{eq:MOexplicit}
contributes the main term to $P_{O^+}(N,x)$; we therefore have
\be\label{eq:POasymp} P_{O^+}(N,x)\sim x^{-1/2}h(N),\ee where
\be h(N):=\res_{s={-1/2}}M_{O^+}(N,s)=2^{-N}\Gamma(N)^{-1}\prod_{j=1}^N
\frac{\Gamma(N+j-1)\Gamma(j)}{\Gamma(j-1/2)\Gamma(j+N-3/2)}.\ee
We also have the asymptotic in large $N$
\be\label{eq:hNasymp} h(N)\sim2^{-7/8}G(1/2)\pi^{-1/4}N^{3/8},\ee
where $G$ is the Barnes $G$-function~\cite{barnes}.
Remarking that $P_{O^+}(Nx,)\d x$ is the probability that a characteristic polynomial
$\Lambda_A(1,N)$ for $A\in SO(2N)$ attains a value between $x$ and $x+\d x$ and
leveraging~\eqref{eq:POasymp} gives
\be\label{eq:probasymp}\Prob\left(0\leq\Lambda_A(1,N)\leq\rho\right)
=\int_0^\rho P_{O^+}(N,x)\d x\sim\int_0^\rho x^{-1/2}h(N)\d x=2\rho^{1/2}h(N),\ee where the
middle asymptotic is valid in the regime of small positive $x$.

With the above explicit asymptotic for the probability the characteristic polynomial of
a random matrix in $SO(2N)$ attains a value at 1 in $[0,\rho]$, the idea in~\cite{DHKMS} is
then to obtain a formula for the corresponding probability in our family $\mathcal F^+_f(X)$,
and then to equate the two probabilities. To this end, we introduce a new probability
density $P_f(d,x)$, which, by~\eqref{eq:momentequivalence}, we expect to serve as a kind of
smooth approximation to the discretization at the central point of the forms in our family
corresponding to twists by fundamental discriminants near $d$. As in~\eqref{eq:rmtlaplace},
we define\be\label{eq:Pfdef}
P_f(d,x)\ :=\ \frac1{2\pi ix}\int_{(c)}a_f(s) M_{O^+}(\log d,s)x^{-s}\d x
\sim a_f(-1/2)P_{O^+}(\log d,x),\ee where the last asymptotic is valid for small
$x$ and is obtained by pulling the line of integration left past the pole at $s=-1/2$.

Recalling Theorem~\ref{KZthm}, we have a discretization at $s=1/2$ for
$f\otimes\psi_d\in\F^+(X)$ of the form ($d>0$) \be L_f(1/2,\psi_d)
\ <\ \kappa_f\frac{|c(d)|^2}{d^{(1-k)/2}}\ \Rightarrow\ L_{f}(1/2,\psi_d)\ =\ 0.\ee
If one were interested in predicting the frequency of vanishing of our family, then, 
it would be natural to calculate the probability that a random variable $Y_d$ with
probability density $P_f(d,x)$, assumes a value less than $\kappa_f|c(d)|^2 d^{(1-k)/2}$,
then sum asymptotically over the family.
This approach, pioneered by Conrey, Keating, Rubinstein, and Snaith in~\cite{CKRS1}
and~\cite{CKRS2}, runs into the essential obstacle that in order to obtain an asymptotic
for the frequency of vanishing with the correct leading coefficient, we need to have some
idea of the statistical behavior of the coefficient $c(d)$ evaluated at the subset of
fundamental discriminants $d$ we are considering. (In this article, when $\chi_f$ is
principal, we are choosing only those positive $d$ that convolve with $f$ to produce an
even form.)

The coefficients $c(d)$ are the Fourier coefficients of a half-integral weight
modular form that is obtained via a generalized Shimura correspondence.
It was Waldspurger~\cite{wald1,wald2} who related the Fourier coefficients $c(n)$ of
this half-integral weight form obtained via the Shimura correspondence to the central
values of $L$-functions arising from quadratic twists of an integral weight form.
Kohnen and Zagier~\cite{KZ} then used this Waldspurger formula to establish a
discretization at the central point of the functions $L_f(s,\psi_d)$, where $f$ is a cusp
form for the full modular group.
In the case that the level $M$ is instead an odd prime and the character $\chi_f$ is
principal, which is a case we consider here, the Waldspurger-type correspondence and
formula is due to Kohnen~\cite{kohnen}. More recently, Baruch and Mao~\cite{BM} obtained a
Waldspurger-type formula valid for automorphic forms over a totally real field, and used it
to remove some conditions on the fundamental discriminants in the formula of Kohnen and
Zagier. Mao~\cite{mao} then extended the correspondence to arbitrary odd level.

Though a notion of Shimura correspondence exists for $\chi_f$ non-principal, there is no
analogous formula of Waldspurger type. For this reason, it is not known whether the values
$L_f(1/2,\psi_d)$ are discretized in the case $\chi_f$ is not principal. Restricting to
$\chi_f$ principal, the discretization is known, but the arithmetic of the coefficients
$c(n)$ is not well-understood. The authors in~\cite[\S4]{CKRS2} use conjectures arising
from random matrix theory for the value distribution of elliptic curve $L$-functions to
make a series of conjectures on the statistics of these coefficients $c(d)$. However, one
would not expect the more complex arithmetic of the coefficients $c(d)$ to be accessible to
random matrix theory, and indeed predicting the coefficient of the main term of the
frequency of vanishing of families of elliptic curve $L$-functions related to the families
we consider here remains out of reach.

It is for this reason that the authors in~\cite{DHKMS} determine this value numerically;
i.e., they introduce a notion of an `effective' cutoff depending on a parameter that is
determined numerically, but which does not depend on $d$. We follow, writing
$\delta_f\kappa_f d^{(1-k)/2}$ for our `effective' cutoff, where $\delta_f$ is a numerical
input. Then, in analogy with~\eqref{eq:probasymp}, we write
\be\begin{aligned}\Prob\left(0\leq Y_d\leq\frac{\delta_f\kappa_f}{d^{(k-1)/2}}\right)
  &\sim\int_0^{\delta_f\kappa_fd^{(1-k)/2}}a_f(-1/2)x^{-1/2}h(\log d)\d x \\
  &=2a_f(-1/2)\sqrt{\delta_f\kappa_f} d^{(1-k)/4}h(\log d).\end{aligned}\ee
Following~\cite{CKRS1}, we then conjecture that
\be\label{eq:probvanish}
\left|\left\{L_f(s,\psi_d)\in\F^+(X):d\text{ prime, }L_f(1/2,\psi_d)=0\right\}\right|=\left.\sum_{\mathclap{\substack{d\leq X\\d\text{ prime}}}}\right.^\ast
\Prob\left(0\leq Y_d\leq\frac{\delta_f\kappa_f}{d^{(k-1)/2}}\right),
\ee where the starred sum in the middle line means that the sum is only over those prime
fundamental discriminants such that $f\otimes\psi_d$ has an even functional equation, of
which there are asymptotically $X/4\log X$.
The convergence of the sum~\eqref{eq:probvanish} is determined by $k$. Namely, if $k=2$ or
$4$, the sum diverges, and we have the asymptotic
\be\begin{aligned}
  \eqref{eq:probvanish}
  &\sim\frac1{4\log X}\sum_{n=1}^{\lfloor X\rfloor}2a_f(-1/2)\sqrt{\delta_f\kappa_f}
  n^{(1-k)/4}h(\log X) \\
  &\sim\frac1{4\log X}\cdot2a_f(-1/2)\sqrt{\delta_f\kappa_f}2^{-7/8}G(1/2)\pi^{-1/4}
  (\log X)^{3/8}\frac4{5-k}X^{(5-k)/4},\end{aligned}\ee
where the asymptotic in the last line comes from~\eqref{eq:hNasymp}.

On the other hand, if $k>4$, then the sum~\eqref{eq:probvanish} converges, and we have,
with the same starred notation,
\be \sqrt{\delta_f}=
\frac{\left|\left\{L_f(s,\psi_d)\in\F^+:d\text{ prime, }
      L_f(1/2,\psi_d)=0\right\}\right|}
{\left.\sum\limits_{\mathclap{\substack{d\text{ prime}}}}\right.^\ast
  2a_f(-1/2)\sqrt{\kappa_f}d^{(1-k)/4}h(\log d)}.\ee

\subsection{Translating the cutoff value}
The probabilities $P_f$ and $P_{O^+}$ are related, for small values of $x$
(see~\eqref{eq:POasymp} and~\eqref{eq:Pfdef}), by
\be P_f(d,x)\sim P_{O^+}\left(\nstd,a^{-2}_f(-1/2)x\right),\ee where $\nstd\sim\log d$.
Thus a cutoff of $\delta_f\kappa_fd^{(1-k)/2}$ applied to $P_f(d,x)$ scales to
\be\cstd\times\exp\left((1-k)\nstd/2\right)
:=a_f^{-2}(-1/2)\delta_f\kappa_f\times\exp\left((1-k)\nstd/2\right)\ee for the distribution
of values of characteristic polynomials of matrices of size $\nstd$.

Now we perform the same calculation for matrices of size $\neff$, referring
to~\eqref{eq:POasymp}, \eqref{eq:hNasymp}, and \eqref{eq:Pfdef}.
\be\begin{aligned}
  P_f(d,x)\sim a_f(-1/2)x^{-1/2}h(\log d)
  &\sim a_f(-1/2)x^{-1/2}h(2a_1\neff) \\
  &\sim a_f(-1/2)x^{-1/2}(2a_1)^{3/8}h(\neff) \\
  &\sim P_{O^+}\left(\neff,a_f^{-2}(-1/2)(2a_1)^{-3/4}x\right).\end{aligned}\ee
Thus a cutoff of $\delta_f\kappa_fd^{(1-k)/2}$ applied to $P_f(d,x)$ scales to
\be\ceff\times\exp\left(-\nstd/2\right):=
a_f^{-2}(-1/2)(2a_1)^{-3/4}\delta_f\kappa_f\times\exp(a_1(1-k)\neff)\ee for the
distribution of values of characteristic polynomials of matrices of size $\neff$.

\section{Basic automorphic $L$-functions}\label{sec:automorphic}
We collect some results on the subject of some various convolution $L$-functions of a
given holomorphic cuspidal newform $L$-function associated to a form of non-principal
character, making note of where these facts diverge from some existing literature.

In what follows, $f\in S_k^\star(M,\chi_f)$ is a newform of level $M$, weight $k$, and with
(possibly non-principal) nebentypus $\chi_f$. Such an $f$ has the Fourier (or $q$)
expansion \be f(s) = \sum_{n\geq1}\lambda_f(n)n^{(k-1)/2}e(nz); \ee
this normalization of the Fourier coefficients shifts the critical line of the
$L$-function $L_f(s)$ associated to $f$ to $\Re(s)=\frac12$ so that, among other
considerations, $L_f(s)$ is a member of the Selberg class of $L$-functions.
We write $\overline f(z)=\overline{f(\overline z)}
=\sum_{n\geq1}\overline{\lambda_f(n)}n^{(k-1)/2}e(nz)$ for the form dual to
$f$ and use the notation $\lambda_{\overline f}(n)$ in lieu of
$\overline{\lambda_f(n)}$. Deligne~\cite{deligne} proved the bound
\be\left|\lambda_f(n)\right|\leq d(n)n^{(k-1)/2}\ee
for any $n\geq1$ as a consequence of the Weil conjectures; $d(n)$ denotes the
divisor function. We also have the duality relation
\be\label{eq:lambdaduality} \lambda_f(n)=\chi_f(n)\lambda_{\overline f}(n)
\quad\text{if $(n,M)=1$} \ee on Fourier coefficients. If $\chi_f$ is principal, then
$f$ is self-dual by the duality relation and strong multiplicity one, and
$L_f(s)$ is given by the Dirichlet series
\be L_f(s)=\sum_{n=1}^\infty \frac{\lambda_f(n)}{n^s},\quad\Re(s)>1, \ee
and Euler product
\be\label{eq:leulerprod} L_f(s) = \prod_p\left(1-\frac{\lambda_f(p)}{p^s}+\frac{\chi_f(p)}{p^{2s}}\right)^{-1}
=\prod_p\left(1-\frac{\alpha_f(p)}{p^s}\right)^{-1}
\left(1-\frac{\beta_f(p)}{p^s}\right)^{-1},\quad\Re(s)>1, \ee
where the Satake parameters $\alpha_f, \beta_f$ satisfy
\be \label{eq:satakeidentity}
\alpha_f(p)+\beta_f(p)=\lambda_f(p)\quad\text{and}\quad
\alpha_f(p)\beta_f(p)=\chi_f(p).
\ee
By comparing the coefficient of the $p^{-ms}$ term
of~\eqref{eq:leulerprod} with the Dirichlet series, we obtain the relation
\be\label{eq:fourierrelation} \lambda_f\left(p^m\right)
=\sum_{\ell=0}^m \alpha_f(p)^\ell \beta_f(p)^{m-\ell}. \ee
Specializing to $m=2$, we obtain the following relation on the Fourier
coefficients of $L_f(s)$:
\be\label{eq:fouriersquare}
\lambda_f(p)^2 = \lambda_f\left(p^2\right) + \chi_f(p). \ee
By dint of the above normalization of the Fourier coefficients, we can relate
$L_f(s)$ to $L_{\overline f}(1-s)$ by the functional equation
\be L_f(s)=\epsilon_f
\left(\frac{\sqrt M}{2\pi}\right)^{1-2s}
\frac{\Gamma\left(\frac{k+1}2-s\right)}{\Gamma\left(s+\frac{k-1}2\right)}
L_{\overline f}(1-s) \ee
where $M$ is the level of $f$ and $\epsilon_f$, the root number associated to
$f$, has unit modulus.

\subsection{Hecke characters and complex multiplication}\label{sec:cmforms}
In this section, we give a very brief characterization of forms constructed from Hecke
grossencharacter. We refer the reader to~\cite[\S12]{iwan},~\cite[\S4.8]{miyake}, and
especially~\cite{ribet} for more details on these forms.

Let $k\geq2$, let $D<0$ be a fundamental discriminant, let $K=\Q(\sqrt D)$ be the
corresponding imaginary quadratic field, $\pazocal O_K$ its ring of integers,
$I$ the group of fractional ideals, $\a\subset\pazocal O_K$ a modulus,
\be I(\a):=\left\{\mathfrak a\in I:(\mathfrak a,\a)=1\right\}\ee
the group of fractional ideals prime to $\a$, and $\xi:I(\a)\ra\C^\times$ a group
homomorphism satisfying $\xi\left(\alpha\pazocal O_K\right)=\alpha^{k-1}$ when
$\alpha\equiv1\mod\a$. Then $\xi$ is a so-called Hecke grossencharacter.
Put $\omega_\xi(n):=\xi\left(n\pazocal O_K\right)/n^{k-1}$. Then $\omega_\xi$ is a
Dirichlet character, and
\be\label{eq:cmdef} g(z)\ =\ \sum_{\mathfrak a\subset\pazocal O_K}\xi(\mathfrak a)e(\Nm\a z)\ \in\ S_k(\Gamma_0(|D|\Nm\a),\psi_K\omega_\xi),\ee
where $\psi_K=\left(\frac D\cdot\right)$ is
the Kronecker symbol attached to the fundamental discriminant $D<0$ corresponding to $K$
and $\Nm:I\ra\Q^\times$ denotes the norm map; in the case of an integral ideal
$\mathfrak a$, $\Nm\mathfrak a$ is simply the number of residue classes modulo
$\mathfrak a$; i.e., $\left|\pazocal O_K/\mathfrak a\right|$.

\begin{defn} A newform $f$ is said to have complex multiplication (CM) if there is a
  nontrivial Dirichlet character $\eta$ such that $\eta(p)\lambda_f(p)=\lambda_f(p)$
  for all primes $p$ in a set of primes of density 1.\end{defn}
On the one hand, Ribet~\cite{ribet} showed that if $f$ has CM by an imaginary quadratic field
$K$, it arises from a Hecke grossencharacter associated to $K$.
On the other, let $g$ (constructed as above) have the un-normalized $q$-expansion
$g(z)=\sum_{n=1}^\infty c_ne(nz)$, and if $p\nmid|D|\Nm\a$, then let $a_p$ denote the
eigenvalue of the Hecke operator $T_p$ on $g$.
It is then the case \cite[Corollary 3.5]{ribet} that there exists a unique newform $h$
of weight $k$ and character $\omega_\xi\psi_K$ with $q$-expansion
$h(z)=\sum_{n=1}^\infty a_ne(nz)$ such that $a_p=c_p$ for $p\nmid|D|\Nm\a$. Furthermore,
$a_p=0$ if $\psi_K(p)=-1$, so that $h$ has complex multiplication.

It is even possible to construct integral-weight cusp forms $f$ with complex multiplication
by their own nebentypuses; Shimura constructed examples of this type from imaginary
quadratic fields with class number 1 and unit group $\{\pm1\}$; the associated modulus is
then all of $\pazocal O_K$, so the level $M=|D|$. Such an $f$ has odd integral weight,
trivial $\omega_\xi$ (the Dirichlet character associated to the grossencharacter), and it
follows that the field of eigenvalues for $f$ is totally real, hence $f$ has all real
Fourier coefficients, and is self-dual.
We call such $f$ self-dual CM forms with nontrivial nebentypus, or `self-CM' forms.
We record a property of self-CM forms that we will refer to later.
\begin{lemma}\label{lem:goinnowhere} Let $f$ be self-CM of level $M$.
  Then for each positive fundamental discriminant $d$ prime to $M$,
  $\ep_{f\otimes\psi_d}=\ep_f$.\end{lemma}
\begin{proof}
  Since $f$ is self-CM, $\Nm\a=1$ and $M=|D|$.
  For any $f\in S^\star_k(M,\chi)$, the behavior of the root number $\ep_f$ under twisting
  by a Dirichlet character $\psi$, primitive modulo $r$ with $(r,M)=1$, is given explicitly
  by \be\ep_{f\otimes\psi}=\ep_f\chi(r)\psi(M)\frac{\tau(\psi)^2}{r},\ee where $\tau(\psi)$
  denotes the Gauss sum. This fact, combined with \eqref{eq:cmdef}, the facts that
  $\psi_d=\left(\frac d\cdot\right)$ is real, $D<0<d$, and $\omega_\xi$ is trivial,
  and quadratic reciprocity, together imply
  \be\label{eq:quadrecip}
  \ep_{f\otimes\psi_d}=\chi_f(d)\psi_d(M)\frac{\tau(\psi_d)^2}{d}\ep_f
  =\psi_D(d)\psi_d(-|D|)\ep_f=\left(\frac Dd\right)\left(\frac d{|D|} \right)\ep_f
  =(-1)^{(D'-1)(d'-1)/4}\ep_f,\ee
  where $D'$, $d'$ denote the odd parts (maximal odd divisors with the same sign) of $D$
  and $d$, respectively. Since $D$ and $d$ are both fundamental discriminants and share no
  prime factors, one of $\left\{D,d\right\}$ must be odd and therefore congruent to
  $1$ mod $4$ since both $D$ and $d$ are fundamental discriminants.
  We conclude $(-1)^{(D'-1)(d'-1)/4}=+1$, and $\ep_{f\otimes\psi_d}=\ep_f$.
\end{proof}

\subsection{Rankin-Selberg convolution}\label{sec:rs}
The main convolution $L$-series we concern ourselves with is the classical one
introduced by Rankin and Selberg. For two newforms $f$ and $g$ of level $M_f$,
$M_g$, respectively, the Dirichlet series of the Rankin-Selberg convolution is
\be\label{eq:rsdirichlet}
L(f\otimes g,s)=L(\chi_f\chi_g,2s)\sum_{n\geq1}\lambda_f(n)\lambda_g(n)n^{-s}
\ee
if $(M_f,M_g)=1$ and $[M_f,M_g]$ is square-free.
Note that $L(f\times g,s)$ is frequently written for the convolution $L$-series
in the factorization~\eqref{eq:rsdirichlet} of $L(f\otimes g,s)$. $L(f\times g,s)$
is an $L$-function.
However, this characterization of $L(f\otimes g,s)$ is not of use to us since we
will be considering the convolution of a Hecke form with itself or with its dual.
Fortunately, $L(f\otimes g,s)$ is an $L$-function, and therefore admits an Euler
product. The local factors at unramified primes (those primes not dividing the level
of $f$ or $g$) are given by
\be\label{eq:rseuler}\begin{aligned} L_p(f\otimes g,s)&=
(1-\alpha_f(p)\alpha_g(p)p^{-s})^{-1}
(1-\alpha_f(p)\beta_g(p)p^{-s})^{-1}
(1-\beta_f(p)\alpha_g(p)p^{-s})^{-1}
(1-\beta_f(p)\beta_g(p)p^{-s})^{-1} \\
&=\begin{aligned}[t]
  &1-\frac{\lambda_f(p)\lambda_g(p)}{p^s}+\frac{\chi_f(p)\lambda_g(p)^2
    +\chi_g(p)\lambda_f(p)^2-2\chi_f(p)\chi_g(p)}{p^{2s}} \\
  &-\frac{\chi_f(p)\chi_g(p)\lambda_f(p)\lambda_g(p)}{p^{3s}}
  +\frac{(\chi_f(p)\chi_g(p))^2}{p^{4s}}.\end{aligned}
\end{aligned}\ee
The local factors at ramified places are harder to write down in all cases,
but in the case that $f$ and $g$ have trivial character (and hence real Fourier
coefficients) and the levels $M_f$, $M_g$ are both square-free, Ogg writes down
the local factors of the convolution $L$-function in~\cite[Theorem 6]{Ogg}, with
a different normalization of the Fourier coefficients.

We first consider a more general case, where $f$ and $g$ are allowed to have
nontrivial characters, but where $p\| M_f$ and $p\nmid M_g$. At primes dividing
the level, $\alpha_f(p)=\lambda_f(p)$ and $\beta_f(p)=0$;
see~\cite[\S2]{simplezeros}. In this case, the local factor at $p$ is still given
by~\eqref{eq:rseuler}.\footnote{To see this, combine the remarks
  on~\cite[p.133]{IK} with the appropriate comments on page 2 of the errata at
  \url{http://www.math.ethz.ch/~kowalski/corrections-ant.pdf}.}
Restricting further to the case that the character of $f$ is principal,
$\alpha_f(p)^2=\lambda_f(p)^2=p^{-1}$, and the local factor at $p$ simplifies to
\be \left(1-\lambda_f(p)\lambda_g(p)p^{-s}+p^{-1-2s}\right)^{-1}.\ee
This is what Ogg writes down in the special case that $f$ and $g$ have trivial
character and square-free levels $M_f$, $M_g$, and $p\mid M_1$, $p\nmid M_2$.

In the case that $p\mid(M_f,M_g)$, even exactly, the local factor at $p$ differs
from~\eqref{eq:rseuler}. In the special case of two cusp forms with trivial
character and square-free levels $M_f$ and $M_g$, Ogg writes down the local
factor explicitly. Put \be \kappa_{fg}(p):=\lambda_f(p)\lambda_g(p)p.\ee
Then, since the Fourier coefficients are real and $p\mid(M_f,M_g)$,
$\left|\lambda_f(p)\right|=p^{-1/2}=\left|\lambda_g(p)\right|$, and
$\kappa_{fg}(p)=\pm1$. In this case, the local factor of $L(f\otimes g,s)$ at
$p$ is given by \be\label{eq:ogglocal2} \left(1-\kappa_{fg}(p)p^{-s}\right)^{-1}
\left(1-\kappa_{fg}(p)p^{-1-s}\right)^{-1}.\ee

Mœglin and Waldspurger~\cite{MW} proved that $L(f\otimes g,s)$ is entire if
$g\ne\overline f$, and has a simple pole at $s=1$ if $g=\overline f$.
There is a formula for the residue of $L(f\otimes\overline f,s)$; it is
\be\label{eq:rsres}
\res_{s=1}L(f\otimes\overline f,s)=\frac{(4\pi)^k\langle f,f\rangle}
{\Gamma(k)\operatorname{vol}(\Gamma_0(M_f)\backslash\mathfrak H)}=\Lad{1},\ee
where $\langle \cdot,\cdot\rangle$ is the Petersson inner product.
The convolution $L$-function $L(f\times\overline f,s)$ defined above contains
the pole of $L(f\otimes\overline f,s)$ at $s=1$ and
\be\res_{s=1} L(f\times\overline f,s)
= \lim_{x\ra\infty}\frac1 x\sum_{n\leq x}\left|\lambda_f(n)\right|^2.\ee
This expression for the residue of $L_f(f\times\overline f,s)$ at $s=1$
essentially follows from the work of Rankin and Selberg
(c.f.~\cite[Remark~1.3, p.4]{simplezeros}). We will not need it here.

\subsection{Symmetric and adjoint square $L$-functions}\label{sec:symad}
We turn now to the definition and properties of the symmetric square and
adjoint square $L$-functions of newforms. Following~\cite{IK}, we define them as
factors of Rankin-Selberg convolutions. Let $\chi_f'$ denote the primitive
character that induces $\chi_f$, the nebentypus of $f$. Then we have
\begin{align}
  \label{eq:symfactorization}
  \Lsym{s} &:= L(f\otimes f,s)L(\chi'_f,s)^{-1} \\
  \label{eq:adfactorization}
  \Lad{s} &:= L(f\otimes\overline f,s)\zeta(s)^{-1}.
\end{align}
This is a reproduction of lines (5.97) and (5.98) in~\cite{IK} with one modification;
namely, the replacement of the character with the primitive character that
induces it.

Note $L_f(\sym^2,s)$ has an Euler product with local factors at primes not dividing
the level given by
\be\label{eq:symeuler}
(1-\alpha_f(p)^2p^{-s})^{-1}
(1-\alpha_f(p)\beta_f(p)p^{-s})^{-1}
(1-\beta_f(p)^2p^{-s})^{-1}.
\ee
Because of the factorization~\eqref{eq:symfactorization}, writing down local factors
at ramified places of $L_f(\sym^2,s)$ is exactly as hard as writing down the
corresponding local factor for $L(f\otimes f,s)$; see the discussion in
\S\ref{sec:rs}. Note that in~\cite{HKS}, equations (2.29) and (2.30)
for the Euler product of the symmetric square agree with the above remarks.
In the case of
trivial character, which is the elliptic curve case considered in~\cite{HKS},
after restricting to prime level as they do, recognizing that
$\lambda_f(p)^2=p^{-1}$ in the~\cite[(2.30)]{HKS} formula, applying Ogg's
formula~\eqref{eq:ogglocal2} (with $\kappa_{ff}=1$), and leveraging the
factorization~\eqref{eq:symfactorization}, we find agreement.

We now compile some analytic facts about these functions in general. First,
note that $L_f(\ad^2,s)$ is always entire since $L(f\otimes\overline f,s)$
always has a simple pole at $s=1$, which cancels that of $\zeta(s)$.
The value $\Lad{s}$ assumes at $s=1$ is of arithmetic significance, as we have the
special equality \be\label{eq:ad1}\Lad{1}=\res_{s=1}L(f\otimes\overline f,s).\ee
In the case that $\chi_f$ is principal, $\chi'_f\equiv1$ and the duality
relation~\eqref{eq:lambdaduality} implies $f=\overline f$.
This is enough conclude that in this case, $L_f(\ad^2,s) = L_f(\sym^2,s)$.
Suppose now that $\chi_f$ is not principal. Then $L(s,\chi_f)$ is entire,
but $L(f\otimes f,s)$ still may not be, since $f$ may still be self-dual.
If $\chi_f$ is nontrivial, $f$ is self-dual only if $f$ has CM, as described
in~\S\ref{sec:cmforms}. More generally, if $f$ is constructed from a Hecke grossencharacter
$\xi$, then we have the factorization $\Lsym{s}=L(s,\xi^2)L(s,\omega_\xi)$. If $f$ is CM
and self-dual with nontrivial nebentypus, then $L(s,\omega_\xi)=\zeta(s)$, so again we see
that $\Lsym{s}$ has a pole.

In view of this, the statement in~\cite[p.\,137]{IK} that `if $f$ has real
coefficients, then the symmetric and adjoint squares coincide,' is not quite
right, since it excludes the possibility of self-CM forms. In this case, the adjoint and
symmetric square $L$-functions do not coincide, since $L(s,\chi'_f)\ne\zeta(s)$. This is
the only case where $L_f(\sym^2,s)$ is \emph{not} entire, since in this case,
$f=\overline f$ so $L(f\otimes f,s)$ has a pole at $s=1$, but $L(\chi'_f,s)$ is entire.
Hence $L_f(\sym^2,s)$ inherits the pole from $L(f\otimes f,s)$.

\section{Ratios conjecture}
It is necessary to obtain a one-level density formula to calculate $\neff$.
We use the ratios conjectures. Let $f\in S_k^\star(M,\chi_f)$ be as
in~\S\ref{sec:automorphic}.

Let $L(f\otimes\psi_d,s)$ denote the $L$-function obtained by twisting $L_f(s)$
by a quadratic character. Here $d$ is a fundamental discriminant, and $\psi_d$
is the Kronecker symbol. Then the twisted $L$-function $L(f\otimes\psi_d,s)$ is given, in
the half-plane $\Re(s)>1$, by
\be\label{eq:deftwistedLfun}
L(f\otimes\psi_d,s) = \sum_{n\geq1}\frac{\lambda_f(n)\psi_d(n)}{n^s}
=
\prod_p\left(1-\frac{\lambda_f(p)\psi_d(p)}{p^s}+
  \frac{\chi_f(p)\psi_d(p)^2}{p^{2s}}\right)^{-1},
\ee
where $\chi_f$ is the nebentypus of $f$.

Provided $(d,M)=1$, $L(f\otimes\psi_d,s)$ obeys the functional equation
\be\label{eq:twistedfuneq}
L(f\otimes\psi_d,s)
=\omega_f(d)\epsilon_f
\left(\frac{\sqrt M|d|}{2\pi}\right)^{1-2s}
\frac{\Gamma\left(\frac{k+1}2-s\right)}{\Gamma\left(s+\frac{k-1}2\right)}
L(\overline f\otimes\psi_d,1-s),\ee
where $\omega_f(d) = \chi_f(d)\psi_d(M)\frac{\tau(\psi)^2}{d}$,\footnote{c.f.~\cite[p.377]{IK}.}
but since $\psi_d$ is quadratic and hence real, we have the identity
$\tau(\psi_d)^2=\psi_d(-1)d$,\footnote{c.f.~\cite[Theorem 2.1.47]{cohenNT}.}
whence \be\omega_f(d) = \chi_f(d)\psi_d(-M).\ee
We record the Dirichlet series which define the $L$-functions dual to $L_f(s)$
and $L(f\otimes\psi_d,s)$, respectively, in the half-plane $\Re(s)>1$.
\be L(\overline f,s)=\sum_{n\geq1}\frac{\overline{\lambda_f(n)}}{n^s}, \quad\text{and}\quad
L(\overline f\otimes\psi_d,s)=\sum_{n\geq1}\frac{\overline{\lambda_f(n)}\psi_d(n)}{n^{s}}.
\ee

To derive a formula for the one-level density of the zeros near the critical
point $s=\frac12$ of $L$-functions for our family, we
consider the average over the family of a ratio of shifted $L$-functions:
\be\label{eq:Rfdef} R_f(\alpha,\gamma):=\sum_{\substack{0<d\leq X\\d\text{ good}}}
\frac{L\left(f\otimes\psi_d,\frac12+\alpha\right)}{L\left(f\otimes\psi_d,\frac12+\gamma\right)},\ee
where the conditions on good $d$ will be described below in~\eqref{eq:dgooddef}.
Using~\eqref{eq:deftwistedLfun}, we have
\be\label{eq:recipseries}
\frac1{L(f\otimes\psi_d,s)}=\sum_{n\geq1}\frac{\mu_f(n)\psi_d(n)}{n^s},\ee
where $\mu_f(n)$ is a multiplicative function defined by
\be \mu_f(n) = \begin{cases}
  -\lambda_f(p) & \text{if $n=p$,} \\
  \chi_f(p) & \text{if $n=p^2$,} \\
  0 & \text{if $n=p^j$, $j>2$.}
\end{cases} \label{eq:mudef} \ee
We now use the approximate functional equation for the $L$-function in the
numerator of~\eqref{eq:Rfdef}:
\be\label{eq:approxfuneq}
L\left(f\otimes\psi_d,\frac12+\alpha\right)
=
\sum_{m<x}\frac{\psi_d(m)\lambda_f(m)}{m^{\frac12+\alpha}}
+
\omega_f(d)\ep_f\left(\frac{\sqrt M |d|}{2\pi}\right)^{-2\alpha}
\frac{\Gamma\left(\frac k2-\alpha\right)}{\Gamma\left(\frac k2+\alpha\right)}
\sum_{n<y}\frac{\overline{\lambda_f(n)}\psi_d(n)}
{n^{\frac12-\alpha}}+\text{remainder},
\ee
where $xy=d^2/2\pi$. We denote the first sum arising from the approximate
functional equation $R_f^1(\alpha,\gamma)$; it is thus given by
\be\label{eq:R1def}R_f^1(\alpha,\gamma):=\sum_{\substack{0<d\leq X\\d\text{ good}}}
\sum_{h,m}\frac{\lambda_f(m)\mu_f(h)\psi_d(mh)}{m^{\frac12+\alpha}h^{\frac12+\gamma}}.\ee
Likewise for $R_f^2(\alpha,\gamma)$:
\be\label{eq:R2def}R_f^2(\alpha,\gamma):=
\omega_f(d)\ep_f\frac{\Gamma\left(\frac k2-\alpha\right)}{\Gamma\left(\frac k2+\alpha\right)}
\sum_{\substack{0<d\leq X\\d\text{ good}}}\left(\frac{\sqrt M |d|}{2\pi}\right)^{-2\alpha}
\sum_{h,m}\frac{\overline{\lambda_f(m)}\mu_f(h)\psi_d(mh)}
{m^{\frac12-\alpha}h^{\frac12+\gamma}}.\ee
The recipe given in~\cite{recipe} instructs to discard the remainder in the
approximate functional equation, and instructs us to replace the root numbers and summands
with their expectations over the family. The expectation of the root numbers
$\langle\omega_f(d)\ep_f\rangle$ will be determined by our choice of family in the next
section, where we will also determine the expectation for the summand. After these
replacements, we are left with a kind of equivalence
\be R_f(\alpha,\gamma)\approx R^1_f(\alpha,\gamma)+R^2_f(\alpha,\gamma). \ee
For other examples, see
{\it\color{Red}[Miller will add]}.

\subsection{Expectations}\label{sec:expectations}
The recipe calls for the replacement of $\omega_f(d)\ep_f$ and $\psi_d(mh)$ with their
averages over the family (the set of fundamental discriminants $d$ being summed over).
Addressing the latter first, we begin by setting
\be X^*_b = \sum_{\substack{0<d\leq X\\d\equiv b\mod M}} 1.\ee
\cite[Theorem 3.1.1]{recipe} contains the following expected value estimate,
provided that $M$ is either odd or divisible by at least 8 and $(b,M)$ is not
divisible by an odd prime square:
\be\label{eq:charexpect}
\frac1{X^*_b}\sum_{\substack{0<d\leq X\\d\equiv b\mod M}}\psi_d(n)
\approx
\begin{cases}
  \psi_b(g)a(n) & \parbox[t]{0.5\textwidth}{if $n=g\square$, where
    $(\square,M)=1$ and $g$ has no prime factors not shared by $M$,} \\
  0 & \text{otherwise,}
\end{cases}\ee
\begin{equation*}
\text{where}\quad a(n)=\prod_{p\mid\square}\frac{p}{p+1}.
\end{equation*}
That is, the terms not of the form $n=g\square$ (with the above conditions on
$g$ and $\square$).

The recipe calls for a splitting of the sums by residue classes of $d\mod M$ and the
replacement of the oscillatory term with its expected value; in our case, the diagonal
character sum estimate. How we split depends on $f$, but the general principle that
freedom in the level and freedom in $d$ are inverse to each other is generally valid.
That is, imposing restrictive conditions on the prime factorization of $M$ allows us to
impose few on $d$, while treating general $M$ causes us to need to split discriminants by
various residue classes depending on the factorization of $M$ to distinguish the various
contributions to the overall oscillation. This principle justifies the following
restrictions on $M$ and $d$.

If $\chi_f$ is trivial or $f\ne\overline f$, we insist $M$ is an odd prime.
Then $g=M^\ell$,
where $\ell$ depends on $n$. In the former case of trivial character, we restrict to even
fundamental discriminants; then
$\psi_b(M^\ell)=\psi_d(M)^\ell=\left(\chi_f(d)\ep_f\right)^\ell=\ep_f^\ell$, since
$b\equiv d\mod M$, $d$ satisfies $\chi_f(d)\psi_d(-M)\ep_f=+1$, and $\chi_f$ is trivial.
In the latter case that $f\ne\overline f$, which implies $\chi_f$ nontrivial, there is
no longer any notion of `even' or `odd' functional equation, and we simply restrict $d$
to the set of quadratic residues modulo $M$.

We also treat the case of self-dual CM forms with nontrivial nebentypus
(self-CM for short; c.f.~\S\ref{sec:cmforms} for their construction and a brief summary of
some of their relevant properties). Per Shimura's construction, these forms come about when
the modulus of an imaginary quadratic field $K$ is taken to be the whole ring of integers
$\pazocal O_K$.
Therefore, the level $M$ of such a form is exactly $|D|$, where $D<0$ is the fundamental
discriminant of $K$. Lemma~\ref{lem:goinnowhere} implies that we cannot change the sign of
such $f$ with quadratic twists by positive discriminants prime to the level. Therefore, we
simply restrict to $|D|$ prime and $\ep_f=+1$, and then split $d$ by residue classes mod
$|D|$.

In view of the above discussion, we now make precise the notion of `good' $d$ by defining
the set $\D(X)$, which will allow us to determine the corresponding expectations.
\begin{defn}\label{def:D}
  Let $\mathcal D$ denote the set of fundamental discriminants: integers either square-free
  and $\equiv1\mod4$, or equal to 4 times a square-free number $\equiv2$ or $3\mod 4$.
  Let $f\in S_k^\star(M,\chi_f)$ be a newform, \emph{with $M$ an odd
    prime. If $f$ is self-CM, assume $\ep_f=+1$.} With these restrictions on $f$, let
  $\heartsuit\in\{\pm1\}$ and $1\leq\diamondsuit<M$ be integers, and put
  \be\label{eq:dgooddef}\D(X):=\left\{\begin{aligned}
      &\left\{d\in\mathcal D:0<d\leq X\text{ and }\psi_d(M)\ep_f=+1\right\}
      &&\text{$\chi_f$ principal,} \\
      &\left\{d\in\mathcal D:0<d\leq X\text{ and }\psi_d(M)=\heartsuit \right\}
      &&\text{$f$ self-CM,} \\
      &\left\{d\in\mathcal D:0<d\leq X\text{ and }d\equiv\diamondsuit\mod M \right\}
      &&\text{$f\ne\overline f$.}\end{aligned}\right.\ee
\end{defn}
We now make precise our family $\F$.
\begin{defn}\label{def:F}
  Let \be\F(X):=\left\{L({f\otimes\psi_d},s):d\in\D(X)\right\},\ee
  with $f\in S^\star_k(M,\chi_f)$ restricted as in Definition~\ref{def:D}.
\end{defn}
Then, if $\chi_f$ is principal, $\F$ is the family of even quadratic twists
of $L_f(s)$. If $f$ is self-CM, $\F$ is the subfamily of the family of even
quadratic twists, with an added condition on the residue class of $d\mod M$.
If $f\ne\overline f$, then there is no notion of the parity of the functional equation
of $L_f(s)$, and $\F$ is a subfamily of the family of quadratic twists of
$L_f(s)$, with an added condition on the residue class of $d\mod M$.

For $d\in\D(X)$, $\psi_d(M)$ assumes the value
\be\E(M)=\begin{cases}
  \ep_f&\chi_f\text{ trivial,} \\
  \heartsuit&\text{$f$ self-CM} \\
  \left(\frac\diamondsuit M\right)&f\ne\overline f.\end{cases}\ee
This allows us to replace each summand of $R_f(\alpha,\gamma)$ by its expectation over
the family. As for the root number $\omega_f(d)\ep_f$, we have the expectation
\be\label{eq:rootexpect}
\eta_f:=\langle\omega_f(d)\ep_f\rangle=\begin{cases}1&f=\overline f \\
  \ep_f&\text{$f$ self-CM}\\
  \chi_f(\diamondsuit)\left(\frac\diamondsuit M\right)\ep_f&f\ne\overline f\end{cases}\ee
over $d$ by construction.
Recall we have restricted to even self-CM forms, so $\ep_f$ actually equals 1 in that
case. We prove the below cardinality estimates in Lemma~\ref{lem:funcount}:
\be\label{eq:cardest}\left|\D(X)\right|=
\begin{cases}
  \frac3{\pi^2}\frac M{2(M+1)}X+O\left(X^{1/2}\right)&f=\overline f \\
  \frac3{\pi^2}\frac M{M^2-1}X+O\left(X^{1/2}\right)&f\ne\overline f.
\end{cases}\ee

\subsection{Factoring divergent $L$-factors}
Since $\lambda_f(n)$ is a multiplicative function, it is natural to express
$R^1_f(\alpha,\gamma)$ as the Euler product
% The below is no longer convenient because we have defined the expectations locally.
% \be R^1_f(\alpha,\gamma)\approx \left|\D(X)\right|\sum_{mh=\square M^\ell}
% \frac{\lambda_f(m)\mu_f(h)a(mh)\E^\ell}
% {m^{\frac12+\alpha}h^{\frac12+\gamma}},
% \quad\text{with $(\square,M)=1$,} \ee which, factors as
\be R^1_f(\alpha,\gamma)\approx
\left|\D(X)\right|V_\mid(\alpha,\gamma)V_\nmid(\alpha,\gamma),\ee where
\begin{align}\label{eq:Vmid}
  & V_{\mid}(\alpha,\gamma) := \prod_{p\mid M}\left(\sum_{h,m\geq 0}
    \frac{\lambda_f(p^m)\mu_f(p^h)\E^{m+h}(p)}
    {p^{m(1/2+\alpha)+h(1/2+\gamma)}}\right),\quad\text{and} \\
  & V_\nmid(\alpha,\gamma) := \prod_{p\nmid M}
  \Bigg(1 + \frac p{p+1}
    \sum_{\substack{m,h\geq 0\\m+h>0\\m+h\text{ even}}}
    \frac{\lambda_f(p^m)\mu_f(p^h)}{p^{m(1/2+\alpha)+h(1/2+\gamma)}}
  \Bigg).
  \label{eq:Vnmidraw}
\end{align}
The definition \eqref{eq:mudef} implies that we need only consider $h=0,1$
for~\eqref{eq:Vmid} and $h=0,1,2$ for~\eqref{eq:Vnmidraw}. We are left with
\begin{align}
&V_\mid(\alpha,\gamma)=\prod_{p\mid M}
\left(
  \sum_{m=0}^\infty\left(
    \frac{\lambda_f(p^{m})\E^{m}(p)}{p^{m(1/2+\alpha)}}
    -\frac{\lambda_f(p)\lambda(p^{m})\E^{m+1}(p)}
    {p^{m(1/2+\alpha)+1/2+\gamma}}
  \right)
\right),\quad\text{and} \\
&V_\nmid(\alpha,\gamma)=\prod_{p\nmid M}
\left(1+\frac p{p+1}\left(
    \sum_{m=1}^\infty
    \frac{\lambda_f(p^{2m})}{p^{m(1+2\alpha)}}
    -\frac{\lambda_f(p)}{p^{1+\alpha+\gamma}}
    \sum_{m=0}^\infty
    \frac{\lambda(p^{2m+1})}{p^{m(1+2\alpha)}}
    +\frac{\chi_f(p)}{p^{1+2\gamma}}
    \sum_{m=0}^\infty
    \frac{\lambda_f(p^{2m})}{p^{m(1+2\alpha)}}
  \right)
\right).
\label{eq:Vnmid}
\end{align}
Recall the definition of the symmetric square $L$-function $L_f(\sym^2,s)$
associated to a fixed newform, given in~\eqref{eq:symfactorization}, and the form of its
local Euler factors at unramified primes, given in~\eqref{eq:symeuler}.
We wish to rewrite the Euler product for $L_f(\sym^2,s)$ in terms of the Fourier
coefficients. Using~\eqref{eq:satakeidentity} and~\eqref{eq:fouriersquare}, we
have that the local factor of $\Lsym{s}$ at a prime not dividing the level
is given by
\be\label{eq:symlocalfactor}
  \left(1-\frac{\lambda_f(p)^2-\alpha_f(p)\beta_f(p)}{p^s}
  + \frac{\alpha_f(p)\beta_f(p)(\lambda_f(p)^2-\alpha_f(p)\beta_f(p))}{p^{2s}}
  - \frac{\alpha_f(p)^3\beta_f(p)^3}{p^{3s}}\right)^{-1}.
\ee
We now consider~\eqref{eq:Vnmid} and~\eqref{eq:symlocalfactor} together.
We want to allow $-\frac14<\Re(\alpha)<\frac14$ and $\log X\ll\Re(\gamma)<\frac14$,
where the bounds at $\frac{1}{4}$ allow us to control the convergence of
Euler products of the type~\eqref{eq:Vnmid}. Suppose
$\Re(\alpha)$ and $\Re(\gamma)$ are very small. We write
\begin{align}
  V_\nmid(\alpha,\gamma)&=\prod_{p\nmid M}
  \left(1+\frac p{p+1}\left(
      \sum_{m=1}^\infty
      \frac{\lambda_f(p^{2m})}{p^{m(1+2\alpha)}}
      -\frac{\lambda_f(p)}{p^{1+\alpha+\gamma}}
      \sum_{m=0}^\infty
      \frac{\lambda_f(p^{2m+1})}{p^{m(1+2\alpha)}}
      +\frac{\chi_f(p)}{p^{1+2\gamma}}
      \sum_{m=0}^\infty
      \frac{\lambda_f(p^{2m})}{p^{m(1+2\alpha)}}
    \right)
  \right) \\
  &=
  \prod_{p\nmid M}\left(1+\frac{\lambda(p^2)}{p^{1+2\alpha}}
    -\frac{\lambda_f(p^2)+\chi_f(p)}{p^{1+\alpha+\gamma}}
    +\frac{\chi_f(p)}{p^{1+2\gamma}}+\cdots\right),
  \label{eq:Vnmidexpanded}
\end{align}
where the $\cdots$ indicate terms that converge like $1/p^2$ when $\alpha$ or
$\gamma$ are small. We now use the following approximations to factor out the
divergent or slowly converging terms. By~\eqref{eq:symlocalfactor} we have
that the local factor of $\Lsym{1+2\alpha}$ at a prime not dividing the level is
\be
\left(1-\frac{\lambda_f(p^2)}{p^{1+2\alpha}}
  +\frac{\lambda_f(p^2)\chi_f(p)}{p^{2(1+2\alpha)}}
  -\frac{\chi_f(p)^3}{p^{3(1+2\alpha)}}
\right)^{-1}
=\left(1+\frac{\lambda(p^2)}{p^{1+2\alpha}}+\cdots\right),
\ee
where here we use the dots $\ldots$ in the same way as above.
We also observe that the local factor at a prime not dividing the level of
$\Lsym{1+\alpha+\gamma}^{-1}\Lchi{1+\alpha+\gamma}^{-1}$ coincides with that
of $L(f\otimes f,1+\alpha+\gamma)^{-1}$ by~\eqref{eq:symfactorization}; this factor
is \be\left(1-\frac{\lambda(p^2)+\chi_f(p)}{p^{1+\alpha+\gamma}}+\cdots\right).\ee
Last, we need to account for the factor $\chi_f(p)/p^{1+2\gamma}$
in~\eqref{eq:Vnmidexpanded}. If $\chi_f$ is nontrivial, this term is convergent
since $\Lchi{s}$ is entire. However, $\chi_f$ may be principal, in which case
we need to account for it. Note that both $\Lchi{s}$ and $\Lchi[']{s}$ have the same
local factors at primes not dividing the modulus of $\chi_f$, which is just $M$.
Therefore, we let a factor of $\Lchi[']{1+2\gamma}$ account for the divergence of
the term $\chi_f(p)/p^{1+2\gamma}$ in~\eqref{eq:Vnmidexpanded}.

Hence we can write
\be V_\nmid(\alpha,\gamma)V_\mid(\alpha,\gamma)
=Y_f(\alpha,\gamma)A_f(\alpha,\gamma), \ee
where
\be\label{eq:Yfdef}
\begin{aligned}
  Y_f(\alpha,\gamma)
  &=\frac{\Lchi[']{1+2\gamma}\Lsym{1+2\alpha}}{L(f\otimes f,1+\alpha+\gamma)} \\
  &=\frac{\Lchi[']{1+2\gamma}\Lsym{1+2\alpha}}
  {\Lchi[']{1+\alpha+\gamma}\Lsym{1+\alpha+\gamma}},
\end{aligned}\ee
  
and $A_f(\alpha,\gamma)$ is given by
\be\label{eq:Afdef} \begin{aligned}
  A_f(\alpha,\gamma) &= Y_f(\alpha,\gamma)^{-1} \\
  &\times
  \prod_{p\nmid M}
  \left(1+\frac p{p+1}\left(
      \sum_{m=1}^\infty
      \frac{\lambda_f(p^{2m})}{p^{m(1+2\alpha)}}
      -\frac{\lambda_f(p)}{p^{1+\alpha+\gamma}}
      \sum_{m=0}^\infty
      \frac{\lambda(p^{2m+1})}{p^{m(1+2\alpha)}}
      +\frac{\chi_f(p)}{p^{1+2\gamma}}
      \sum_{m=0}^\infty
      \frac{\lambda_f(p^{2m})}{p^{m(1+2\alpha)}}
    \right)
  \right) \\
  &\times
  \prod_{p\mid M}
  \left(
    \sum_{m=0}^\infty\left(
      \frac{\lambda_f(p^{m})\E^{m}(p)}{p^{m(1/2+\alpha)}}
      -\frac{\lambda_f(p)}{p^{1/2+\gamma}}
      \frac{\lambda_f(p^{m})\E^{m+1}(p)}
      {p^{m(1/2+\alpha)}}
    \right)
  \right),
\end{aligned} \ee
which is analytic as $\alpha,\gamma\ra0$. Hence, recalling~\eqref{eq:R1def}, we have
\be R^1_f(\alpha,\gamma)\approx\sum_{d\in\D(X)}Y_f(\alpha,\gamma)A_f(\alpha,\gamma).\ee
We obtain the other sum $R_f^2(\alpha,\gamma)$ in~\eqref{eq:R2def}) by using the
second term of the approximate functional equation~\eqref{eq:approxfuneq} and
carrying out the same steps as above. We have analogous expressions for
$V_\mid(\alpha,\gamma)$ and $V_\nmid(\alpha,\gamma)$, which we denote by
$\tilde{V}_\mid(\alpha,\gamma)$ and $\tilde{V}_\nmid(\alpha,\gamma)$:
\begin{align}
  &\tilde{V}_\mid(-\alpha,\gamma)=\prod_{p\mid M}
  \left(
    \sum_{m=0}^\infty\left(
      \frac{\overline{\lambda_f(p^{m})}\E^{m}(p)}{p^{m(1/2-\alpha)}}
      -\frac{\lambda_f(p)\overline{\lambda(p^{m})}\E^{m+1}(p)}
      {p^{m(1/2-\alpha)+1/2+\gamma}}
    \right)
  \right), \quad\text{and} \\
  &\begin{aligned}\tilde{V}_\nmid(-\alpha,\gamma)&=\prod_{p\nmid M}
    \left(1+\frac p{p+1}\left(
        \sum_{m=1}^\infty
        \frac{\overline{\lambda_f(p^{2m})}}{p^{m(1-2\alpha)}}
        -\frac{\lambda_f(p)}{p^{1-\alpha+\gamma}}
        \sum_{m=0}^\infty
        \frac{\overline{\lambda(p^{2m+1})}}{p^{m(1-2\alpha)}}
        +\frac{\chi_f(p)}{p^{1+2\gamma}}
        \sum_{m=0}^\infty
        \frac{\overline{\lambda_f(p^{2m})}}{p^{m(1-2\alpha)}}
      \right)
    \right), \\
    &=\prod_{p\nmid M}\left(1+\frac{\overline{\lambda(p^2)}}{p^{1-2\alpha}}
      -\frac{\left|\lambda_f(p)\right|^2}{p^{1-\alpha+\gamma}}
      +\frac{\chi_f(p)}{p^{1+2\gamma}}+\cdots\right).
  \end{aligned}
\label{eq:tildeVnmid}
\end{align}
Note that $L_{\overline f}(\sym^2,1-2\alpha)$, with local factor at primes
not dividing the level given by
\be\prod_p\left(1+\frac{\overline{\lambda_f(p^2)}}{p^{1-2\alpha}}+\cdots\right),\ee
handles the $\overline{\lambda_f(p^2)}/p^{1-2\alpha}$ term, and the
$\chi_f(p)/p^{1+2\gamma}$ is handled identically to the previous case.
To handle the $p^{1-\alpha+\gamma}$ term, note that the local factor of
$L(f\otimes\overline f,1+\alpha+\gamma)^{-1}$ is
\be\left(1-\frac{\left|\lambda_f(p)\right|^2}{p^s}+\cdots\right),\ee
so we may then write
\be \tilde{V}_\mid(-\alpha,\gamma)\tilde{V}_\nmid(-\alpha,\gamma)
= \tilde{Y}_f(-\alpha,\gamma)\tilde{A}_f(-\alpha,\gamma), \ee
where
\begin{align}\label{eq:tildeYfdef}
  \tilde{Y}_f(-\alpha,\gamma)
  &=\frac{\Lchi[']{1+2\gamma}L_{\overline f}(\sym^2,1-2\alpha)}
    {L(f\otimes\overline f,1-\alpha+\gamma)} \\
  &=\frac{\Lchi[']{1+2\gamma}L_{\overline f}(\sym^2,1-2\alpha)}
    {\zeta(1-\alpha+\gamma)\Lad{1-\alpha+\gamma}},
\end{align}
and $\tilde{A}_f(-\alpha,\gamma)$ is given by
\be\label{eq:tildeAfdef} \begin{aligned}
  \tilde{A}_f(-\alpha,\gamma) &= \tilde{Y}_f(-\alpha,\gamma)^{-1} \\
  &\times \prod_{p\nmid M}
  \left(1+\frac p{p+1}\left(
      \sum_{m=1}^\infty
      \frac{\overline{\lambda_f(p^{2m})}}{p^{m(1-2\alpha)}}
      -\frac{\lambda_f(p)}{p^{1-\alpha+\gamma}}
      \sum_{m=0}^\infty
      \frac{\overline{\lambda(p^{2m+1})}}{p^{m(1-2\alpha)}}
      +\frac{\chi_f(p)}{p^{1+2\gamma}}
      \sum_{m=0}^\infty
      \frac{\overline{\lambda_f(p^{2m})}}{p^{m(1-2\alpha)}}
    \right)
  \right) \\
  &\times \prod_{p\mid M}
  \left(
    \sum_{m=0}^\infty\left(
      \frac{\overline{\lambda_f(p^{m})}\E^{m}(p)}{p^{m(1/2-\alpha)}}
      -\frac{\lambda_f(p)\overline{\lambda(p^{m})}\E^{m+1}(p)}
      {p^{m(1/2-\alpha)+1/2+\gamma}}
    \right)
  \right)
\end{aligned}\ee
and is analytic as $\alpha,\gamma\ra0$. Recalling~\eqref{eq:R2def}
and~\eqref{eq:rootexpect}, we have
\be R^2_f(\alpha,\gamma)\approx
\sum_{d\in\D(X)}\eta_f\left(\frac{\sqrt M |d|}{2\pi}\right)^{-2\alpha}
\frac{\Gamma\left(\frac k2-\alpha\right)}{\Gamma\left(\frac k2+\alpha\right)}
\tilde{Y}_f\tilde{A}_f(-\alpha,\gamma). \ee
The ratios recipe then gives us
\begin{conjecture}[Ratios Conjecture]\label{conj:ratios}
  For some reasonable conditions such as $-\frac14<\Re(\alpha)<\frac14$,
  $\frac1{\log X}\ll\Re(\gamma)<\frac14$ and $\Im(\alpha),\Im(\gamma)\ll X^{1-\vep}$,
  we have
  \be\begin{aligned}
    R_f(\alpha,\gamma)&=
    \sum_{d\in\D(X)}
    \frac{L_f\left(\frac12+\alpha,\psi_d\right)}{L_f\left(\frac12+\gamma,\psi_d\right)} \\
    &=\sum_{d\in\D(X)}
    \left[Y_fA_f(\alpha,\gamma)+
      \eta_f\left(\frac{\sqrt M |d|}{2\pi}\right)^{-2\alpha}
      \frac{\Gamma\left(\frac k2-\alpha\right)}{\Gamma\left(\frac k2+\alpha\right)}
      \tilde{Y}_f\tilde{A}_f(-\alpha,\gamma)\right]
    +O\left(X^{1/2+\vep}\right),
  \end{aligned}\ee
  where $Y_f$, $A_f$, $\tilde Y_f$, and $\tilde A_f$ are defined
  at~\eqref{eq:Yfdef}, \eqref{eq:Afdef}, \eqref{eq:tildeYfdef},
  \eqref{eq:tildeAfdef}, respectively, $f$ and $\D(X)$ are as in Definition~\ref{def:D},
  $M$ is the (odd prime) level of the $L$-function $L_f(s)$, and $\ep_f$ is the root number
  of its functional equation.
\end{conjecture}
Note that the error term $O\left(X^{1/2+\vep}\right)$ is part of the statement of
the ratios conjecture.
\subsection{Logarithmic derivative}
To calculate the one-level density we need the average of the logarithmic
derivative of $L$-functions in the family, so we note that
\be \sum_{d\in\D(X)}
\frac{L'_f\left(\frac12+r,\psi_d\right)}{L_f\left(\frac12+r,\psi_d\right)}
=\pp\alpha\left.R_f(\alpha,\gamma)\right|_{\alpha=\gamma=r}. \ee
Using~\eqref{eq:fourierrelation}, we obtain the relation
\be\label{eq:fouriercombo} \lambda_f(p)\lambda_f(p^{2m+1})
=\lambda_f(p^{2m+2})+\chi_f(p)\lambda_f(p^{2m}). \ee
Considering now $p\mid M$, combining~\eqref{eq:fourierrelation} with the fact
that $\chi_f(p)=0$ yields
\be\label{eq:fouriermultiplicativity}
\lambda_f(p^m)=\sum_{\ell=0}^m \alpha_f(p)^\ell \beta_f(p)^{m-\ell}
= \alpha_f(p)^m + \beta_f(p)^m = \lambda_f(p)^m, \ee
which establishes the multiplicativity of $\lambda_f(p)$ for $p\mid M$.
In fact, for $p\mid M$, actually $\alpha_f(p)=\lambda_f(p)$ and $\beta_f(p)=0$, which
amounts to the same thing.

Using~\eqref{eq:fouriermultiplicativity} for $p\mid M$,
we observe that the sum in $V_\mid(r,r)$ telescopes, yielding
\be\label{eq:euler1eq1}
\prod_{p\mid M}
  \left(
    \sum_{m=0}^\infty\left(
      \frac{\lambda_f(p^{m})\E^{m}(p)}{p^{m(1/2+r)}}
      -\frac{\lambda_f(p)}{p^{1/2+r}}
      \frac{\lambda(p^{m})\E^{m+1}(p)}
      {p^{m(1/2+r)}}
    \right)
  \right)
  = \prod_{p\mid M}
  \left(1-\lim_{m\ra\infty}\frac{\lambda_f(p)^m\E^{m+1}(p)}
  {p^{m(1/2+r)}}\right) = 1.
\ee
Likewise, applying~\eqref{eq:fouriercombo} to $V_\nmid(r,r)$ yields
\be\label{eq:euler2eq1}\begin{aligned}
  &\prod_{p\nmid M}
  \left(1+\frac p{p+1}\left(
      \sum_{m=1}^\infty
      \frac{\lambda_f(p^{2m})}{p^{m(1+2r)}}
      -\frac{\lambda_f(p)}{p^{1+r+r}}
      \sum_{m=0}^\infty
      \frac{\lambda(p^{2m+1})}{p^{m(1+2r)}}
      +\frac{\chi_f(p)}{p^{1+2r}}
      \sum_{m=0}^\infty
      \frac{\lambda_f(p^{2m})}{p^{m(1+2r)}}
    \right)
  \right) \\
  &=\prod_{p\nmid M}\left(1+\frac p{p+1}
    \sum_{m=0}^\infty\left(\frac{\lambda_f(p^{2m+2})}{p^{(m+1)(1+2r)}}
      -\frac{\lambda_f(p)}{p^{(m+1)(1+2r)}}\lambda_f(p^{2m+1})
      +\frac{\chi_f(p)}{p^{(m+1)(1+2r)}}\lambda_f(p^{2m})
    \right)\right) \\
  &=\prod_{p\nmid M}\left(1+\frac p{p+1}
    \sum_{m=0}^\infty\left(\frac{\lambda_f(p^{2m+2})}{p^{(m+1)(1+2r)}}
      -\frac{\lambda_f(p^{2m+2})}{p^{(m+1)(1+2r)}}
    \right)\right) = 1.
\end{aligned}\ee
Last,
\be Y_f(r,r) = 1, \ee
so we can conclude $A_f(r,r)=1$. Put
\be\label{eq:A1fdef}
A_f^1(r,r)=\pp\alpha\left.A_f(\alpha,\gamma)\right|_{\alpha=\gamma=r}.
\ee
Then
\begin{align}
  \pp\alpha\left.Y_f(\alpha,\gamma)A_f(\alpha,\gamma)\right|_{\alpha,\gamma=r}
  &=-\frac{L'\left(\chi'_f,1+2r\right)}{\Lchi[']{1+2r}}A_f(r,r)
  +\frac{L'_f(\sym^2,1+2r)}{L_f(\sym^2,1+2r)}A_f(r,r)+A_f^1(r,r) \\
  &=-\frac{L'\left(\chi'_f,1+2r\right)}{\Lchi[']{1+2r}}
  +\frac{L'_f(\sym^2,1+2r)}{L_f(\sym^2,1+2r)}+A_f^1(r,r).
\end{align}
Next, by virtue of the pole of the zeta function in the denominator and the
nonvanishing of $\Lad{s}$ at $s=1$, we have the simple expression
\be\begin{aligned}
  &\pp\alpha\left.
    \eta_f\left(\frac{\sqrt M |d|}{2\pi}\right)^{-2\alpha}
    \frac{\Gamma\left(\frac k2-\alpha\right)}{\Gamma\left(\frac k2+\alpha\right)}
    \tilde{Y}_f(-\alpha,\gamma)\tilde{A}_f(-\alpha,\gamma)\right|_{\alpha=\gamma=r} \\
  &=-\eta_f\left(\frac{\sqrt M |d|}{2\pi}\right)^{-2r}
  \frac{\Gamma\left(\frac k2-r\right)}{\Gamma\left(\frac k2+r\right)}
  \frac{\Lchi[']{1+2r}L_{\overline f}(\sym^2,1-2r)}
  {L_f(\ad^2,1)}\tilde{A}_f(-r,r).
\end{aligned}\ee
Therefore we have the following for the logarithmic derivative.
\begin{theorem}\label{thm:logderiv}
  Assuming the Ratios Conjecture~\ref{conj:ratios} and $\tfrac1{\log X}\ll\Re(r)<\tfrac14$
  and $\Im(r)\ll X^{1-\vep}$, the average of the logarithmic derivative over the family
  $\F(X)$ given in Definition~\ref{def:F}, a family of quadratic twists of the
  $L$-function of a holomorphic cuspidal newform $f$ of weight $k$ and (odd prime) level
  $M$, is
  \be\label{eq:logderiv}\begin{aligned}
    &\sum_{d\in\D(X)}
    \frac{L'_f\left(\frac12+r,\psi_d\right)}{L_f\left(\frac12+r,\psi_d\right)}
    \\ =
    &\sum_{d\in\D(X)}
    \begin{multlined}[t]
      \Bigg[-\frac{L'(\chi'_f,1+2r)}{\Lchi[']{1+2r}}
      +\frac{L'_f(\sym^2,1+2r)}{L_f(\sym^2,1+2r)}+A_f^1(r,r) \\
      -\eta_f\left(\frac{\sqrt M |d|}{2\pi}\right)^{-2r}
      \frac{\Gamma\left(\frac k2-r\right)}{\Gamma\left(\frac k2+r\right)}
      \frac{\Lchi[']{1+2r}L_{\overline f}(\sym^2,1-2r)}
      {L_f(\ad^2,1)}\tilde{A}_f(-r,r)\Bigg]
      +O(X^{1/2+\vep}),
    \end{multlined}
  \end{aligned}
  \ee
  where $f$ and $\D(X)$ are as in Definition~\ref{def:D}, $\overline f$ is the dual form,
  $\ep_f$ is the root number of the functional equation of $L_f(s)$, $L_f(\sym^2,s)$ is the
  associated symmetric square $L$-function (defined at~\eqref{eq:symfactorization}),
  $L_f(\ad^2,s)$ is the associated adjoint square $L$-function
  (defined at~\eqref{eq:adfactorization}), and $A_f$, $\tilde A_f$, and $A^1_f$ are
  arithmetic factors defined at~\eqref{eq:Afdef}, \eqref{eq:tildeAfdef}, and
  \eqref{eq:A1fdef} respectively.
\end{theorem}
Note that although we do not know the dependence of the error term on $\alpha$,
we will not significantly increase its size by differentiating because it is
analytic, being the difference of $R_f(\alpha,\gamma)$ and the main term in
Conjecture~\ref{conj:ratios}, both of which are analytic.
\section{One-level density}\label{sec:1ld}
Let $\gamma_d$ denote the ordinate of a generic non-trivial zero of
$L_f(s,\psi_d)$ and let $\varphi\in\mathcal S(\R)$
(an even Schwartz test function). We consider the one-level density
\be D_1(f;\varphi)\ :=\ \frac1{\left|\D(X)\right|} S_1(f;\varphi),\quad\text{where }
S_1(f;\varphi)\ :=\ \sum_{d\in\D(X)}\sum_{\gamma_d}\varphi(\gamma_d). \ee
\subsection{The formula}
By the argument principle and Cauchy's theorem for contour integrals, and
assuming GRH, we have
\be S_1(f;\varphi)=\sum_{d\in\D(X)}\frac1{2\pi i}\left(\int_{(c)}-\int_{(1-c)}\right)
\frac{L'_f(s,\psi_d)}{L_f(s,\psi_d)}\varphi(-i(s-1/2)) \d s, \ee
where $1/2+1/\log X<c<3/4$ is fixed. Turning first to the integral on the $c$-line,
\be\label{eq:clineint} \frac1{2\pi}\int_{-\infty}^\infty \varphi(t-i(c-1/2))
\sum_{d\in\D(X)}\frac{L'_f\left(\tfrac12+\left(c-\tfrac12+it\right),\psi_d\right)}
  {L_f\left(\tfrac12+\left(c-\tfrac12+it\right),\psi_d\right)} \d t, \ee
the sum over $d$ can be replaced by Theorem~\ref{thm:logderiv}. The bounds on
the size $t$ coming from the ratios conjecture should not pose a problem.
See~\cite{HKS} and the one-level density section of~\cite{CS} for more details.
Next, we note that after the replacement, the integrand is regular at $t=0$.
This can be seen as follows.
By the prime number theorem for Dirichlet $L$-functions, $L(\chi_f,s)$ does not
vanish on the line $\Re(s)=1$.
Hence, the potential obstacles to regularity at $t=0$ are possible poles of
$\Lchi[']{s}$ or $\Lsym{s}$ at $s=1$. $\Lchi{s}$ has a pole at $s=1$ if and only
if $\chi_f$ is principal. $\Lsym{s}$ is entire unless $f$ is self-CM.
In both cases, $\eta_f=1$, so we omit it in the below arguments.

We first consider the case where $\chi_f$ is principal. Note that in this case
$\Lchi[']{s}=\zeta(s)$, $f$ is self-dual and $\Lsym{s}$ coincides with $\Lad{s}$ and
is entire.
By examining~\eqref{eq:logderiv}, we get the desired regularity at $t=0$ in this
case iff the two terms with poles at $r=0$ cancel exactly. As $r\ra0$,
\begin{align}
  &\zeta(1+2r)=\frac1{2r}+O(1)\quad\text{and} \\
  &-\frac{\zeta'(1+2r)}{\zeta(1+2r)} = \frac1{2r} + O(1).
\end{align}
Hence, the integrand is regular iff $\tilde A_f(0,0)=1$. But since $f$ is self-dual,
it is easily seen that $\tilde A_f$ coincides with $A_f$, and so this fact was
already established by combining~\eqref{eq:euler1eq1} and~\eqref{eq:euler2eq1}.
This establishes regularity of the integrand at $t=0$ in the case that $\chi_f$ is
principal.

Now we consider the case that $\Lsym{s}$ has a pole at $s=1$. This occurs only when
$\chi_f$ is not principal and $f$ is self-CM; c.f. \S\ref{sec:symad}.
In this case, $f$ is self-dual but $\chi_f$ is not principal, so $\Lchi[']{s}$ is
entire but $\Lsym{s}$ inherits the pole of $L(f\otimes f,s)$ at $s=1$. We note that
$\tilde A_f(-r,r)=A_f(-r,r)$ since $f=\overline f$ and hence all Fourier
coefficients are real. $A_f$ is analytic, and we have already shown that
$A_f(r,r)=1$. This allows us to conclude that in this case, $\tilde A_f(0,0)=1$ as
well. With this simplification in hand, we recall~\eqref{eq:ad1},
\eqref{eq:symfactorization}, and the argument principle and write, keeping in mind
$f=\overline f$,
\be\begin{aligned}
  &\lim_{r\ra0}\frac{L'_f(\sym^2,1+2r)}{L_f(\sym^2,1+2r)}
  -\left(\frac{\sqrt M |d|}{2\pi}\right)^{-2r}
      \frac{\Gamma\left(\frac k2-r\right)}{\Gamma\left(\frac k2+r\right)}
      \frac{\Lchi[']{1+2r}L_{\overline f}(\sym^2,1-2r)}
  {L_f(\ad^2,1)}\tilde{A}_f(-r,r) \\
  &=\lim_{r\ra0}\left(-\frac1{2r}+O(1)\right)
  -\frac{\Lchi[']{1+2r}\Lchi[']{1-2r}^{-1}L(f\otimes f,1-2r)}{\res_{s=1}L(f\otimes f,s)} \\
  &=\lim_{r\ra0}\left(-\frac1{2r}+O(1)\right)-\left(-\frac1{2r}+O(1)\right)=O(1).
\end{aligned}\ee
This establishes regularity of the integrand in~\eqref{eq:clineint} at $t=0$ in the
case that $f$ is self-CM and allows us to move the path of integration to
$c=\tfrac12$ to obtain
\begin{multline} \frac1{2\pi}\int_{-\infty}^\infty \varphi(t)\;\;\;\;\;\;\;
\sum_{\mathclap{\substack{d\in\D(X)\\\ }}}\;\;\;\;\;\;\;
\begin{aligned}[t]\Bigg[
&-\frac{L'(\chi'_f,1+2it)}{L(\chi'_f,1+2it)}
+\frac{L'_f(\sym^2,1+2it)}{L_f(\sym^2,1+2it)}+A_f^1(it,it) \\
&-\eta_f\left(\frac{\sqrt M |d|}{2\pi}\right)^{-2it}
\frac{\Gamma\left(\frac k2-it\right)}{\Gamma\left(\frac k2+it\right)}
\frac{\Lchi{1+2it}L_{\overline f}(\sym^2,1-2it)}
{L_f(\ad^2,1)}\tilde{A}_f(-it,it)
\Bigg] \d t\end{aligned} \\ +O(X^{1/2+\vep}).\end{multline}
For the integral on the line with real part $1-c$, we use the functional equation
\be\label{eq:sleektwistedfuneq}
L_{f}(s,\psi_d)=\chi_f(d)\psi_d(-M)\epsilon_f\gfactor{s}L_{\overline f}(1-s,\psi_d)\ee with
\be\label{eq:funeqgammafactor} \gfactor{s}=
\left(\frac{\sqrt M|d|}{2\pi}\right)^{1-2s}
\frac{\Gamma\left(\frac{k+1}2-s\right)}{\Gamma\left(s+\frac{k-1}2\right)}
\ee
to obtain (after a change $s\mapsto1-s$):
\be\label{eq:logderivfuneq} \frac{L'_f(1-s,\psi_d)}{L_f(1-s,\psi_d)} =
\frac{\gfactor[']{1-s}}{\gfactor{1-s}}
- \frac{L'_{\overline f}(s,\psi_d)}{L_{\overline f}(s,\psi_d)}. \ee
Now, the logarithmic derivative of~\eqref{eq:funeqgammafactor} evaluated at
$s=\tfrac12-\alpha$ is
\be\label{eq:logderivgammafactor}
\frac{\gfactor[']{\frac12-\alpha}}{\gfactor{\frac12-\alpha}}
=
-2\log\left(\frac{\sqrt M|d|}{2\pi}\right)
-\frac{\Gamma'}{\Gamma}\left(\frac k2+\alpha\right)
-\frac{\Gamma'}{\Gamma}\left(\frac k2-\alpha\right).
\ee
For the integral on the $(1-c)$ line we change variables $s\mapsto1-s$ and
use~\eqref{eq:logderivfuneq} (the sign of~\eqref{eq:logderivgammafactor} gets
switched back to negative when we flip the integral in $t$ coming from the
$(1-c)$ line back to $(-\infty,\infty)$). We have proved the following.
\begin{theorem}\label{thm:oneleveldensity}
  Assuming the Ratios Conjecture~\ref{conj:ratios}, the one-level density for
  the zeros of the family
  $\F(X)$ given in Definition~\ref{def:F}, a family of quadratic twists of the
  $L$-function of a holomorphic cuspidal newform $f$ of weight $k$ and (odd prime) level
  $M$, is given by
  \be\label{eq:densitythm}\begin{aligned}
    &S_1(f;\varphi) \\ &=\sum_{d\in\D(X)}
    \sum_{\gamma_d}\varphi(\gamma_d) \\
    &=\begin{aligned}[t]
      &\frac1{2\pi}\int_{-\infty}^\infty \varphi(t)\;
      \sum_{\mathclap{\substack{d\in\D(X)\\\ }}}\;\;\;\;\;\;
      \begin{aligned}[t]\Bigg[
        &2\log\left(\frac{\sqrt M|d|}{2\pi}\right)
        +\frac{\Gamma'}{\Gamma}\left(\frac k2+it\right)
        +\frac{\Gamma'}{\Gamma}\left(\frac k2-it\right) \\
        &-\frac{L'(\chi'_f,1+2it)}{L(\chi'_f,1+2it)}
        +\frac{L'_f(\sym^2,1+2it)}{L_f(\sym^2,1+2it)}+A_f^1(it,it) \\
        &-\eta_f\left(\frac{\sqrt M |d|}{2\pi}\right)^{-2it}
        \frac{\Gamma\left(\frac k2-it\right)}{\Gamma\left(\frac k2+it\right)}
        \frac{\Lchi[']{1+2it}L_{\overline f}(\sym^2,1-2it)}
        {L_f(\ad^2,1)}\tilde{A}_f(-it,it) \\
        &-\frac{L'(\chi'_{\overline f},1+2it)}{L(\chi'_{\overline f},1+2it)}
        +\frac{L'_{\overline f}(\sym^2,1+2it)}
        {L_{\overline f}(\sym^2,1+2it)}+A_{\overline f}^1(it,it) \\
        &-\eta_{\overline f}\left(\frac{\sqrt M |d|}{2\pi}\right)^{-2it}
        \frac{\Gamma\left(\frac k2-it\right)}{\Gamma\left(\frac k2+it\right)}
        \frac{\Lchibar[']{1+2it}L_f(\sym^2,1-2it)}
        {L_{\overline f}(\ad^2,1)}\tilde{A}_{\overline f}(-it,it)
        \Bigg] \d t\end{aligned} \\
      &+O(X^{1/2+\vep}),\end{aligned}
  \end{aligned}\ee
  where $f$ and $\D(X)$ are as in Definition~\ref{def:D}, $\overline f$ is the dual form,
  $\gamma_d$ is a generic zero of $L_f(s,\psi_d)$, $\varphi\in\mathcal S(\R)$ is an even
  test function, $\eta_f$ is the expectation of root numbers in the family $\F$ given
  in~\eqref{eq:rootexpect}, $\chi_f$
  is the nebentypus of $f$ and $\chi'_f$ the primitive character that induces it,
  $L_f(\sym^2,s)$ is the associated symmetric square $L$-function (defined
  at~\eqref{eq:symfactorization}), $L_f(\ad^2,s)$ is the associated adjoint square
  $L$-function (defined at~\eqref{eq:adfactorization}), and $A_f$, $\tilde A_f$, and
  $A^1_f$ are arithmetic factors defined at~\eqref{eq:Afdef},
  \eqref{eq:tildeAfdef}, and \eqref{eq:A1fdef} respectively.
\end{theorem}

\subsection{The expansion}
Finally, we calculate the one-level density for \emph{scaled} zeros and recover the
limit and the next-to-leading-order term from~\eqref{eq:densitythm}. We first
need to rescale the variable $t$ in~\eqref{eq:densitythm} as
\be\label{eq:Rdef} \tau = \tfrac t \pi R,
\quad\text{where $R:=\log \left(\tfrac{\sqrt M X}{2\pi e}\right)$,} \ee
and define
\be\label{eq:gdef} \varphi(t)= g\left(\tfrac t \pi R\right).\ee
After a change of variables, we obtain
\be\label{eq:normalizeddensity}\begin{aligned}
  &\sum_{d\in\D(X)}
  \sum_{\gamma_d}g\left(\frac{\gamma_d R}\pi\right) \\
  &=\frac1{2R}\int_{-\infty}^\infty g(\tau)
  \sum_{d\in\D(X)}
  \begin{aligned}[t]\Bigg[
    &2\log\left(\frac{\sqrt M|d|}{2\pi}\right)
    +\frac{\Gamma'}{\Gamma}\left(\frac k2+\frac{i\pi\,\tau}{R}\right)
    +\frac{\Gamma'}{\Gamma}\left(\frac k2-\frac{i\pi\,\tau}{R}\right) \\
    &-\frac{L'\left(\chi'_f,1+\frac{2i\pi\,\tau}{R}\right)}
    {\Lchi[']{1+\frac{2i\pi\,\tau}{R}}}
    +\frac{L'_f\left(\sym^2,1+\frac{2i\pi\,\tau}{R}\right)}
    {L_f\left(\sym^2,1+\frac{2i\pi\,\tau}{R}\right)}
    +A_f^1\left(\frac{i\pi\,\tau}{R},\frac{i\pi\,\tau}{R}\right) \\
    &-\eta_f\left(\frac{\sqrt M |d|}{2\pi}\right)^{-2i\pi\,\tau/R}
    \frac{\Gamma\left(\frac k2-\frac{i\pi\,\tau}{R}\right)}
    {\Gamma\left(\frac k2+\frac{i\pi\,\tau}{R}\right)}
    \frac{\Lchi[']{1+\frac{2i\pi\,\tau}{R}}
      L_{\overline f}\left(\sym^2,1-\frac{2i\pi\,\tau}{R}\right)}{L_f(\ad^2,1)} \\
    &\times\tilde{A}_f\left(-\frac{i\pi\,\tau}{R},\frac{i\pi\,\tau}{R}\right) \\
    &-\frac{L'\left(\chi'_{\overline f},1+\frac{2i\pi\,\tau}{R}\right)}
    {\Lchibar[']{1+\frac{2i\pi\,\tau}{R}}}
    +\frac{L'_{\overline f}\left(\sym^2,1+\frac{2i\pi\,\tau}{R}\right)}
    {L_{\overline f}\left(\sym^2,1+\frac{2i\pi\,\tau}{R}\right)}
    +A_{\overline f}^1\left(\frac{i\pi\,\tau}{R},\frac{i\pi\,\tau}{R}\right)\\
    &-\eta_{\overline f}\left(\frac{\sqrt M |d|}{2\pi}\right)^{-\frac{2i\pi\,\tau}{R}}
    \frac{\Gamma\left(\frac k2-\frac{i\pi\,\tau}{R}\right)}{\Gamma\left(\frac k2+\frac{i\pi\,\tau}{R}\right)}
    \frac{\Lchibar[']{1+\frac{2i\pi\,\tau}{R}}
      L_f\left(\sym^2,1-\frac{2i\pi\,\tau}{R}\right)}
    {L_{\overline f}(\ad^2,1)} \\
    &\times\tilde{A}_{\overline f}\left(-\frac{i\pi\,\tau}{R},\frac{i\pi\,\tau}{R}\right)
    \Bigg]\d t
    +O(X^{1/2+\vep}).\end{aligned}
\end{aligned}\ee
Summation by parts together with the cardinality estimate~\eqref{eq:cardest} yield the
following approximations (c.f.~\cite[Lemma A.2]{HMM} for details):
\begin{align}
  \sum_{d\in\D(X)}\log\left(\frac{\sqrt M |d|}{2\pi}\right)
  &=\left|\D(X)\right|
    \left[\log\left(\frac{\sqrt M X}{2\pi}\right)-1\right]+O\left(X^{1/2+\vep}\right),
    \text{ and} \\
  \sum_{d\in\D(X)}
  \left(\frac{\sqrt M |d|}{2\pi}\right)^{-2\pi i\,\tau/R}
  &=\left|\D(X)\right|\left(1-\frac{2i\pi\,\tau}{R}\right)^{-1}e^{-2i\pi\,\tau(1+1/R)}
    +O\left(X^{1/2+\vep}\right).
\end{align}
We now wish to obtain series expansions for the other terms in terms of $R$.
First, note that $\Lchi[']{s}$ is entire unless $\chi_f$ is principal, in
which case $\Lchi[']{s}=\zeta(s)$. We split into three cases: $\chi_f$ principal, $\chi_f$
non-principal but $f$ self-dual, and $f\ne\overline f$.
\subsection{Principal character}\label{subsec:pc}
For now, we assume $\chi_f$ principal, and observe that now $f$ is self-dual,
which yields simplifications. Among others, we have $\tilde A_f=A_f$ and $\eta_f=1$.
We rewrite~\eqref{eq:normalizeddensity}:
\be\label{eq:principalnormalizeddensity}\begin{aligned}
  &\sum_{d\in\D(X)}
  \sum_{\gamma_d}g\left(\frac{\gamma_d R}\pi\right) \\
  &=\frac1{2R}\int_{-\infty}^\infty g(\tau)\sum_{d\in\D(X)}
  \begin{aligned}[t]\Bigg(
    &2\log\left(\frac{\sqrt M|d|}{2\pi}\right)
    +\frac{\Gamma'}{\Gamma}\left(\frac k2+\frac{i\pi\,\tau}{R}\right)
    +\frac{\Gamma'}{\Gamma}\left(\frac k2-\frac{i\pi\,\tau}{R}\right) \\
    &+2\Bigg[-\frac{\zeta'\left(1+\frac{2i\pi\,\tau}{R}\right)}
    {\zeta\left(1+\frac{2i\pi\,\tau}{R}\right)}
    +\frac{L'_f\left(\sym^2,1+\frac{2i\pi\,\tau}{R}\right)}
    {L_f\left(\sym^2,1+\frac{2i\pi\,\tau}{R}\right)}
    +A_f^1\left(\frac{i\pi\,\tau}{R},\frac{i\pi\,\tau}{R}\right) \\
    &-\left(\frac{\sqrt M |d|}{2\pi}\right)^{-2i\pi\,\tau/R}
    \frac{\Gamma\left(\frac k2-\frac{i\pi\,\tau}{R}\right)}
    {\Gamma\left(\frac k2+\frac{i\pi\,\tau}{R}\right)}
    \frac{\zeta\left(1+\frac{2i\pi\,\tau}{R}\right)
      L_f\left(\sym^2,1-\frac{2i\pi\,\tau}{R}\right)}{L_f(\sym^2,1)} \\
    &\times{A}_f\left(-\frac{i\pi\,\tau}{R},\frac{i\pi\,\tau}{R}\right)
    \Bigg]
    \Bigg) \d t
    +O(X^{1/2+\vep}).\end{aligned}
\end{aligned}\ee
Writing
\be\zeta(1+s)=\frac1 s+\sum_{n=0}^\infty\frac{(-1)^n}{n!}\gamma_n s^n,\ee
we have
\be
\frac{\zeta'(1+s)}{\zeta(1+s)}=-s^{-1}+\gamma+(-\gamma^2-2\gamma_1)s+O(s^2),
\ee
where $\gamma=\gamma_0$ is the Euler-Mascheroni constant, and so
\be\zeta\left(1+\frac{2i\pi\,\tau}{R}\right)
=\frac{R}{2i\pi\,\tau}+\gamma-\gamma_1\frac{2i\pi\,\tau}{R}+O\left(R^{-2}\right)\ee
and
\be
\frac{\zeta'\left(1+\frac{2i\pi\,\tau}{R}\right)}
{\zeta\left(1+\frac{2i\pi\,\tau}{R}\right)}
=-\frac{R}{2i\pi\,\tau}+\gamma+(-\gamma^2-2\gamma_1)\frac{2i\pi\,\tau}{R}
+O\left(R^{-2}\right).\ee

% This Laurent expansion for primitive Dirichlet $L$-function about pole at $s=1$
% was performed in the original analysis before it was determined that in the
% principal case the Dirichlet becomes simply zeta. Therefore, this analysis is
% archaic and can be excised.
% 
% It follows that, for $f$ a newform of prime level $M$ and principal nebentypus
% $\chi_f$,
% \begin{align}
%   &L\left(\chi_f,1+\frac{2i\pi\,\tau}{R}\right)
%   =\left(1-M^{-1-\frac{2i\pi\,\tau}{R}}\right)
%   \zeta\left(1+\frac{2i\pi\,\tau}{R}\right) \\
%   &=(1-M^{-1})\frac{R}{2i\pi\,\tau}+
%   \left(\gamma-\frac{\gamma}M+\frac{\log M}M\right)
%   +M^{-1}(\gamma\log M-\tfrac12(\log M)^2+\gamma_1-2M\gamma_1)
%   \frac{2i\pi\,\tau}{R}
%   +O\left(R^{-2}\right)
% \end{align}
% and
% \be\frac{L'\left(\chi_f,1+\frac{2i\pi\,\tau}{R}\right)}
% {L\left(\chi_f,1+\frac{2i\pi\,\tau}{R}\right)}
% =-\frac{R}{2i\pi\,\tau}+\left(\gamma+\frac{\log M}{M-1}\right)
% +\left(-\gamma^2-2\gamma_1-\frac{M(\log M)^2}{(M-1)^2}\right)
% \frac{2i\pi\,\tau}{R}
% +O\left(R^{-2}\right).\ee

We record other expansions for completeness:
\begin{align}
  \frac{\Gamma'}{\Gamma}\left(\frac k2+\frac{i\pi\,\tau}{R}\right)
  &=\psi\left(\tfrac k2\right)+\psi^{(1)}\left(\tfrac k2\right)
  \frac{i\pi\,\tau}{R}+O\left(R^{-2}\right), \\
  \frac{\Gamma'}{\Gamma}\left(\frac k2-\frac{i\pi\,\tau}{R}\right)
  &=\psi\left(\tfrac k2\right)-\psi^{(1)}\left(\tfrac k2\right)
  \frac{i\pi\,\tau}{R}+O\left(R^{-2}\right), \\
  \frac{\Gamma\left(\frac k2-\frac{i\pi\,\tau}{R}\right)}
  {\Gamma\left(\frac k2+\frac{i\pi\,\tau}{R}\right)}
  &=1-2\psi\left(\frac k2\right)\frac{i\pi\,\tau}{R}+O\left(R^{-2}\right),
\end{align}
where $\psi(s)=\psi^{(0)}(s)=\frac{\Gamma'}{\Gamma}(s)$ is the digamma function. Further,
\begin{align}
  \frac{L'_f\left(\sym^2,1+\frac{2i\pi\,\tau}{R}\right)}
  {L_f\left(\sym^2,1+\frac{2i\pi\,\tau}{R}\right)}
  &=
  \frac{L'_f\left(\sym^2,1\right)}{L_f\left(\sym^2,1\right)}
  +
  \frac{L_f\left(\sym^2,1\right)L''_f\left(\sym^2,1\right)
    -L'_f\left(\sym^2,1\right)^2}{L_f\left(\sym^2,1\right)^2}
  \frac{2i\pi\,\tau}{R}
  +O\left(R^{-2}\right) \\
  L_f\left(\sym^2,1-\frac{2i\pi\,\tau}{R}\right)
  &=
  L_f\left(\sym^2,1\right)-L'_f\left(\sym^2,1\right)\frac{2i\pi\,\tau}{R}
  +O\left(R^{-2}\right) \\
  A_f^1\left(\frac{i\pi\,\tau}{R},\frac{i\pi\,\tau}{R}\right)
  &=
  A_f^1(0,0)+\dd{r} \left.A_f^1(r,r)\right|_{r=0}\frac{i\pi\,\tau}{R}
  -\dd[2]{r}\left.A_f^1(r,r)\right|_{r=0}\frac{\pi^2\,\tau^2}{2R^2}
  +O\left(R^{-2}\right), \\
  {A}_f\left(-\frac{i\pi\,\tau}{R},\frac{i\pi\,\tau}{R}\right)
  &=
  {B}_f(0)+ B'_f(0)\frac{i\pi\,\tau}{R}
  - B''_f(0)\frac{\pi^2\,\tau^2}{2R^2}+O\left(R^{-3}\right),
\end{align}
with
\begin{align}
  B_f(s) = A_f(-s,s), &&
  B^{(n)}_f(s)=\dd[n]{r}\left.A_f(-r,r)\right|_{r=s}.
\end{align}
Note, in the case that $\chi_f$ is principal,
\be\label{eq:Aftildezero}
A_f(0,0) = 1, \ee
since in that case $f$ is self-dual and the adjoint and symmetric square
$L$-functions coincide, and~\eqref{eq:euler1eq1} and \eqref{eq:euler2eq1}
apply since the Fourier coefficients are real.
This allows us to record the following expansion:
\be\label{eq:unsimplifiedprincipalexpansion}\begin{aligned}
  \frac1 {\left|\D(X)\right|} S_1(f;\varphi)
  &=
  \frac1 {\left|\D(X)\right|}\sum_{d\in\D(X)}
  \sum_{\gamma_d}g\left(\frac{\gamma_d R}\pi\right) \\
  &=
  \int_{-\infty}^\infty g(\tau)
  \left(
    1 + \frac{1-e^{-2i\pi\,\tau}}{2i\pi\,\tau}
    - a_1\frac{1+e^{-2i\pi\,\tau}}{R}
    - a_2\frac{i\pi\,\tau\,e^{-2i\pi\,\tau}}{R^2}
    + O\left(R^{-3}\right)
  \right)\d\tau,
\end{aligned}\ee
where
\be\label{eq:a1def}
a_1 = \gamma - \psi\left(\frac k 2\right) - A^1_f(0,0)
-\frac{L'_f(\sym^2,1)}{L_f(\sym^2,1)} \ee
and
\be\label{eq:a2def}
a_2 =
\begin{aligned}[t]
  & 1 -2\gamma\cdot\psi\left(\tfrac k2\right) +\psi\left(\tfrac k2\right)^2 -2\gamma_1
  +\gamma B'_f(0) -\psi\left(\tfrac k2\right) B'_f(0) +\frac{B''_f(0)}4 \\
  & -\left(2\gamma+B'_f(0)-2\psi\left(\tfrac k2\right)\right)
  \frac{\Lsym[']{1}}{\Lsym{1}}+\frac{\Lsym['']{1}}{\Lsym{1}}.
\end{aligned}\ee
To obtain~\eqref{eq:a1def}, we need the relation
\begin{equation}\label{eq:nightmareidentity}
  -\frac12 B'_f(0) = A^1_f(0,0). \end{equation}
The brute force computation required to establish this identity is straightforward and
unrewarding, and we omit it.

Recalling that $g$ is even, we can clear the odd terms in the integrand
of~\eqref{eq:unsimplifiedprincipalexpansion} and arrive at the following:
\be\begin{aligned}\label{eq:ratiosexpansionprincipal}
  D_1(f;\varphi)
  &=
  \frac1 {\left|\D(X)\right|}\sum_{d\in\D(X)}
  \sum_{\gamma_d}g\left(\frac{\gamma_d R}\pi\right) \\
  &=
  \int_{-\infty}^\infty g(\tau)
  \left(1+\frac{\sin(2\pi\,\tau)}{2\pi\,\tau}
    -a_1\frac{1+\cos(2\pi\,\tau)}{R}
    -a_2\frac{\pi\,\tau\,\sin(2\pi\,\tau)}{R^2}
    +O\left(R^{-3}\right)\right)\d\tau,
\end{aligned}\ee
with $a_1$, $a_2$ as in~\eqref{eq:a1def}, \eqref{eq:a2def},
respectively.
We note that specializing~\eqref{eq:ratiosexpansionprincipal}
to $k=2$ gives a slightly different expression than equation (3.18) of~\cite{HKS}.
This is because they scale by $\log \left(\tfrac{\sqrt M X}{2\pi}\right)$ while we scale
instead by $\log \left(\tfrac{\sqrt M X}{2\pi e}\right)$. Otherwise, the two match up.

\subsection{Non-principal character: generic case}\label{subsec:npc}
We return to~\eqref{eq:normalizeddensity}, but now we assume $\chi_f$ is
non-principal and $f\ne\overline f$. In this case, $\Lchi[']{s}$ and
$\Lsym{s}$ are entire (c.f. \S\ref{sec:symad} for details on the latter), and
the value of $\Lchi[']{s}$ and its derivatives at $s=1$ is not well known except in
particular cases, such as when $\chi'_f$ is a quadratic character associated to a
fundamental discriminant. In this case, $\Lchi[']{1}$ is given by Dirichlet's class
number formula. In this case, we have
\be\label{eq:unsimplifiednonprincipalexpansion}\begin{aligned}
  D_1(f;\varphi)
  &=
  \frac1 {\left|\D(X)\right|}\sum_{d\in\D(X)}
  \sum_{\gamma_d}g\left(\frac{\gamma_d R}\pi\right) \\
  &=
  \int_{-\infty}^\infty g(\tau)
  \left(
    1
    + \frac{b_1+b_2e^{-2i\pi\,\tau}}{R}
    + c_1\frac{i\pi\,\tau\,e^{-2i\pi\,\tau}}{R^2}
    + O\left(R^{-3}\right)
  \right)\d\tau,
\end{aligned}\ee
where
\begin{align}\label{eq:b1def}
  b_1 &= \psi\left(\tfrac k2\right)
  + \frac12 A^1_f(0,0) + \frac12 A^1_{\overline f}(0,0)
  -\frac12\frac{L'(1,\chi'_f)}{\Lchi[']{1}}
  -\frac12\frac{L'(1,\chi'_{\overline f})}{\Lchibar[']{1}}
  +\frac12\frac{\Lsym[']{1}}{\Lsym{1}}
  +\frac12\frac{L'_{\overline f}(1,\sym^2)}{L_{\overline f}(1,\sym^2)}, \\
  \label{eq:b2def}
  b_2 &= -\frac12\eta_f\tilde A_f(0,0)\Lchi[']{1}
  \frac{L_{\overline f}(1,\sym^2)}{\Lad{1}}
  -\frac12\eta_{\overline f}\tilde A_{\overline f}(0,0)\Lchibar[']{1}
  \frac{\Lsym{1}}{L_{\overline f}(1,\ad^2)},
\end{align}
and
\be\label{eq:c1def}
  c_1 = \begin{aligned}[t]
    & \eta_f\frac{L_{\overline f}(\sym^2,1)}{L_f(\ad^2,1)}
    \left(      
      - \frac12\tilde B'_f(0) \Lchi[']{1}
      + \psi\left(\tfrac k2\right) \cdot \tilde A_f(0,0) \Lchi[']{1}
      - \tilde A_f(0,0) L'(\chi'_f,1)
    \right) \\
    + & \eta_{\overline f}\frac{\Lsym{1}}{L_{\overline f}(\ad^2,1)}
    \left(
      - \frac12 \tilde B'_{\overline f}(0) \Lchibar[']{1}
      + \psi\left(\tfrac k2\right) \cdot \tilde A_{\overline f}(0,0) \Lchibar[']{1}
      - \tilde A_{\overline f}(0,0)L'(1,\chi'_{\overline f})
    \right) \\
    + & \eta_f\frac{L'_{\overline f}(1,\sym^2)}{L_f(\ad^2,1)} \tilde A_f(0,0) \Lchi[']{1}
    + \eta_{\overline f}\frac{\Lsym[']{1}}{L_{\overline f}(1,\ad^2)}
    \tilde A_{\overline f}(0,0) \Lchibar[']{1}.
  \end{aligned}
\ee
Note: $\tilde A_f(0,0)$ can no longer be simplified in an obvious way.
Clearing the odd terms in the integrand
of~\eqref{eq:unsimplifiednonprincipalexpansion}, we have
\be\begin{aligned}\label{eq:ratiosexpansionnonprincipal}
  D_1(f;\varphi)
  &=
  \frac1 {\left|\D(X)\right|}\sum_{d\in\D(X)}
  \sum_{\gamma_d}g\left(\frac{\gamma_d R}\pi\right) \\
  &=
  \int_{-\infty}^\infty g(\tau)
  \left(
    1
    + \frac{b_1+b_2\,\cos(2\pi\,\tau)}{R}
    + c_1\frac{\pi\,\tau\,\sin(2\pi\,\tau)}{R^2}
    + O\left(R^{-3}\right)
  \right)\d\tau,
\end{aligned}\ee
with $b_1$, $b_2$, and $c_1$ as in~\eqref{eq:b1def}, \eqref{eq:b2def},
and~\eqref{eq:c1def}, respectively.

\subsection{Non-principal character: the self-CM case}
We now consider the case when $L_f(\sym^2,s)$ is not entire. In this case,
$f$ is self-dual but has a non-principal nebentypus. This implies, among other
things, that most of the Fourier coefficients of $f$ are zero. It also implies
that $f$ arises from a Hecke grossencharacter on an imaginary quadratic field, and
has complex multiplication by its own nebentypus (c.f. \S\ref{sec:cmforms} and the
definition of the family $\F(X)$ given in Definition~\ref{def:F}). Since we assume
$\ep_f=+1$, we also have $\eta_f=+1$ by~\eqref{eq:rootexpect}.

In this special case, by the factorization~\eqref{eq:symfactorization},
$\Lsym{s}$ has a simple pole at $s=1$ with residue 
\be\symres:=\res_{s=1}L_f(\sym^2,s)=\frac{\res_{s=1}L(f\otimes f,s)}{\Lchi[']{1}}
=\frac{\Lad{1}}{\Lchi[']{1}},\ee
where $\res_{s=1}L(f\otimes f,s)$ in turn is given in~\eqref{eq:rsres}.

We write the Laurent expansion of $L_f(\sym^2,1+s)$ at $s=0$ as
\be L_f(\sym^2,1+s) = \frac{\symres}{s} + \xi_0 + \xi_1s + O(s^2), \ee
where \be \xi_n = \oint \frac{L_f(\sym^2,\zeta)}{\zeta^{n+1}} \d\zeta \ee
and the contour encloses the point $s=1$. Then the Laurent expansion at $s=0$
of the logarithmic derivative evaluated at $1+s$ is
\be \frac{L'_f(\sym^2,1+s)} {L_f(\sym^2,1+s)}
= -\frac1 s+\frac{\xi_0}{\symres} +
\left(\frac{2\xi_1}{\symres}-\frac{\xi_0^2}{\symres^2}\right)s
+ O(s^2). \ee

Since $f$ is self-dual, we can return to the simplified
equation~\eqref{eq:principalnormalizeddensity}, but now there is no pole in
$\Lchi[']{s}$ since $\chi_f$ is non-principal. We may thus
rewrite~\eqref{eq:normalizeddensity} as
\be\label{eq:principalnormalizeddensity}\begin{aligned}
  &\sum_{d\in\D(X)}
  \sum_{\gamma_d}g\left(\frac{\gamma_d R}\pi\right) \\
  &=\frac1{2R}\int_{-\infty}^\infty g(\tau)\sum_{d\in\D(X)}
  \begin{aligned}[t]\Bigg(
    &2\log\left(\frac{\sqrt M|d|}{2\pi}\right)
    +\frac{\Gamma'}{\Gamma}\left(\frac k2+\frac{i\pi\,\tau}{R}\right)
    +\frac{\Gamma'}{\Gamma}\left(\frac k2-\frac{i\pi\,\tau}{R}\right) \\
    &+2\Bigg[-\frac{L'\left(\chi'_f,1+\frac{2i\pi\,\tau}{R}\right)}
    {\Lchi[']{1+\frac{2i\pi\,\tau}{R}}}
    +\frac{L'_f\left(\sym^2,1+\frac{2i\pi\,\tau}{R}\right)}
    {L_f\left(\sym^2,1+\frac{2i\pi\,\tau}{R}\right)}
    +A_f^1\left(\frac{i\pi\,\tau}{R},\frac{i\pi\,\tau}{R}\right) \\
    &-\left(\frac{\sqrt M |d|}{2\pi}\right)^{-2i\pi\,\tau/R}
    \frac{\Gamma\left(\frac k2-\frac{i\pi\,\tau}{R}\right)}
    {\Gamma\left(\frac k2+\frac{i\pi\,\tau}{R}\right)}
    \frac{\Lchi[']{1+\frac{2i\pi\,\tau}{R}}
      L_f\left(\sym^2,1-\frac{2i\pi\,\tau}{R}\right)}{L_f(\ad^2,1)} \\
    &\times{A}_f\left(-\frac{i\pi\,\tau}{R},\frac{i\pi\,\tau}{R}\right)
    \Bigg]
    \Bigg) \d t
    +O(X^{1/2+\vep}),\end{aligned}
\end{aligned}\ee
The expansion is then
\be\label{eq:unsimplifiedcmexpansion}\begin{multlined}
  D_1(f;\varphi)
  =
  \frac1 {\left|\D(X)\right|}\sum_{d\in\D(X)}
  \sum_{\gamma_d}g\left(\frac{\gamma_d R}\pi\right) \\
  =
  \int_{-\infty}^\infty g(\tau)
  \left(
    1+\frac{e^{-2i\pi\,\tau}}{2i\pi\,\tau}+d_1\frac{1-e^{-2i\pi\,\tau}}R
    +d_2\frac{\pi i\,\tau e^{-2i\pi\,\tau}}{R^2}+O\left(R^{-3}\right)
  \right)\d\tau,
\end{multlined}\ee
where
\be\label{eq:d1def}
d_1=\psi\left(\tfrac k2\right)+\frac{\xi_0\cdot\Lchi[']{1}}{\Lad{1}}+A^1_f(0,0)
-\frac{L'(\chi'_f,1)}{\Lchi[']{1}}\ee
and
\be\label{eq:d2def} d_2=
\begin{aligned}[t]
  &1+\psi\left(\tfrac k2\right)^2 -\psi\left(\tfrac k2\right)B'_f(0) +\frac{B''_f(0)}4
  -\frac{\xi_0\cdot B'_f(0)\Lchi[']{1}}{\Lad{1}} +\frac{2\xi_1\cdot\Lchi[']{1}}{\Lad{1}} \\
  &+\frac{2\xi_0\cdot\psi\left(\tfrac k2\right)\Lchi[']{1}}{\Lad{1}}
  -\frac{2\xi_0\cdot L'(1,\chi'_f)}{\Lad{1}}+\frac{B'_f(0)L'(1,\chi'_f)}{\Lchi[']{1}} \\
  & -\frac{2\psi\left(\tfrac k2\right)L'(1,\chi'_f)}{\Lchi[']{1}}
  +\frac{L''(1,\chi'_f)}{\Lchi[']{1}},
  \end{aligned}\ee
where we have again used the identity \be -\frac12B'_f(0)=A^1_f(0,0).\ee

Clearing odd terms and recalling~\eqref{eq:rsres} and~\eqref{eq:symfactorization},
we are left with
\be\label{eq:simplifiedcmexpansion}\begin{aligned}
  D_1(f;\varphi)
  &=
  \frac1 {\left|\D(X)\right|}\sum_{d\in\D(X)}
  \sum_{\gamma_d}g\left(\frac{\gamma_d R}\pi\right) \\
  &=
  \int_{-\infty}^\infty g(\tau)
  \left(
    1-\frac{\sin(2\pi\,\tau)}{2\pi\,\tau}
    +d_1\frac{1-\cos(2\pi\,\tau)}R
    +d_2\frac{\pi\,\tau\sin(2\pi\,\tau)}{R^2}
    + O\left(R^{-3}\right)
  \right)\d\tau,
\end{aligned}\ee
with $d_1$ and $d_2$ as in~\eqref{eq:d1def} and~\eqref{eq:d2def}, respectively.

\subsection{Collection of one-level density results}
We collect the results of sections~\ref{subsec:pc} and~\ref{subsec:npc} in
the following
\begin{corollary}\label{cor:expansion}
  Assuming the Ratios Conjecture~\ref{conj:ratios}, the one-level density for
  the \emph{scaled} zeros of the family $\F(X)$ given in Definition~\ref{def:F},
  a family of quadratic twists of the $L$-function of a holomorphic cuspidal newform $f$ of
  weight $k$ and (odd prime) level $M$, is given by
  \begin{align}
    D_1(f;\varphi) &=
    \int_{-\infty}^\infty g(\tau)
    \left(1+\frac{\sin(2\pi\,\tau)}{2\pi\,\tau}
      +a_1\frac{1+\cos(2\pi\,\tau)}{R}
      -a_2\frac{\pi\,\tau\,\sin(2\pi\,\tau)}{R^2}
      +O\left(R^{-3}\right)\right)\d\tau
    \intertext{if $\chi_f$ is principal,}
    D_1(f;\varphi) &=
    \int_{-\infty}^\infty g(\tau)
    \left(
      1
      + \frac{b_1+b_2\,\cos(2\pi\,\tau)}{R}
      + c_1\frac{\pi\,\tau\,\sin(2\pi\,\tau)}{R^2}
      + O\left(R^{-3}\right)
    \right)\d\tau
    \intertext{if $\chi_f$ is not principal and $f\ne\overline f$, and}
    D_1(f;\varphi) &=
    \int_{-\infty}^\infty g(\tau)
    \left(
      1-\frac{\sin(2\pi\,\tau)}{2\pi\,\tau}
      +d_1\frac{1-\cos(2\pi\,\tau)}R
      +d_2\frac{\pi\,\tau\sin(2\pi\,\tau)}{R^2}
      + O\left(R^{-3}\right)
    \right)\d\tau,
  \end{align}
  if $\chi_f$ is not principal and $f=\overline f$,
  \newline
  
  \noindent where $f$ and $\D(X)$ are as in Definition~\ref{def:D},
  $R$ is defined at~\eqref{eq:Rdef}, $g$ is related to $\varphi\in\mathcal S(\R)$, an even
  test function, by~\eqref{eq:gdef},
  and the coefficients $a_1$, $a_2$, $b_1$, $b_2$, $c_1$, $d_1$, and $d_2$ are given
  in~\eqref{eq:a1def}, \eqref{eq:a2def}, \eqref{eq:b1def}, \eqref{eq:b2def},
  \eqref{eq:c1def}, \eqref{eq:d1def}, and~\eqref{eq:d2def}, respectively.
\end{corollary}
We take a moment to note that in the case where $\chi_f$ is principal, the
one-level density converges to that of the one-level scaling density of
eigenvalues near 1 in the group of random $SO(\text{even})$
matrices; in the case where $\chi_f$ is not principal and
$f\ne \overline f$, the one-level
density converges to that of the one-level scaling density of eigenvalues
near 1 in the group of random unitary matrices; and in the case where $\chi_f$ is
not principal and $f=\overline f$, the one-level density converges to that of
the one-level scaling density of eigenvalues near 1 in the group of random
symplectic matrices. Thus, we say that the
symmetry group of the family of quadratic twists of an $L$-function associated
to a fixed holomorphic cuspidal newform $f$ with principal nebentypus is even
orthogonal, the symmetry group of the family when $f$ has non-principal
nebentypus and is not CM is unitary, and the symmetry group of the
family when $f$ is CM is symplectic.

\section{Pair-correlation}
The one-level density results for the generic family of cusp forms $\F$ where
$f\ne\overline f$ suggest that this family has unitary symmetry. In this case, the detailed
expansion of the one-level density for the family is useless since there are no lower-order
terms on the random matrix side to match against. Therefore, we compute the
pair-correlation for $L$-functions in this generic family only, so as to obtain lower-order
terms of arithmetic origin with which to calibrate the matrix size of our random matrix
ensemble.

We remark that the pair-correlation is sensitive to any finite set of zeros, and is a
statistic that is computed for one $L$-function. That is, since we have an asymptotic
parameter for each function (height along the critical line), we have no need to average
over an infinite family of $L$-functions. Of course, the point of this calculation is to
calibrate the effective matrix size for our family $\F$. It is expected that the local zero
statistics of $L$-functions in a family such as $\F$ should agree to leading order with
corresponding random matrix eigenvalue statistics in the correct asymptotic limit.
Therefore, if we can find an effective matrix size that matches arithmetic lower-order
terms of the pair-correlation for one $L$-function, it should do a better job of
approximating \emph{all} local statistics for that function. Therefore, we compute lower
order terms of the pair-correlation for one $L$-function $L(f\otimes\psi_d,s)\in\F(X)$
(reserving the possibility that the twist is trivial), and expect that the effective matrix
size for low-lying zeros in the family $\F(X)$ will be given by the expectation of the
lower order terms over $d\in\D(X)$ as $X\ra\infty$.

Let $\gamma$, $\gamma'$ denote the ordinates of generic non-trivial zeros of
$L_f(s,\psi_d)$, and suppose
\be\label{def:testconditions}\parbox[c][][c]{24em}
{$\varphi(s)$ is holomorphic throughout the strip $|\Im s|<2$,
  real on the real line, even, and satisfies the bound
  $\varphi(x)\ll1/(1+x^2)$ as $x\ra\infty$.}\ee
We would like to evaluate
\be\label{eq:paircorr}
P(\varphi)=\sum_{0<\gamma,\gamma'<T}\varphi\left(\gamma-\gamma'\right).\ee
\subsection{Ratios lemma}
To properly evaluate~\eqref{eq:paircorr}, we will need a formula for the average of
logarithmic derivatives of shifted $L$-functions:
\be\int_0^T\frac{L'}{L}(s+\alpha,f\otimes\psi_d)
\frac{L'}{L}(1-s+\beta,\overline f\otimes\psi_d)\d t.\ee
To this end, we again apply the ratios recipe to the average of ratios of shifted
$L$-functions \be\T:=\int_0^T
\frac{L(s+\alpha,f\otimes\psi_d)L(1-s+\beta,\overline f\otimes\psi_d)}
{L(s+\gamma,f\otimes\psi_d)L(1-s+\delta,\overline f\otimes\psi_d)}\d t.\ee
In contrast to the definition of $R_f(\alpha,\gamma)$ in~\eqref{eq:Rfdef}, $d$ is now fixed
and we are averaging over $t$. We make use of the same approximate functional equation as
in~\eqref{eq:approxfuneq}, the same reciprocal Dirichlet series~\eqref{eq:recipseries} and
$\mu_g(n)$~\eqref{eq:mudef} (corresponding to the appropriate cusp form $g=f\otimes\psi_d$
or $\overline f\otimes\psi_d$), and we reprise $\gfactor{s}$ as
in~\eqref{eq:funeqgammafactor} so that the functional equation~\eqref{eq:twistedfuneq} for
$\L{s}$ may be written as in~\eqref{eq:sleektwistedfuneq}.

The recipe is mostly the same. We start by replacing the $L$-functions in the numerator by
their approximate functional equations (where now $xy=t/2\pi$) and discard the remainder.
We replace the $L$-functions in the denominator by their reciprocal series.
We then multiply through the denominator and retain only those terms with the same number
of $\gfactor{s}$ as $\gfactor{1-s}$, dropping the other factors, which are oscillatory.
By the duplication formula for the gamma function, we may write the local archimedian
factor of $L(s,f\otimes\psi_d)$ (equivalently its dual $L(s,\overline f\otimes\psi_d)$) as
\be\left(\frac{2^k}{8\pi}\right)^{1/2}\left(\frac{\sqrt M |d|}{\pi}\right)^s
\Gamma\left(\frac s2+\frac{k-1}4\right)\Gamma\left(\frac s2+\frac{k+1}4\right).\ee
It is in this form that we may specialize~\cite[Lemma 6.5]{ours} to our degree 2
$L$-function to obtain the estimate
\be\gfactor{s+\alpha}\gfactor{1-s+\beta}=Q^{-2(\alpha+\beta)}
\left(\frac t2\right)^{-2(\alpha+\beta)}\left(1+O\left(\frac1t\right)\right),
\qquad\text{with }\label{eq:defQ} Q:=\frac{\sqrt M|d|}{\pi}.\ee
We replace our product of local archimedian factors by this estimate.
We integrate term-by-term, and retain only the diagonal. (One implication of the ratios
conjectures is exceptional cancellation on the off-diagonal.) We then complete the sums
that result. Since $d$ is fixed, we let
$\lambda(n):=\lambda_{f\otimes\psi_d}(n)=\lambda_f(n)\psi_d(n)$ be the Fourier coefficients
and $\chi:=\chi_{f\otimes\psi_d}=\chi_f\psi_d^2$ be the nebentypus of the twisted form,
and then we let $\mu_d(n)$ denote the coefficient of $n^{-s}$ of the reciprocal series
$L(s,f\otimes\psi_d)^{-1}$ for $\Re s>1$; then
\be\mu_d(n)=\begin{cases}\lambda(n)&\text{if $n=p$,} \\ \chi(n)&\text{if $n=p^2$,} \\
  0&\text{if $n=p^j$, $j>2$.}\end{cases}\ee With this definition, $\overline \mu_d(n)$ is
equal to the Dirichlet coefficient of $L(s,\overline f\otimes\psi_d)^{-1}$ when $\Re s>1$.
Additionally, $\overline\lambda=\lambda_{\overline f\otimes\psi_d}$, and
$\overline\chi=\chi_{\overline f\otimes\psi_d}$. We are left with a kind of equivalence
\be\begin{aligned}[t]\T&\approx
  \begin{multlined}[t]\int_0^T\sum_{nk=mh}\frac{\lambda(m)\overline\lambda(n)
      \mu_d(h)\overline\mu_d(k)}{m^{1/2+\alpha}n^{1/2+\beta}h^{1/2+\gamma}k^{1/2+\delta}}
  + \\ +
  Q^{-2(\alpha+\beta)}\left(\frac t2\right)^{-2(\alpha+\beta)}
  \sum_{mk=nh}\frac{\overline\lambda(m)\lambda(n)\mu_d(h)\overline\mu_d(k)}
  {m^{1/2-\alpha}n^{1/2-\beta}h^{1/2+\gamma}k^{1/2+\delta}}\d t.\end{multlined} \\
  &=\int_0^T J_1 + J_2 \d t,\quad\text{say.}\end{aligned}\ee
Examining now the first diagonal sum, we rewrite it as the Euler product
\be J_1=\prod_p
\left(\sum_{n+k=h+m}\frac{\lambda(p^m)\overline\lambda(p^n)\mu_d(p^h)\overline\mu_d(p^k)}
  {p^{(1/2+\alpha)m+(1/2+\beta)n+(1/2+\gamma)h+(1/2+\delta)k}}\right).\ee Recalling the
definition of $\mu_d$, we note that if $p\mid N=M|d|^2$, we discard all terms except for
$h,k\in\{0,1\}$. If $p\nmid N$, we discard all except $h,k\in\{0,1,2\}$. We therefore have
\be J_1\approx W_\mid(\alpha,\beta,\gamma,\delta)W_\nmid(\alpha,\beta,\gamma,\delta),\ee
with
\begin{align}
  &W_\mid(\alpha,\beta,\gamma,\delta)=\prod_{p\mid M}
    \begin{multlined}[t]\Bigg[
      \sum_{m\geq0}\frac{\left|\lambda(p^m)\right|^2}{p^{(1+\alpha+\beta)m}}
      -\sum_{m\geq0}\frac{\lambda(p^{m+1})\overline\lambda(p^m)\overline\lambda(p)}
      {p^{(1+\alpha+\beta)m+(1+\alpha+\delta)}} \\
      -\sum_{m\geq0}\frac{\lambda(p^m)\overline\lambda(p^{m+1})\lambda(p)}
      {p^{(1+\alpha+\beta)m+(1+\beta+\gamma)}}
      +\sum_{m\geq0}\frac{\left|\lambda(p^m)\right|^2\left|\lambda(p)\right|^2}
      {p^{(1+\alpha+\beta)m+(1+\gamma+\delta)}}
      \Bigg],\text{ and}
    \end{multlined} \\
  \intertext{}
  &W_\nmid(\alpha,\beta,\gamma,\delta)=\prod_{p\nmid N}
    \begin{aligned}[t]\Bigg[
      &\sum_{m\geq0}\frac{\left|\lambda(p^m)\right|^2}{p^{(1+\alpha+\beta)m}}
      -\sum_{m\geq0}\frac{\lambda(p^{m+1})\overline\lambda(p^m)\overline\lambda(p)}
      {p^{(1+\alpha+\beta)m+(1+\alpha+\delta)}} 
      -\sum_{m\geq0}\frac{\lambda(p^m)\overline\lambda(p^{m+1})\lambda(p)}
      {p^{(1+\alpha+\beta)m+(1+\beta+\gamma)}} \\
      &+\sum_{m\geq0}\frac{\overline\lambda(p^m)\lambda(p^{m+2})\overline\chi(p)}
      {p^{(1+\alpha+\beta)m+(2+2\alpha+2\delta)}}
      +\sum_{m\geq0}\frac{\lambda(p^m)\overline\lambda(p^{m+2})\chi(p)}
      {p^{(1+\alpha+\beta)m+(2+2\beta+2\gamma)}} \\
      &+\sum_{m\geq0}\frac{\left|\lambda(p^m)\right|^2\left|\lambda(p)\right|^2}
      {p^{(1+\alpha+\beta)m+(1+\gamma+\delta)}}
      -\sum_{m\geq0}\frac{\lambda(p^{m+1})\overline\lambda(p^m)\lambda(p)\overline\chi(p)}
      {p^{(1+\alpha+\beta)m+(2+\alpha+\gamma+2\delta)}} \\
      &-\sum_{m\geq0}\frac{\lambda(p^m)\overline\lambda(p^{m+1})\overline\lambda(p)\chi(p)}
      {p^{(1+\alpha+\beta)m+(2+\beta+2\gamma+\delta)}}
      +\sum_{m\geq0}\frac{\left|\lambda(p^m)\right|^2\left|\chi(p)\right|^2}
      {p^{(1+\alpha+\beta)m+(2+2\gamma+2\delta)}}
      \Bigg].
    \end{aligned}
\end{align}
(Recall $\chi(p)$ may be zero on primes not dividing $N$.)
The places at $p\mid d$ are easily seen to be 1.
The divisor bound on the modulus of $\lambda(n)$,
\be|\lambda(n)|\leq\tau(n),\ee assures us
of the convergence of the sums in the Euler factor for a particular prime $p$.
We are again interested in examining the
slowly-converging parts of the $W_\nmid(\alpha,\beta,\gamma,\delta)$ Euler product.
Assuming $\alpha,\beta,\gamma,\delta\ll1$, we look at terms on the order of $1/p$ and see
that these terms correspond to local factors of the Rankin-Selberg convolution
of $f\otimes\psi_d$ with its dual. We write down the factorization
\be W_\nmid(\alpha,\beta,\gamma,\delta)W_\mid(\alpha,\beta,\gamma,\delta)
=\pazocal Y(\alpha,\beta,\gamma,\delta)\pazocal A(\alpha,\beta,\gamma,\delta),\ee where
\be\pazocal Y(\alpha,\beta,\gamma,\delta):=
\frac{L\left(1+\alpha+\beta,(f\otimes\psi_d)\otimes(\overline f\otimes\psi_d)\right)
  L\left(1+\gamma+\delta,(f\otimes\psi_d)\otimes(\overline f\otimes\psi_d)\right)}
{L\left(1+\alpha+\delta,(f\otimes\psi_d)\otimes(\overline f\otimes\psi_d)\right)
  L\left(1+\beta+\gamma,(f\otimes\psi_d)\otimes(\overline f\otimes\psi_d)\right)}\ee and
\be\label{eq:defpazoA} \pazocal A(\alpha,\beta,\gamma,\delta):=
\pazocal Y(\alpha,\beta,\gamma,\delta)^{-1}
W_\mid(\alpha,\beta,\gamma,\delta)W_\nmid(\alpha,\beta,\gamma,\delta)\ee
is analytic after setting $\gamma\mapsto\alpha$, $\delta\mapsto\beta$ and letting
$\alpha,\beta\ra0$.
We now proceed to get an expression for $J_2$ by the same techniques. The process is the
same, so we just record the result. We have
\be J_2\approx Q^{-2(\alpha+\beta)}\left(\frac t2\right)^{-2(\alpha+\beta)}
\pazocal Y(-\beta,-\alpha,\gamma,\delta)\pazocal A(-\beta,-\alpha,\gamma,\delta).\ee
We now arrive at the following conjecture.
\begin{conjecture}\label{conj:pairratios}
  With constraints on $\alpha,\beta,\gamma,\delta$, we have
  \be\begin{aligned}[t]
    &\T=\int_0^T\frac{L(s+\alpha,f\otimes\psi_d)L(1-s+\beta,\overline f\otimes\psi_d)}
    {L(s+\gamma,f\otimes\psi_d)L(1-s+\delta,\overline f\otimes\psi_d}\d t \\
    &=\int_0^T\Bigg[
    \frac{L\left(1+\alpha+\beta,(f\otimes\psi_d)\otimes(\overline f\otimes\psi_d)\right)
      L\left(1+\gamma+\delta,(f\otimes\psi_d)\otimes(\overline f\otimes\psi_d)\right)}
    {L\left(1+\alpha+\delta,(f\otimes\psi_d)\otimes(\overline f\otimes\psi_d)\right)
      L\left(1+\beta+\gamma,(f\otimes\psi_d)\otimes(\overline f\otimes\psi_d)\right)}
    \pazocal A(\alpha,\beta,\gamma,\delta) \\
    &\begin{multlined}+\left(\frac{\sqrt M|d|t}{2\pi}\right)^{-2(\alpha+\beta)}
    \frac{L\left(1-\alpha-\beta,(f\otimes\psi_d)\otimes(\overline f\otimes\psi_d)\right)
      L\left(1+\gamma+\delta,(f\otimes\psi_d)\otimes(\overline f\otimes\psi_d)\right)}
    {L\left(1-\beta+\delta,(f\otimes\psi_d)\otimes(\overline f\otimes\psi_d)\right)
      L\left(1-\alpha+\gamma,(f\otimes\psi_d)\otimes(\overline f\otimes\psi_d)\right)}
    \times \\ \times
    \pazocal A(-\beta,-\alpha,\gamma,\delta)\Bigg]\d t
    +O\left(T^{1/2+\vep}\right),\end{multlined}
  \end{aligned}\ee
  where $\pazocal A$ is defined at~\eqref{eq:defpazoA}.
\end{conjecture}
For the constraints on the size of the shifts, we follow~\cite{CS} and insist that for
$\alpha$ a generic shift in the numerator ($\alpha$ and $\beta$ in the above) and $\delta$
a generic shift in the denominator ($\gamma$, $\delta$ in the above), we have
\begin{align}
  &-\frac14<\Re\alpha<\frac14,\\ &\frac1{\log T}\ll\Re\delta<\frac14, \\
  &\Im\alpha,\Im\delta\ll_\vep T^{1-\vep}\quad(\forall\vep>0).
\end{align}
For the application to the pair-correlation, we need the following
\begin{theorem}\label{thm:pairlogderiv}
  Assuming Conjecture~\ref{conj:pairratios}, we have
  \be\begin{aligned}[b]
    &\int_0^T \frac{L'}{L}(s+\alpha,f\otimes\psi_d)
    \frac{L'}{L}(1-s+\beta,\overline f\otimes\psi_d)\d t \\
    &=\int_0^T\begin{aligned}[t]
      &\left(\frac{\Lunram'}\Lunram\right)'
      \left(1+\alpha+\beta,(f\otimes\psi_d)\otimes(\overline f\otimes\psi_d)\right)
      +\frac{1}{c_{f\otimes\psi_d}^2}
      \left(\frac{\sqrt M|d|t}{2\pi}\right)^{-2(\alpha+\beta)} \\
      &\times L(1+\alpha+\beta,(f\otimes\psi_d)\otimes(\overline f\otimes\psi_d))
      L(1-\alpha-\beta,(f\otimes\psi_d)\otimes(\overline f\otimes\psi_d))
      \pazocal A(-\beta,-\alpha,\alpha,\beta) \\
      &-\mathcal B(1+\alpha+\beta)
      +\sum_{p\mid M}\frac{(\log p)^2}
      {\left|\lambda_{f\otimes\psi_d}(p)\right|^{-2}p^{1+\alpha+\beta}-1}\d t
      +O\left(T^{1/2+\vep}\right),
    \end{aligned}\end{aligned}\ee where
  $f$ is new of level $M$, $M|d|^2=N$, $\psi_d$ is the Kronecker character associated to
  a fundamental discriminant $d>0$ prime to $M$,
  $c_{f\otimes\psi_d}=\res_{s=1}L(s,(f\otimes\psi_d)\otimes(\overline f\otimes\psi_d))$
  is given in~\eqref{eq:twistedrsres}, $\mathcal B(s)$ is given in~\eqref{def:mathcalB},
  and $\Lunram(s,(f\otimes\psi)\otimes(\overline f\otimes\psi_d))$ is the `unramified' part
  of the Rankin-Selberg convolution, as defined in~\eqref{def:Lunram}.
\end{theorem}
The appearance of the Rankin-Selberg convolution of the cusp form $f\otimes\psi_d$ in
Theorem~\ref{thm:pairlogderiv} is not a surprise. We take a moment to compare
to~\cite[Theorem 2.5]{CS}. The Rankin-Selberg convolution of $\zeta(s)$ with itself is just
$\zeta(s)$. We see that Theorem~\ref{thm:pairlogderiv} has the same shape as
in the corresponding theorem for $\zeta(s)$. We believe that Theorem~\ref{thm:pairlogderiv}
constitutes a fairly generic expression for the average of a product of shifted logarithmic
derivatives of automorphic $L$-functions, insofar as such formulæ should generally be
controlled by the $L$-function attached to the appropriate automorphic Rankin-Selberg lift.
It is important to note is that on $\GL(2)$, the Rankin-Selberg lift counts $\zeta(s)$
as a factor. It is likely that moments of all automorphic functions also involve
contributions from $\zeta(s)$ to the main term.
\begin{proof}[Proof of Theorem~\ref{thm:pairlogderiv}]
  The theorem follows from differentiating with respect to $\alpha$ and $\beta$ and setting
  $\gamma=\alpha$ and $\delta=\beta$. The calculations follow like more complicated
  analogues of those involved in Conjecture~\ref{conj:ratios}, since we now have a ratio
  with two $\GL(2)$ $L$-functions in the numerator and denominator.
  We detail the appearance of each contribution, and give sufficient indication of how to
  derive each contribution as it appears.

  First, the fact that $L(s,(f\otimes\psi_d)\otimes(\overline f\otimes\psi_d))$ has
  a simple pole at $s=1$ makes evaluating
  \be\frac{\partial^2}{\partial\beta\partial\alpha}
  \left(\frac{\sqrt M|d|t}{2\pi}\right)^{-2(\alpha+\beta)}
  \frac{L\left(1-\alpha-\beta,(f\otimes\psi_d)\otimes(\overline f\otimes\psi_d)\right)
    L\left(1+\gamma+\delta,(f\otimes\psi_d)\otimes(\overline f\otimes\psi_d)\right)}
  {L\left(1-\beta+\delta,(f\otimes\psi_d)\otimes(\overline f\otimes\psi_d)\right)
    L\left(1-\alpha+\gamma,(f\otimes\psi_d)\otimes(\overline f\otimes\psi_d)\right)}
  \pazocal A(-\beta,-\alpha,\gamma,\delta)\ee
  straightforward, since the only term that survives after setting $\gamma=\alpha$,
  $\delta=\beta$ is the one corresponding to successive differentation of the denominator,
  which is given (pre-substitution) by
  \be\begin{aligned}[t]&
    \left(\frac{\sqrt M|d|t}{2\pi}\right)^{-2(\alpha+\beta)}
    \frac{L'\left(1-\beta+\delta,(f\otimes\psi_d)\otimes(\overline f\otimes\psi_d)\right)}
    {\left(L\left(1-\beta+\delta,(f\otimes\psi_d)\otimes(\overline f\otimes\psi_d)\right)\right)^2}
    \frac{L'\left(1-\alpha+\gamma,(f\otimes\psi_d)\otimes(\overline f\otimes\psi_d)\right)}
    {\left(L\left(1-\alpha+\gamma,(f\otimes\psi_d)\otimes(\overline f\otimes\psi_d)\right)\right)^2} \\
    &\times L\left(1-\alpha-\beta,(f\otimes\psi_d)\otimes(\overline f\otimes\psi_d)\right)
    L\left(1+\gamma+\delta,(f\otimes\psi_d)\otimes(\overline f\otimes\psi_d)\right)    
    \pazocal A(-\beta,-\alpha,\gamma,\delta).\end{aligned}\ee
  By the argument principle,
  $\left(L'/L\right)\left(s,(f\otimes\psi_d)\otimes(\overline f\otimes\psi_d)\right)$
  has one simple pole at $s=1$ only with residue $-1$. Per~\eqref{eq:rsres},
  $L(s,(f\otimes\psi_d)\otimes(\overline f\otimes\psi_d))$ has a simple pole at $s=1$
  with residue
  \be\label{eq:twistedrsres}c_{f\otimes\psi_d}
  =\frac{(4\pi)^k\langle f\otimes\psi_d,f\otimes\psi_d\rangle}
  {\Gamma(k)\operatorname{vol}(\Gamma_0(M|d|^2)\backslash\H)}.\ee Therefore
  $\left(L'/L^2\right)\left(s,(f\otimes\psi_d)\otimes(\overline f\otimes\psi_d)\right)$ is
  entire and attains the value \be-\frac1{c_{f\otimes\psi_d}}\qquad\text{at $s=1$.}\ee

  Next, straightforward differentiation shows that the contribution from the term
  \be\frac{\partial^2}{\partial\beta\partial\alpha}
  \frac{L\left(1+\alpha+\beta,(f\otimes\psi_d)\otimes(\overline f\otimes\psi_d)\right)
    L\left(1+\gamma+\delta,(f\otimes\psi_d)\otimes(\overline f\otimes\psi_d)\right)}
  {L\left(1+\alpha+\delta,(f\otimes\psi_d)\otimes(\overline f\otimes\psi_d)\right)
    L\left(1+\beta+\gamma,(f\otimes\psi_d)\otimes(\overline f\otimes\psi_d)\right)}
  \pazocal A(\alpha,\beta,\gamma,\delta)\ee in Conjecture~\ref{conj:pairratios}
  is \be\label{eq:positiveratiocontribution}
  \left(\frac{L'}{L}\right)'
  \left(1+\alpha+\beta,(f\otimes\psi_d)\otimes(\overline f\otimes\psi_d)\right)
  \pazocal A(\alpha,\beta,\alpha,\beta)+
  \frac{\partial^2}{\partial\beta\partial\alpha}\left.\pazocal
    A(\alpha,\beta,\gamma,\delta)\right|_{\substack{\gamma=\alpha\\\delta=\beta}}.\ee
  It is a tedious but straightforward computation to verify that
  $\pazocal A(\alpha,\beta,\alpha,\beta)=1$ by reindexing sums and repeatedly applying the
  identity \be\label{eq:lambdamagic}
  \lambda(p)\lambda(p^m)=\lambda(p^{m+1})+\chi(p)\lambda(p^{m-1})\ee
  obtained from~\eqref{eq:fourierrelation}.

  Differentiating $\pazocal A(\alpha,\beta,\gamma,\delta)$ is the most involved part of the
  computation. Since each local factor of
  $\pazocal Y(\alpha,\beta,\gamma,\delta)^{-1}$ and the Euler products
  $W_\mid(\alpha,\beta,\gamma,\delta)$ and $W_\nmid(\alpha,\beta,\gamma,\delta)$
  are equal to 1 when $\gamma\mapsto\alpha$ and $\delta\mapsto\beta$,
  we can focus on each part of the product separately.

  First, a straightforward calculation shows that
  \be\frac{\partial^2}{\partial\beta\partial\alpha}
  \left.\pazocal Y(\alpha,\beta,\gamma,\delta)^{-1}
  \right|_{\substack{\gamma=\alpha\\\delta=\beta}}
  =-\left(\frac{L'}L\right)'
  (1+\alpha+\beta,(f\otimes\psi_d)\otimes(\overline f\otimes\psi_d)),\ee
  cancelling part of the contribution in~\eqref{eq:positiveratiocontribution}.

  Next, since the Fourier coefficients $\lambda(p)$ are completely multiplicative at
  primes dividing $M$ (and vanish at primes dividing $d$ because of the twist by $\psi_d$),
  we can make short work of the differentiation
  \be\frac{\partial^2}{\partial\beta\partial\alpha}\left.
    W_\mid(\alpha,\beta,\gamma,\delta)\right|_{\substack{\gamma=\alpha\\\delta=\beta}}.\ee
  Let $\nu$ stand in for $1+\alpha+\beta$. A straightforward reindexing and cancelation
  leaves us with a geometric series at each prime given by
  \be\sum_{p\mid M}(\log p)^2\frac{\left|\lambda(p)\right|^2p^{-\nu}}
  {1-\left|\lambda(p)\right|^2p^{-\nu}}.\ee

  Finally, we consider the result of differentiating the Euler product in
  $W_\nmid(\alpha,\beta,\gamma,\delta)$ term-by-term.
  After substituting for $\gamma$ and $\delta$, we obtain the sum over primes
  \be\sum_{p\nmid N}(\log p)^2\left[\clubsuit+\spadesuit\right],\ee where
  \be\begin{aligned}[t]
    &\clubsuit:=\begin{aligned}[t]&
    \sum_{m\geq0}\frac{\left|\lambda(p^m)\right|^2m^2}{p^{\nu m}}
    -\sum_{m\geq0}\frac{\lambda(p^{m+1})\overline\lambda(p^m)\overline\lambda(p)m^2}
    {p^{\nu(m+1)}}
    -\sum_{m\geq0}\frac{\overline\lambda(p^{m+1})\lambda(p^m)\lambda(p)m^2}
    {p^{\nu(m+1)}} \\
    &+\sum_{m\geq0}\frac{\overline\lambda(p^{m})\lambda(p^{m+2})\overline\chi(p)m^2}
    {p^{\nu(m+2)}}
    +\sum_{m\geq0}\frac{\lambda(p^{m})\overline\lambda(p^{m+2})\chi(p)m^2}
    {p^{\nu(m+2)}}
    +\sum_{m\geq0}\frac{\left|\lambda(p^m)\right|^2\left|\lambda(p)\right|m^2}
    {p^{\nu(m+1)}} \\
    &-\sum_{m\geq0}\frac{\lambda(p^{m+1})\overline\lambda(p^m)\lambda(p)\overline\chi(p)m^2}
    {p^{\nu(m+2)}}
    -\sum_{m\geq0}\frac{\overline\lambda(p^{m+1})\lambda(p^m)\overline\lambda(p)\chi(p)m^2}
    {p^{\nu(m+2)}}
    +\sum_{m\geq0}\frac{\left|\lambda(p^m)\right|^2\left|\chi(p)\right|^2m^2}{p^{\nu(m+2)}},
    \quad\text{and}
  \end{aligned} \\
  &\spadesuit:=\begin{aligned}[t]
    &-\sum_{m\geq0}\frac{\lambda(p^{m+1})\overline\lambda(p^m)\overline\lambda(p)m}
    {p^{\nu(m+1)}}
    -\sum_{m\geq0}\frac{\lambda(p^m)\overline\lambda(p^{m+1})\lambda(p)m}
    {p^{\nu(m+1)}}
    +\sum_{m\geq0}\frac{2\overline\lambda(p^m)\lambda(p^{m+2})\overline\chi(p)m}
    {p^{\nu(m+2)}} \\
    &+\sum_{m\geq0}\frac{2\lambda(p^m)\overline\lambda(p^{m+2})\chi(p)m}
    {p^{\nu(m+2)}}
    -\sum_{m\geq0}\frac{\lambda(p^{m+1})\overline\lambda(p^m)\lambda(p)\overline\chi(p)m}
    {p^{\nu(m+2)}}
    -\sum_{m\geq0}\frac{\lambda(p^m)\overline\lambda(p^{m+1})\overline\lambda(p)\chi(p)m}
    {p^{\nu(m+2)}}.
  \end{aligned}\end{aligned}\ee
Next, we re-index the sums in $\clubsuit$ and $\spadesuit$ to put them over a common
denominator of $p^{\nu m}$. After this reindexing, we again group the sums in
terms of $m^2$, $m$, and now also $1$ coefficients. Making frequent recourse
to~\eqref{eq:lambdamagic}, we find that the sums with coefficients in $m^2$ and $m$
evaluate to zero, and we are left with
\be\sum_{p\nmid N}(\log p)^2\left[\frac{\left|\lambda(p)\right|^2}{p^{1+\alpha+\beta}}
  +\sum_{m\geq0}
  \frac{(\lambda(p^{m+2})-\chi(p)\lambda(p^m))
    (\overline\lambda(p^{m+2})-\overline\chi(p)\overline\lambda(p^m))}
  {p^{\nu(m+2)}}\right].\ee
We now expand $\lambda(p^{m+2})-\chi(p)\lambda(p^m)$ in terms of Satake parameters
attached to $f\otimes\psi_d$ (c.f.~\eqref{eq:satakeidentity}) and write
\be\lambda(p^{m+2})-\chi(p)\lambda(p^m)
=\sum_{\ell=0}^{m+2}\alpha(p)^\ell\beta(p)^{m+2-\ell}
-\alpha(p)\beta(p)\sum_{k=0}^m\alpha(p)^k\beta(p)^{m-k}
=\alpha(p)^{m+2}+\beta(p)^{m+2}.\ee
It is then a simple matter to obtain the expression
\be\begin{aligned}[b]&\sum_{p\nmid N}(\log p)^2\left[
  \frac{\lambda(p)\overline\lambda(p)}{p^\nu}
  +\frac{(\alpha(p)\overline\alpha(p))^2p^{-2\nu}}{1-\alpha(p)\overline\alpha(p)p^{-\nu}}
  +\frac{(\alpha(p)\overline\beta(p))^2p^{-2\nu}}{1-\alpha(p)\overline\beta(p)p^{-\nu}}
  +\frac{(\overline\alpha(p)\beta(p))^2p^{-2\nu}}{1-\overline\alpha(p)\beta(p)p^{-\nu}}
  +\frac{(\beta(p)\overline\beta(p))^2p^{-2\nu}}{1-\beta(p)\overline\beta(p)p^{-\nu}}
\right] \\
&=\sum_{p\nmid N}\left[
  \frac{(\alpha(p)\overline\alpha(p)\log^2p)p^{-\nu}}{1-\alpha(p)\overline\alpha(p)p^{-\nu}}
  +\frac{(\alpha(p)\overline\beta(p)\log^2p)p^{-\nu}}{1-\alpha(p)\overline\beta(p)p^{-\nu}}
  +\frac{(\overline\alpha(p)\beta(p)\log^2p)p^{-\nu}}{1-\overline\alpha(p)\beta(p)p^{-\nu}}
  +\frac{(\beta(p)\overline\beta(p)\log^2p)p^{-\nu}}{1-\beta(p)\overline\beta(p)p^{-\nu}}
\right] \\
&=\begin{aligned}[t]
  &\left(\frac{\Lunram'}\Lunram\right)'
  \left(\nu,(f\otimes\psi_d)\otimes(\overline f\otimes\psi_d)\right) \\
  &-\sum_{p\nmid N}\left[
  \left(\frac{\alpha(p)\overline\alpha(p)\log p}{p^\nu-\alpha(p)\overline\alpha(p)}\right)^2
  +\left(\frac{\alpha(p)\overline\beta(p)\log p}{p^\nu-\alpha(p)\overline\beta(p)}\right)^2
  +\left(\frac{(\overline\alpha(p)\beta(p)\log p}{p^\nu-\overline\alpha(p)\beta(p)}\right)^2
  +\left(\frac{\beta(p)\overline\beta(p)\log p}{p^\nu-\beta(p)\overline\beta(p)}\right)^2
\right]
\end{aligned} \\
&:=\left(\frac{\Lunram'}\Lunram\right)'
\left(\nu,(f\otimes\psi_d)\otimes(\overline f\otimes\psi_d)\right)
-\mathcal B(\nu),\quad\text{say,}\label{def:mathcalB}
\end{aligned}\ee
where $\Lunram$ denotes the `unramified' part of the $L$-function and may be given
explicitly in the half-plane of convergence of the Euler product by
\be\label{def:Lunram}\Lunram(s)=\prod_{p\nmid N}L_p(s)=L(s)\prod_{p\mid N}L_p(s)^{-1},\ee
where $L_p(s)$ is the appropriate local archimedean factor, and therefore is equal to
$L(s)$ times the partial Euler product at primes dividing the level. (It therefore inherits
the analytic continuation, etc. of $L(s)$.)
\end{proof}

\subsection{Contour estimation}
We now use contour integration and Theorem~\ref{thm:pairlogderiv} to evaluate the pair
correlation for $L(s,f\otimes\psi_d)$.
Let $1/2+1/\log T<a<b<3/4$, let $\pazocal C_1$ be the positively-oriented contour with
corners at $a,a+iT,1-a+iT$, and $1-a$, and let $\pazocal C_2$ be the contour with corners
at $b,b+iT,1-b+iT$, and $1-b$. We want to evaluate the following sum on GRH:
\be\label{eq:paircorrcontour}
P(\varphi)=\sum_{0<\gamma,\gamma'<T}\varphi\left(\gamma-\gamma'\right)=\frac1{(2\pi i)^2}
\int_{\pazocal C_1}\int_{\pazocal C_2}\frac{L'}{L}(z,f\otimes\psi_d)
\frac{L'}{L}(w,f\otimes\psi_d)\varphi(-i(z-w))\d w\d z.\ee
A standard convexity estimate for $\GL(2)$ confirms that the integrals along the horizontal
sides can be absorbed into the error term of Theorem~\ref{thm:pairlogderiv} and can be
ignored.
We end up with four double integrals $I_1,\ldots,I_4$.
Let $I_1$ denote the integral with vertical parts $a$ and $b$.
Let $I_2$ have vertical parts $1-a$ and $1-b$, $I_3$ have vertical parts $a$ and $1-b$,
and $I_4$ have vertical parts $1-a$ and $b$.

We may move the contours in $I_1$ to the right on GRH without picking up any residues.
Integrating term-by-term, we see $I_1\ll T^\vep$.

For $I_2$, we use the functional equation~\eqref{eq:logderivfuneq} (after $1-s\mapsto s$)
for both terms and find similarly that
\be I_2=\frac1{(2\pi)^2}\int_0^T\int_0^T\frac{\gf'}{\gf}(1/2+iu)
\frac{\gf'}{\gf}(1/2+iv)\varphi(u-v)\d u\d v+O(T^\vep).\ee
The asymptotic \be\frac{\gf'}{\gf}(1/2+it)=-2\log\left(A t\right)
\left(1+O\left(\frac 1t\right)\right),\qquad A:=\frac{\sqrt M |d|}{2\pi}\ee is valid for
$t\ra\infty$; together with the fact $\varphi$ is even and with a substitution $u=v+\rho$
and change in the order of integration, and then subsequent substitution $v\mapsto vT$, we
have
\be\label{eq:I2}\begin{aligned}[b]
  I_2&=\frac8{(2\pi)^2}\int_0^T\int_v^T\log(Au)\log(Av)\varphi(u-v)\d u\d v+O(T^\vep) \\
  &=\frac8{(2\pi)^2}\int_0^T\varphi(\rho)\int_0^{T-\rho}\log(A(v+\rho))\log(Av)\d v\d\rho
  +O(T^\vep) \\
  &=\frac8{(2\pi)^2}T\int_0^T\varphi(\rho)\int_0^{1-\rho/T}\log(A(vT+\rho))\log(AvT)
  \d v\d\rho+O(T^\vep) \\
  &=\frac8{(2\pi)^2}T\int_0^T\varphi(\rho)\int_0^1\log(A(vT+\rho))\log(AvT)
  \d v\d\rho+O(T^\vep) \\
  &=\frac8{(2\pi)^2}T\int_0^T\varphi(\rho)\int_0^1\log^2(AvT)\d v\d\rho+O(T^\vep) \\
  &=\frac4{(2\pi)^2}\int_{-T}^T\varphi(\rho)\int_0^T\log^2(Av)\d v\d\rho+O(T^\vep).
\end{aligned}\ee
Recall $\varphi(x)\ll1/(1+x^2)$ for real $x$ as $x\ra+\infty$.
This allows us to extend the upper limit of integration of the inner integral to $v=1$ in
the third-to-last line, with an error of $\log^3 T$. It also allows us to replace
$\log(A(vT+\rho))$ with $\log(AvT)$ in the second-to-last line, also with an error of
$\log^3 T$. Evenness of $\varphi$ allows us to make the last step.

We now turn to $I_3$. We substitute $z=w+\rho$ (for a complex variable $\rho$) and exchange
the order of integration to integrate over $\rho$ and write
\be\begin{aligned}[b]
  I_3&=\frac{-1}{(2\pi i)^2}\int_{1-b}^{1-b+iT}\int_a^{a+iT}\frac{L'}{L}(w,f\otimes\psi_d)
  \frac{L'}{L}(z,f\otimes\psi_d)\varphi(-i(z-w))\d w\d z \\
  &=\frac{-1}{(2\pi)^2i}\int_{1-a-b-iT}^{1-a-b+iT}\varphi(-i\rho)
  \int_{T_1}^{T_2}\frac{L'}{L}(a+it,f\otimes\psi_d)
  \frac{L'}{L}(a+it+\rho,f\otimes\psi_d)\d t\d\rho,
\end{aligned}\ee
where $T_1=\max\{0,-\Im\rho\}$ and $T_2=\min\{T,T-\Im\rho\}$.
Next, we use the functional equation~\eqref{eq:logderivfuneq}, in the form
\be\frac{L'}{L}(a+\rho+it,f\otimes\psi_d)=\frac{\gf'}{\gf}(a+\rho+it)
-\frac{L'}{L}(1-a-\rho-it,\overline f\otimes\psi_d)\ee and note that the term with
$\gf'/\gf$ is small (as is evident after moving the contour to the right) to write
\be\begin{aligned}[b]
  I_3&=\frac{1}{(2\pi)^2i}\int_{1-a-b-iT}^{1-a-b+iT}\varphi(-i\rho)
  \int_{T_1}^{T_2}\frac{L'}{L}(a+it,f\otimes\psi_d)
  \frac{L'}{L}(1-a-it-\rho,\overline f\otimes\psi_d)\d t\d\rho+O\left(T^\vep\right) \\
  &=\frac{1}{(2\pi)^2i}\int_{1-a-b-iT}^{1-a-b+iT}\varphi(-i\rho)
  \int_{T_1}^{T_2}\frac{L'}{L}(s+(a-1/2),f\otimes\psi_d)
  \frac{L'}{L}(1-s+(1/2-a-\rho),\overline f\otimes\psi_d)\d t\d\rho+O\left(T^\vep\right),
\end{aligned}\ee
where $s=1/2+it$. By Theorem~\ref{thm:pairlogderiv}, we have
\be\label{eq:I3postsub}\begin{aligned}[b]
  I_3\ = \ &
  \frac{1}{(2\pi)^2i}\int_{1-a-b-iT}^{1-a-b+iT}\varphi(-i\rho)
  \int_{T_1}^{T_2}\Bigg[
  \left(\frac{\Lunram'}\Lunram\right)'\left(1-\rho,f_d\otimes\overline f_d\right) \\
  &+\frac{1}{c_{f_d}^2}
  \left(\frac{\sqrt M|d|t}{2\pi}\right)^{2\rho}
  L(1-\rho,f_d\otimes\overline f_d)L(1+\rho,f_d\otimes\overline f_d)
  \pazocal A(-1/2+a+\rho,1/2-a,a-1/2,1/2-a-\rho) \\
  &-\mathcal B(1-\rho)+\sum_{p\mid M}\frac{(\log p)^2}
  {\left|\lambda_{f_d}(p)\right|^{-2}p^{1-\rho}-1}\Bigg]\d t\d\rho
  +O\left(T^{1/2+\vep}\right).
\end{aligned}\ee
Here we have used $f_d$ in place of $f\otimes\psi_d$ to make the notation more compact.
Since we will leave $M$ fixed, we have that the sum over primes dividing $M$ is $O(1)$ and
hence its contribution to $I_3$ may be absorbed into the existing error term.
We let $\delta:=a+b-1$ and let $\mathcal I(-\rho,t)$ stand in for the innermost integrand
in~\eqref{eq:I3postsub} minus the sum over $p\mid M$.
We may then extend the range of the inner integration over $t$ to all of $[0,T]$ with an
error term of size $T^\vep\int_\rho|\rho||f(\rho)|\d\rho\ll T^\vep$. We are left with
\be\label{eq:I3final}
I_3=\frac{1}{(2\pi)^2i}\int_0^T\int_{-\delta-iT}^{-\delta+iT}\varphi(-i\rho)
\mathcal I(-\rho,t)\d\rho\d t+O\left(T^{1/2+\vep}\right).\ee
We now turn to $I_4$. We again let $z=w+\rho$. Then,
\be\begin{aligned}[b]
  I_4&=
  \frac{-1}{(2\pi i)^2}\int_{1-a}^{1-a+iT}\int_b^{b+iT}
  \frac{L'}{L}(w,f\otimes\psi_d)\frac{L'}{L}(z,f\otimes\psi_d)\varphi(-i(z-w))\d z\d w \\
  &=\frac{-1}{(2\pi)^2i}\int_{a+b-1-iT}^{a+b-1+iT}\varphi(-i\rho)\int_{T_1}^{T_2}
  \frac{L'}{L}(1-a+it)\frac{L'}{L}(1-a+it+\rho)\d t\d\rho.
\end{aligned}\ee
We use the functional equation
\be\frac{L'}{L}(1-a+it,f\otimes\psi_d)
=\frac{\gf'}{\gf}(1-a+it)-\frac{L'}{L}(a-it,\overline f\otimes\psi_d).\ee
Again, the term corresponding to $\gf'/\gf$ is $\ll T^\vep$, and we are left with
\be\begin{aligned}[t]
  I_4&=\frac1{(2\pi)^2i}\int_{a+b-1-iT}^{a+b-1+iT}\varphi(-i\rho)\int_{T_1}^{T_2}
  \frac{L'}{L}(a-it,\overline f\otimes\psi_d)
  \frac{L'}{L}(1-a+it+\rho,f\otimes\psi_d)\d t\d\rho+O(T^\vep) \\
  &=\frac1{(2\pi)^2i}\int_{a+b-1-iT}^{a+b-1+iT}\varphi(-i\rho)\int_{T_1}^{T_2}
  \frac{L'}{L}(1-s+(a-1/2),\overline f\otimes\psi_d)
  \frac{L'}{L}(s+(1/2-a+\rho),f\otimes\psi_d)\d t\d\rho+O(T^\vep).
\end{aligned}\ee
We may now replace the integrand per Theorem~\ref{thm:pairlogderiv} and obtain
\be\begin{aligned}[b]
  I_4\ =\ &\frac1{(2\pi)^2i}\int_{a+b-1-iT}^{a+b-1+iT}\varphi(-i\rho)\int_{T_1}^{T_2}
  \Bigg[\left(\frac{\Lunram'}\Lunram\right)'
      \left(1+\rho,(f\otimes\psi_d)\otimes(\overline f\otimes\psi_d)\right) \\
  &+\frac{1}{c_{f_d}^2}
  \left(\frac{\sqrt M|d|t}{2\pi}\right)^{-2\rho}
  L(1+\rho,f_d\otimes\overline f_d)L(1-\rho,f_d\otimes\overline f_d)
  \pazocal A(1/2-a,a-1/2-\rho,1/2-a+\rho,a-1/2) \\
  &-\mathcal B(1+\rho)
  +\sum_{p\mid M}\frac{(\log p)^2}{\left|\lambda_{f_d}(p)\right|^{-2}p^{1+\rho}-1}\Bigg]
  \d t\d\rho+O\left(T^{1/2+\vep}\right),
\end{aligned}\ee
with $f_d=f\otimes\psi_d$ as above.
Re-using the same notation as in the discussion of $I_3$, we again extend the range of
integration, absorb the sum over $p\mid M$ into the existing error, and write
\be\label{eq:I4final}
I_4=\frac1{(2\pi)^2i}\int_0^T\int_{\delta-iT}^{\delta+iT}\varphi(-i\rho)
\mathcal I(\rho,t)\d\rho\d t+O\left(T^{1/2+\vep}\right).\ee
To be able to write the above, we must have that
\be\label{def:mathcalA}\mathcal A(\rho):=\pazocal A(-1/2+a+\rho,1/2-a,a-1/2,1/2-a-\rho)
=\pazocal A(1/2-a,a-1/2+\rho,1/2-a-\rho,a-1/2).\ee
This identity may be verified straightaway.
Combining~\eqref{eq:I4final} with the expression obtained for $I_3$ in~\eqref{eq:I3final},
we have, recalling that $\varphi$ is even and changing variables in $I_3$,
\be\label{eq:I3I4}
I_3+I_4=\frac2{(2\pi)^2i}\int_0^T\int_{\delta-iT}^{\delta+iT}\varphi(i\rho)
\mathcal I(\rho,t)\d\rho\d t+O\left(T^{1/2+\vep}\right).\ee
We have that $\mathcal I(\rho,t)$ assumes the form
\begin{multline}\mathcal I(\rho,t)=\left(\frac{\Lunram'}\Lunram\right)'
\left(1+\rho,(f\otimes\psi_d)\otimes(\overline f\otimes\psi_d)\right)-\mathcal B(1+\rho) \\
+\frac{1}{c_{f\otimes\psi_d}^2}\left(\frac{\sqrt M|d|t}{2\pi}\right)^{-2\rho}
L(1+\rho,(f\otimes\psi_d)\otimes(\overline f\otimes\psi_d))
L(1-\rho,(f\otimes\psi_d)\otimes(\overline f\otimes\psi_d))
\mathcal A(\rho).\end{multline}
It is straightforward but laborious to verify that $\mathcal A'(0)=0$.
(We have from before that $\mathcal A(0)=1$.)
With these facts in hand, and recalling the factorization of $L(s,f_d\otimes\overline f_d)$
given in~\eqref{eq:adfactorization} and the special value~\eqref{eq:ad1}, we see that the
first term of $\mathcal I(\rho,t)$ has a Laurent series development at $\rho=0$ given by
\be\begin{aligned}[b]&\begin{multlined}[b]\frac1{c_{f\otimes\psi_d}^2}
\left(1-2\log\left(\frac{\sqrt M|d|t}{2\pi}\right)+O(\rho^2)\right)
\left(\frac1\rho+\gamma+O(\rho)\right)
\left(c_{f\otimes\psi_d}+L'(1,\ad^2(f\otimes\psi_d))\rho+O(\rho^2)\right) \\
\times\left(-\frac1\rho+\gamma+O(\rho)\right)
\left(c_{f\otimes\psi_d}-L'(1,\ad^2(f\otimes\psi_d))\rho+O(\rho^2)\right)
\left(1+O(\rho^2)\right)\end{multlined} \\
&=-\frac1{\rho^2}+\frac{2\log\left(\sqrt M|d|t/2\pi\right)}{\rho}+O(1).
\end{aligned}\ee
Since $(\Lunram'/\Lunram)'(s,f_d\otimes\overline f_d)$ differs from
$(L'/L)'(s,f_d\otimes\overline f_d)$ by a sum over a finite number of primes that
constitutes an entire function of $s$, so
\be\left(\frac{\Lunram'}{\Lunram}\right)'(1+\rho,f_d\otimes\overline f_d)
=\frac1{\rho^2}+O(1).\ee
It is then the case that $\mathcal I(\rho,t)$ has a Laurent development
about $\rho=0$ given by
\be\mathcal I(\rho,t)=\frac{2\log\left(\sqrt M|d|t/2\pi\right)}{\rho}+O(1).\ee
We now want to let $\delta\ra0$ and pass the line of integration in $\rho$ to the imaginary
axis from $-T$ to $T$. Since it becomes singular at $\rho=0$, we regard it as a principal
value as we pass through 0 and add the contribution from half the residue of
$\mathcal I(\rho,t)$ at $\rho=0$ in the integral~\eqref{eq:I3I4}, which is
\be\frac1{\pi}\int_0^T\varphi(0)\log\left(\frac{\sqrt M|d|t}{2\pi}\right)\d t.\ee
\subsection{Pair-correlation theorem and expansion}\label{sec:pcthm}
Combining our expression for $I_1$, $I_2$~\eqref{eq:I2}, and $I_3+I_4$~\eqref{eq:I3I4} and
changing variables $\rho\mapsto ir$, we have proved the following.
\begin{theorem}
  Assuming Conjecture~\ref{conj:pairratios}, and with $\varphi$
  satisfying~\eqref{def:testconditions}, we have
  \be\begin{aligned}[b]P(\varphi)
    \ =\ &\sum_{0<\gamma,\gamma'<T}\varphi\left(\gamma-\gamma'\right)
    =\frac2{(2\pi)^2}\int_0^T\Bigg[
    2\pi\varphi(0)\log\left(\frac{\sqrt M|d|t}{2\pi}\right)
    +\int_{-T}^T\varphi(r)\Bigg(
    2\log^2\left(\frac{\sqrt M|d|t}{2\pi}\right) \\
    &+\left(\frac{\Lunram'}\Lunram\right)'
    \left(1+ir,(f\otimes\psi_d)\otimes(\overline f\otimes\psi_d)\right) \\
    &+\frac{1}{c_{f\otimes\psi_d}^2}
    \left(\frac{\sqrt M|d|t}{2\pi}\right)^{-2 ir}
    L(1+ir,(f\otimes\psi_d)\otimes(\overline f\otimes\psi_d))
    L(1-ir,(f\otimes\psi_d)\otimes(\overline f\otimes\psi_d))\mathcal A(ir) \\
    &-\mathcal B(1+ir)\Bigg)\d r\Bigg]\d t+O\left(T^{1/2+\vep}\right),
  \end{aligned}\ee
  where the inner integral is to be regarded as a principal value near $r=0$,
  $f$ is new of level $M$, $M|d|^2=N$, $\psi_d$ is the Kronecker character associated to
  a fundamental discriminant $d>0$ prime to $M$,
  $c_{f\otimes\psi_d}=\res_{s=1}L(s,(f\otimes\psi_d)\otimes(\overline f\otimes\psi_d))$
  is given in~\eqref{eq:twistedrsres},
  $\mathcal A(\rho)$ is defined in~\eqref{def:mathcalA} together with~\eqref{eq:defpazoA},
  $\mathcal B(1+\rho)$ is defined in~\eqref{def:mathcalB},
  and $\Lunram(s,(f\otimes\psi)\otimes(\overline f\otimes\psi_d))$ is the `unramified' part
  of the Rankin-Selberg convolution, as defined in~\eqref{def:Lunram}.
\end{theorem}
We now want to obtain a series development for large $T$ of $P(\varphi)$. As in the
one-level density, we rescale variables and test functions by the mean density.

It is a fact that if we let $L(s,f)$ be the $L$-function of a cusp form $f$ of level $N$
on $\GL(2)$ and $N(T,f)$ be the number of zeros $\rho=\beta+i\gamma$ of $L(s,f)$ such that
$0\leq\beta\leq1$ and $0<\gamma\leq T$, then
\be N(T,f)=\frac T\pi\log\left(\frac{\sqrt N T}{2\pi e}\right)+O(\log T).\ee
Therefore, we let
\be\label{eq:Rdef2} y = \tfrac r \pi R,
\quad\text{where }R:=\log \left(\tfrac{\sqrt M |d|T}{2\pi e}\right),\ee
and define the rescaled test function $g$ implicitly by
\be\label{eq:gdef2} \varphi(r)= g\left(\tfrac r \pi R\right).\ee
We change variables and obtain
\be\begin{aligned}[b]
  &\sum_{0<\gamma,\gamma'<T}\varphi\left((\gamma-\gamma')\frac R\pi\right) \\
  &=\ \begin{aligned}[t]
    &\frac2{(2\pi)^2}\int_0^T\Bigg[
    2\pi g(0)\log\left(\frac{\sqrt M|d|t}{2\pi}\right)
    +\frac \pi R\int_{-T(R/\pi)}^{T(R/\pi)} g(y)\Bigg(
    2\log^2\left(\frac{\sqrt M|d|t}{2\pi}\right) \\
    &+\left(\frac{\Lunram'}\Lunram\right)'
    \left(1+\frac{i\pi y}R,(f\otimes\psi_d)\otimes(\overline f\otimes\psi_d)\right) \\
    &+\frac{1}{c_{f\otimes\psi_d}^2}
    \left(\frac{\sqrt M|d|t}{2\pi}\right)^{-2\pi i y/R}
    L\left(1+\frac{i\pi y}{R},(f\otimes\psi_d)\otimes(\overline f\otimes\psi_d)\right)
    L\left(1-\frac{i\pi y}{R},(f\otimes\psi_d)\otimes(\overline f\otimes\psi_d)\right)
    \mathcal A\left(\frac{i\pi y}{R}\right) \\
    &-\mathcal B\left(1+\frac{i\pi y}R\right)
    \Bigg)\d y\Bigg]\d t+O\left(T^{1/2+\vep}\right).\end{aligned}\end{aligned}\ee
Next, we integrate over $t$. We remark on some individual points.
Very explicitly, \begin{gather}\int_0^T\log\left(\frac{\sqrt M|d|T}{2\pi}\right)\d t
=T\left(\log\left(\frac{\sqrt M|d|T}{2\pi}\right)-1\right)
=T\log\left(\frac{\sqrt M|d|T}{2\pi e}\right), \\
\int_0^T\log^2\left(\frac{\sqrt M|d|T}{2\pi}\right)\d t
=T\left(1+\log^2\left(\frac{\sqrt M|d|T}{2\pi e}\right)\right)=T(1+R^2),\text{ and} \\
\int_0^T\left(\frac{\sqrt M|d|T}{2\pi}\right)^{-2\pi iy/R}\d t
=\frac{T(AT)^{-2\pi iy/R}}{1-2\pi iy/R}=\frac{Te^{-2\pi iy(1+1/R)}}{1-2\pi iy/R}.
\end{gather}
These are the only terms that appear with $t$-dependence. We have
\be\begin{aligned}[b]
  &\sum_{0<\gamma,\gamma'<T}\varphi\left((\gamma-\gamma')\frac R\pi\right) \\
  &=\ \begin{aligned}[t]
    &\frac T\pi\log\left(\frac{\sqrt M|d|T}{2\pi e}\right)
    \Bigg[g(0)+\frac1{2R^2T}\int_{-\infty}^\infty g(y)\Bigg(
    2T(R^2+1)+T\left(\frac{\Lunram'}\Lunram\right)'
    \left(1+\frac{i\pi y}R,(f\otimes\psi_d)\otimes(\overline f\otimes\psi_d)\right) \\
    &+\frac{1}{c_{f\otimes\psi_d}^2}
    \left[\frac{Te^{-2\pi iy(1+1/R)}}{1-2\pi iy/R}\right]
    L\left(1+\frac{i\pi y}{R},(f\otimes\psi_d)\otimes(\overline f\otimes\psi_d)\right)
    L\left(1-\frac{i\pi y}{R},(f\otimes\psi_d)\otimes(\overline f\otimes\psi_d)\right)
    \mathcal A\left(\frac{i\pi y}{R}\right) \\
    &-T\mathcal B\left(1+\frac{i\pi y}R\right)
    \Bigg)\d y\Bigg]\d t+O\left(T^{1/2+\vep}\right).\end{aligned}\end{aligned}\ee
We now expand in large $R$ and arrive at
\be\begin{aligned}[b]
  &\sum_{0<\gamma,\gamma'<T}\varphi\left((\gamma-\gamma')\frac R\pi\right) \\
  &=\ \begin{multlined}[t]
    \frac T\pi\log\left(\frac{\sqrt M|d|T}{2\pi e}\right)
    \Bigg(g(0)+ \\ +\int_{-\infty}^\infty g(y)\Bigg(
    1-\frac1{2\pi^2y^2}
    +\frac{e^{-2\pi iy}}{2\pi^2y^2}-\frac{e_1(1-e^{-2\pi iy})}{R^2}
    -\frac{e_2 e^{-2\pi i y}\pi i y}{R^3}
    +O\left(R^{-4}\right)
    \Bigg)\d y\Bigg)+O\left(T^{1/2+\vep}\right)\end{multlined} \\
  &=\ \begin{multlined}[0.885\textwidth][t]
    \frac T\pi\log\left(\frac{\sqrt M|d|T}{2\pi e}\right)
    \Bigg(g(0)+ \\ +\int_{-\infty}^\infty g(y)\Bigg(
    1-\left(\frac{\sin\pi y}{\pi y}\right)^2
    -\frac{e_1\sin^2\pi y}{R^2}
    -\frac{e_2\pi y\sin2\pi y}{R^3}
    +O\left(R^{-4}\right)
    \Bigg)\d y\Bigg)
    +O\left(T^{1/2+\vep}\right)\end{multlined} \label{eq:pairexpansion} \\
  &\sim
    N\left(T,f\otimes\psi_d\right)
    \left(g(0)+\int_{-\infty}^\infty g(y)\Bigg(
    1-\left(\frac{\sin\pi y}{\pi y}\right)^2    
    \Bigg)\d y\right),
\end{aligned}\ee
where \begin{align}\label{eq:a1pc}
e_1&=-2+\gamma^2+2\gamma_1-\frac{\mathcal A''(0)}{2}
+\frac{L'(1,\ad^2f\otimes\psi_d)^2}{c_{f\otimes\psi_d}^2}
     -\frac{L''(1,\ad^2,f\otimes\psi_d)}{c_{f\otimes\psi_d}} \\
   &=-2+\gamma^2+2\gamma_1-\frac{\mathcal A''(0)}{2}
     -\left(\frac{L'}{L}\right)'(1,\ad^2 f\otimes\psi_d),\quad\text{and} \\ \label{eq:a2pc}
        e_2&=\frac{16+\mathcal A'''(0)}{12}.\end{align}
To obtain the arithmetic constants $e_1$ and $e_2$, it is necessary to express the various
$L$-functions that appear in terms of possible factorizations, where applicable.
To obtain $e_1$, it is also necessary to use the identity
\be\frac{\mathcal A''(0)}{2}=-\mathcal B(1)-\left(\frac{\Lram'}{\Lram}\right)'
\left(1,(f\otimes\psi_d)\otimes(\overline f\otimes\psi_d)\right),\ee
which is laborious to obtain.
(Here, $\Lram(s)=L/\Lunram(s)$ is entire.)
The expression $1-(\sin^2\pi y)/(\pi y)^2$ is the limiting two-point correlation function
predicted by Montgomery~\cite{montgomery}.
The (scaled) pair-correlation $\pazocal Q_{U(N)}(v)$ for $U(N)$
(c.f., e.g.~\cite{conreymatrix}) is \be\label{eq:Upc}
\pazocal Q_{U(N)}(v)=1-\left(\frac1 N\frac{\sin(\pi v)}{\sin(\pi v/N)}\right)^2
=1-\left(\frac{\sin\pi v}{\pi v}\right)^2-\frac{\sin^2\pi v}{3N^2}+O\left(N^{-4}\right).\ee
We see good agreement between~\eqref{eq:pairexpansion} and~\eqref{eq:Upc}.
Notably, there is the appearance of a term of order $1/R^3$ in~\eqref{eq:pairexpansion},
which does not appear in $\pazocal Q_{U(N)}(v)$. Bogolmony \etal\ encounter the same
phenomenon when dealing with $\zeta(s)$; they propose a shift in the variable $y$ to absorb
the term of order $1/R^3$ into the term of order $1/R^2$. A similar shift could easily be
made here, but it is not necessary for our application so we do not pursue the argument
here.

Our goal throughout has been to suggest an effective matrix size to model a family of
quadratic twists outside the asymptotic limit. The arithmetic constants $e_1$ and $e_2$ are
easily seen to depend on the particular cusp form $f\otimes\psi_d$.
It is not clear how to obtain expressions for expectations over $d$ of $e_1$ or $e_2$; we
propose that, to find the effective matrix size over some range of $d$, the expectation of
the arithmetic constants over $d$ in that range should be found numerically.
So, for the $L$-function attached to the fixed form $f\otimes\psi_d$, we should have an
effective matrix size (useful for approximating local statistics outside the asymptotic
limit for $L(s,f\otimes\psi_d)$) given by \be\neff=\frac R{\sqrt{3e_1}}.\ee
For a family of quadratic twists where $d$ varies over some set, we need to find the
average $\langle e_1\rangle$ of the arithmetic constant over the set, and then $\neff$ for
the family is given by \be\neff=\frac R{\sqrt{3\langle e_1\rangle}}.\ee

\appendix

\section{Counting fundamental discriminants}\label{sec:funcountproof}
\begin{lemma}\label{lem:funcount}
  Let $\mathcal D(X)$ denote the set of fundamental discriminants $d$ that satisfy
  $0<d\leq X$, and fix $M$ an odd prime and $\heartsuit\in\{\pm1\}$. Then
  \be\label{eq:funasymp1}
  \sum_{\substack{d\in\mathcal D(X)\\\left(\frac d M\right)=\heartsuit}}1
  =\frac3{\pi^2}\frac M{2(M+1)}X+O\left(X^{1/2}\right),\ee
  where $\left(\frac\cdot M\right)$ denotes the Legendre symbol.
  Fix $0<\diamondsuit<M$ an integer. Then
  \be\label{eq:funasymp2}\sum_{\substack{d\in\mathcal D(X)\\d\equiv\diamondsuit\mod M}}1
  =\frac3{\pi^2}\frac M{M^2-1}X+O\left(X^{1/2}\right).\ee
\end{lemma}
\begin{proof}
  There are two different asymptotics to establish.
  Modulo trivial modification,~\eqref{eq:funasymp1} is done in~\cite[Lemma A.1]{HMM}, and
  we do not repeat the proof here.

  Equation~\eqref{eq:funasymp2} follows in much the same way, and may be viewed as
  complementary to~\cite[Lemma~B.1]{mil08}. We have
  \be\sum_{\substack{d\in\mathcal D(X)\\d\equiv\diamondsuit\mod M}}1
  =\sum_{\substack{0<d\leq X\\d\equiv1\mod4\\d\equiv\diamondsuit\mod M}}\mu(d)^2
  +\sum_{\substack{0<d\leq X\\4\mid d\\d/4\equiv2\mod4\\d\equiv\diamondsuit\mod M}}\mu(d/4)^2
  +\sum_{\substack{0<d\leq X\\4\mid d\\d/4\equiv3\mod4\\d\equiv\diamondsuit\mod M}}\mu(d/4)^2.\ee
  Using elementary group operations and the Chinese remainder theorem, we can rewrite this
  trio of sums as
  \be\label{eq:chinese}\sum_{\substack{d\in\mathcal D(X)\\d\equiv\diamondsuit\mod M}}1
  =\sum_{\substack{0<d\leq X\\d\equiv A\mod 4M}}\mu(d)^2
  +\sum_{\substack{0<d\leq X/4\\d\equiv A'\mod 4M}}\mu(d)^2
  +\sum_{\substack{0<d\leq X/4\\d\equiv A''\mod 4M}}\mu(d)^2.\ee
  where $(A,4M)=1=(A'',4M)$ and $(A',4M)=2$.
  We recall the identity \be\mu(n)^2=\sum_{m^2\mid n}\mu(m)\ee and note that as $m$ ranges
  over the integers, $(m^2,4M)$ is never 2, but $A'\equiv2\mod4$.
  With these facts in hand, we let $A^\circ$ stand in for $A$, $A'$, or $A''$, exchange the
  order of summation, and write
  \be\label{eq:slinkysum}
  \begin{aligned}[b]\sum_{\substack{0<d\leq X\\d\equiv A^\circ\mod 4M}}\mu(d)^2
    &=\sum_{\substack{0<d\leq X\\d\equiv A^\circ\mod 4M}}\sum_{m^2\mid d}\mu(m) \\
    &=\sum_{m\leq X^{1/2}}\mu(m)\sum_{\substack{d'\leq X/m^2\\d'm^2\equiv A^\circ\mod 4M}}1 \\
    &=\sum_{\substack{m\leq X^{1/2}\\(m^2,4M)\mid A^\circ}}\mu(m)
    \left(\frac X{m^2}\frac{(m^2,4M)}{4M}+O(1)\right) \\
    &=\frac{X}{4M}\sum_{\substack{m\leq X^{1/2}\\(m^2,4M)\mid A^\circ}}
    \frac{\mu(m)(m^2,4M)}{m^2}+O\left(X^{1/2}\right) \\
    &=\frac X{4M}\sum_{\substack{m\leq X^{1/2}\\(m^2,4M)=1}}\frac{\mu(m)}{m^2}
    +O\left(X^{1/2}\right) \\
    &=\frac X{4M}\frac{\left(1-\frac14\right)^{-1}\left(1-\frac1{M^2}\right)^{-1}}
    {\zeta(2)}+O\left(X^{1/2}\right) \\
    &=\frac{2}{\pi^2}\frac{M}{M^2-1}X+O\left(X^{1/2}\right).
  \end{aligned}\ee
  Inputting into~\eqref{eq:chinese} yields the claim.
\end{proof}

\section{Guessing the symmetry constant}
\subsection{Quadratic twists}
Consider first the family $\F^+(X)$ of even quadratic twists of a fixed holomorphic
cuspidal newform, i.e. primitive Hecke cusp form $f\in S^\star_k(M,\chi_f)$. For each
fundamental discriminant $d>0$, let $\psi_d=\left(\frac d\cdot\right)$ be the Jacobi
symbol. The
asymptotic parameter of our family is then $d$. Let $M$, the level of our fixed form $f$,
be prime. Note $f\otimes\psi_d$ is a primitive cusp form, provided $(d,M)=1$; restrict
henceforth to such $d$ not divisible by $M$. If $f$ admits the Fourier expansion
\be f(z)=\sum_{n=1}^\infty a_f(n)n^{(k-1)/2}e(nz),\ee then $f\otimes\psi_d$ has the
expansion \be f\otimes\psi_d(z)=\sum_{n=1}^\infty\psi_d(n)a_f(n)n^{(k-1)/2}e(nz).\ee
Therefore, the Fourier coefficients $a_{f\otimes\psi_d}(n)$ equal the Fourier coefficients
twisted by the character $\psi_d$. Also, $\chi_{f\otimes\psi_d}=\psi_d^2\chi_f$, and the
level of $f\otimes\psi_d$ equals $Md^2$, since $\chi_d$ is primitive of modulus $d$ prime
to $M$.
Therefore, with the restriction in place on $d$ and since $f\in S^\star_k(M,\chi_f)$, we
have $f\otimes\psi_d\in S^\star_k\left(Md^2,\psi_d^2\chi_f\right)$. Notating the Satake
parameters of $f$ as $\alpha_f$, $\beta_f$, we have that, per~\cite[Definition 2.1]{conv},
\be b_f\left(p^2\right)=\alpha_f(p)^2+\beta_f(p)^2,\ee where $b_f(n)$ is the coefficient of
the Dirichlet series of the logarithmic derivative of $L(s,f)$, which is supported on prime
powers only. We then have the following heuristic for the symmetry constant of the family,
following~\cite[Definition 1.1(3ii)]{conv}.
\begin{heuristic}\label{heur}
  For some $c_{\F^+}\in\{-1,0,1\}$, \be\label{eq:symsniff}
  -2\sum_p\frac1p\frac{\log p}{\log R_N}
  \hat\phi\left(2\frac{\log p}{\log R_N}\right)\frac1{\left|\F^+(N)\right|}
  \sum_{g\in\F^+(N)}b_g\left(p^2\right)=-c_{\F^+}\frac{\phi(0)}2+o(1),\ee
  where $\phi\in\mathcal S(\R)$ with some compact support, $R_N$ is a free parameter
  we usually take to be $\asymp\left|\F^+(N)\right|$, and $c_{\F^+}$ is
  the symmetry constant of the family.
  Often,~\eqref{eq:symsniff} is satisfied because $\exists\delta_2>0,\mu_2\geq0,
  c_{\F^+}\in\{-1,0,1\}$ such that \be\label{eq:shortsniff}
  \frac1{\left|\F^+(N)\right|}
  \sum_{g\in\F^+(N)}b_g\left(p^2\right)=c_{\F^+}
  +O\left(\left|\F^+(N)\right|^{-\delta_2}p^{\mu_2}\right).\ee
\end{heuristic}
{\it\color{Plum}This is now wrong, since the below has not yet been updated to reflect the
  change in definition of $d$ good.}
\begin{remark};Since our family is parametrized by positive fundamental discriminants $d$,
  the average over the family can be written as an average over fundamental discriminants;
  i.e. \be\frac1{\left|\F^+(X)\right|}\sum_{g\in\F^+(X)}b_g\left(p^2\right)
  =\frac1{X^\ast}\left.\sum_{\mathclap{d\leq X}}\right.^\ast
  b_{f\otimes\psi_d}\left(p^2\right),\ee where the star indicates it is a sum over only
  those fundamental discriminants $d$ such that $f\otimes\psi_d$ is an even form, and
  \begin{align}
    X^\ast&=\left|\left\{0<d\leq X:f\otimes\psi_d\in\F^+\right\}\right|,\text{ and}\\
    X^\ast_p&=\left|\left\{0<d\leq X:p\nmid d,f\otimes\psi_d\in\F^+\right\}\right|
  \end{align}
  are the number of fundamental discriminants less than or equal to $X$ that twist $f$ to a
  form with an even functional equation, and the number of such $d$ indivisible by $p$,
  respectively.
\end{remark}
  
We seek to apply Heuristic~\ref{heur} to our family $\F^+$. We have that
\begin{align}
  b_{f\otimes\psi_d}\left(p^2\right)
  =\alpha_{f\otimes\psi_d}(p)^2+\beta_{f\otimes\psi_d}(p)^2
  =a_{f\otimes\psi_d}\left(p^2\right)-\chi_{f\otimes\psi_d}(p)
  &=a_{f\otimes\psi_d}(p)^2-2\chi_{f\otimes\psi_d}(p) \\
  &=\psi_d(p)^2a_f(p)^2-2\psi_d(p)^2\chi_f(p),
\end{align}
where here we have used the equations
\begin{align}
  a_g(n)&=\alpha_g(n)+\beta_g(n), \\ \chi_g(n)&=\alpha_g(n)\beta_g(n), \\
  a_g(p)^2&=a_g\left(p^2\right)+\chi_g(p),
\end{align} valid for any cusp form $g$. We conclude that since $\psi_d$ is quadratic,
\be b_{f\otimes\psi_d}\left(p^2\right)=\left(a_f(p)^2-2\chi_f(p)\right)\psi_d(p)^2
=\begin{cases}b_f\left(p^2\right)&(d,p)=1 \\ 0&\text{otherwise},\end{cases}\ee
whence \be\frac1{\left|\F^+(X)\right|}\sum_{g\in\F^+(X)}b_g\left(p^2\right)
=\frac1{X^\ast}\sum_{0<d\leq X}b_{f\otimes\psi_d}\left(p^2\right)
=b_{f}\left(p^2\right)\frac{X^\ast_p}{X^\ast}.\ee
Let \be\Delta(X,p):=\frac{X^\ast_p}{X^\ast}.\ee $\Delta(X,p)$ attains positive values less
than 1 only. Then
\be\frac1{\left|\F^+(X)\right|}\sum_{f\in\F^+(X)}b_f\left(p^2\right)
=b_f\left(p^2\right)\Delta(X,p).\ee Therefore the left hand side of~\eqref{eq:symsniff}
simplifies to \be\label{eq:sniffapplied}-2\sum_p\frac{\Delta(N,p)}p\frac{\log p}{\log R_N}
\hat\phi\left(2\frac{\log p}{\log R_n}\right)b_f\left(p^2\right),\ee where again $f$ is the
fixed form we're twisting.
\begin{remark} Let $\mathcal Q_p(N)$ denote the number of square-free numbers that avoid
  a fixed prime $p$. Then, specializing a theorem of Cellarosi and
  Sinai~\cite[Theorem 2.3]{CS}, we have the asymptotic
  \be\left|\mathcal Q_p(N)\right|=\frac6{\pi^2}\frac p{p+1}N+O\left(N^{-1/2}\right),\ee
  where the implied constant depends on $p$. $\left|\mathcal Q(N)\right|$, the number of
  square-free numbers less than or equal to $N$, is asymptotically $(6/\pi^2)N$.\end{remark}
Since $d$ is a fundamental discriminant iff $\mathcal Q\ni d\equiv1\mod4$ or
$d=4m$, where $\mathcal Q\ni m\equiv2\text{ or }3\mod4$, it is reasonable to expect that
\be\Delta(N,p)\sim\frac6{\pi^2}\frac p{p+1}N\ee for fixed $p$.

Returning to~\eqref{eq:sniffapplied}, replacing $b_f\left(p^2\right)$ with its expectation
over primes, we achieve for the family of quadratic twists the same symmetries as for the
general families described in detail in the next section.

\subsection{Underlying symmetry of general families}
Let $\mathcal F^\circ$ now denote the family of all cusp forms with primitive
character of weight $k$ and level $N$ ($\circ=\heartsuit$), the family of all cusp forms
with non-primitive character of weight $k$ and level $N$ ($\circ=\diamondsuit$), or
the family of all CM cusp forms of weight $k$ and level $N$ ($\circ=\clubsuit$). In all
cases, let $N$ be our asymptotic parameter. Then, applying~\eqref{eq:shortsniff} is
successful. 

In the case of $\mathcal F^\heartsuit$, we have generically that
\be b_f\left(p^2\right)=a_f\left(p^2\right)-1,\ee and, as we average over the whole family,
orthogonality of Fourier coefficients produces $c_{\mathcal F^\heartsuit}=-1$.

In the case of $\mathcal F^\diamondsuit$, we have generically that
\be b_f\left(p^2\right)=a_f\left(p^2\right)-\chi_f,\ee and, as we average over the whole
family, orthogonality of Fourier coefficients and orthogonality of characters produces
$c_{\mathcal F^\diamondsuit}=0$.

In the case of $\mathcal F^\clubsuit$, we have generically that
\be b_f\left(p^2\right)=a_f\left(p^2\right)-\chi_f=a_f(p)^2-2\chi_f,\ee $\chi_f$ is
non-principal, and $a_f$ is real. As we average over the whole
family, orthogonality of characters means the contribution from $\chi_f$ is zero, while
orthogonality of Fourier coefficients means that the contribution from $a_f(p)^2$ is 1,
so $c_{\mathcal F^\clubsuit}=1$.



% \nocite{*}                      
\bibliographystyle{amsalpha}
\bibliography{central}

% \begin{thebibliography}{DHKMS}
% \bibitem[CFKRS]{recipe} CFKRS, \textit{Integral moments of $L$-functions.}
% \bibitem[CS]{CS} CS, \textit{Applications of the $L$-functions ratios conjectures.}
% \bibitem[DHKMS]{DHKMS} DHKMS, \textit{A random matrix model for elliptic
%     curve $L$-functions of finite conductor.}
% \bibitem[DM]{conv} DM, \textit{The effect of convolving families of L-functions on
%     the underlying group symmetries.}
% \bibitem[HKS]{HKS} HKS, \textit{Lower order terms for the one-level density
%     of elliptic curve $L$-functions.}
% \bibitem[HMM]{HMM} HMM, \textit{An elliptic curve test of the $L$-Functions
%     Ratios Conjecture.}
% \bibitem[IK]{IK} Iwaniec-Kowalski, \textit{Analytic number theory.}
% \bibitem[KZ]{KZ} Kohnen, Zagier, \textit{Values of $L$-series of Modular Forms
%     at the Center of the Critical Strip.}
% \bibitem[BM]{BM} Baruch, Mao, \textit{Central value of automorphic $L$-functions.}
% \bibitem[Ma]{mao} Mao, Zhengyu, \textit{A generalized Shimura correspondence
%     for newforms.}
% \bibitem[Ogg]{Ogg} Ogg, \textit{On a Convolution of $L$-Series.}
% \bibitem[MW]{MW} MW, \textit{Pôles des fonctions $L$ de paires pour $GL(N)$.}
%   (In \textit{Le spectre résiduel de $GL(N)$}.)
% \bibitem[MiNg]{simplezeros} Milinivich, Ng, \textit{Simple zeros of modular
%     $L$-functions.}
% \bibitem[De]{deligne} Deligne, P., \textit{La conjecture de Weil, I.}

% \bibitem[Mi06]{mil06} Miller, S.J. 2006 repulsion
% \bibitem[BBLM]{bblm} Bogomolny, Bohigas, Leboeuf, and Monastra -- the cool cats.
% \bibitem[Co]{conreymatrix} Conrey, Matrix Eigenvalues, in
%   \textit{Perspectives on RMT and NT}.
% \bibitem[CS]{cs} arXiv:1112.4691 [math.DS].
% \bibitem[CKRS1]{CKRS1} CKRS. On the frequency of vanishing of quadratic twists of modular
%   $L$-functions. arXiv:math/0012043 [math.NT]
% \bibitem[CKRS2]{CKRS2} CKRS. RMT and the Fourier coefficients of half-integral weight forms.
% \bibitem[KeSn1]{rmtzeta} Keating, Snaith. Random matrix theory and $\zeta(1/2+it)$.
% \bibitem[KeSn2]{rmtL} Keating, Sniath. Random matrix theory and $L$-functions at $s=1/2$.
% \bibitem[Ba]{barnes} Barnes. The theory of the $G$-function.
% \bibitem[Ko1]{kohnen} W. Kohnen, Fourier coefficients of modular forms of half-integral weight, Math. Ann. 271 (1985) 237--268.
% \bibitem[Wa1]{wald1} Waldspurger, Correspondance de Shimura, J. Math. Pures Appl. (9)
%   59 (1980) 1--113.
% \bibitem[Wa2]{wald2} Waldspurger, Sur les coefficients de Fourier des formes modulaires
%   de poids demi-entier.
% \bibitem[Iw]{iwan} Iwaniec, Classical Theory of Automorphic Forms.
% \bibitem[Ri]{ribet} Galois representations attached to eigenforms with Nebentypus.
%   In Modular functions of one variable, V
%   (Proc. Second Internat. Conf., Univ. Bonn, Bonn, 1976),
%   pages 17--51. Lecture Notes in Math., Vol. 601. Springer, Berlin, 1977
% \bibitem[Mi]{miyake} Miyake, Modular Forms.
% \bibitem[Mi08]{mil08} Miller, S.J. 2008 Symplectic test.
% \bibitem[us]{ours} Gaps between zeros of $GL(2)$ $L$-functions.
% \bibitem[Mo]{montgomery} Montgomery, \textit{The Pair Correlation of Zeros of the
% Zeta Function}
% \bibitem[He]{hejhal} Hejhal, \textit{On the triple correlation of zeros of the zeta function}
% \bibitem[RS]{RS} Rudnick, Sarnak, \textit{The n-level correlations of zeros of the zeta function}
% \bibitem[KaSa1]{KS1} Katz, Sarnak, \textit{Random Matrices, Frobenius Eigenvalues and Monodromy}
% \bibitem[KaSa2]{KS2} Katz, Sarnak, \textit{Zeros of zeta functions and symmetries}
% \bibitem[ILS]{ILS} Iwaniec, Luo, Sarnak, \textit{Low lying zeros of families of L-functions}
% \end{thebibliography}
\end{document}
